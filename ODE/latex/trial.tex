\documentclass{article}
\usepackage{amssymb}
\usepackage{graphicx}
\usepackage{caption}
\usepackage{subcaption}
\usepackage{listings}
\usepackage{float} %figure inside minipage
\graphicspath{ {./images/} }
\usepackage[export]{adjustbox}
\usepackage{apacite}




\begin{document}
\begin{titlepage}
\author{Humaid Agowun and Paul Muir} 
\title{Performance Analysis of a Collection of ODE Solvers on Covid-19 Models with Discontinuities 
} 
\date{\today} 
\maketitle
\end{titlepage}

\begin{center}
    \textbf{Abstract}
\end{center}

In this report, we consider the numerical solution of two challenging
Covid-19 ordinary differential equation (ODE) 
models that have discontinuities. One of the models has a 
time-dependent, i.e., at a given point in time, a discontinuity is introduced
into the model. The second type of discontinuity is a state-dependent discontinuity;
this means that the time at which the discontinuity arises depends on the
value of one of the solution components, and thus it is not known apriori.

These discontinuities make the models quite challenging for standard ODE solvers.
Furthermore, the presence of exponentially growing solution components adds further
to the difficulties faced by standard ODE solvers when trying to solve these types of 
problems.

We report on an investigation of performance of a wide array of ODE solvers (we
consider 21 solvers) available in four major software environments: R, Python, 
Scilab, and Matlab, applied to these models.

We focus on straightforward implementations of the models within the above environments
where the user employs the solvers and codes the problems using default settings for 
the solvers, e.g., default tolerances, and simple implementations for the
discontinuities, e.g., the introduction of if-then statements into the functions that 
define the right-hand sides of the ODE systems. Such implementations of the models
and usage of the solvers are typical of what researchers might employ.

We also include an investigation of the solution of the models using slightly more
sophisticated, but easily implementable, treatments of the models; these treatments
involve making better use of the capabilities of the solvers, and slightly more careful
implementations of the models themselves.

We also highlight a number of issues with the way that some of the solvers are implemented
in some of the software environments. For example, the way in which output points, where 
the user specifies points in the domain where the solution value should be provided,
is an issue in some of the software environments. 

{\it We show that the standard use of ODE solvers available within  widely used software 
environments, applied to simple implementations of these Covid-19 models, will
frequently deliver numerical solutions that have no significant digits of accuracy.
Furthermore, the solvers give no indication that the return solutions
may be inaccurate.} We also show that the straightforward treatments of the models 
are always less efficient than the slightly more sophisticated treatments.
{\it We show that the slightly more sophisticated treatments of the models can 
result in more efficient computations while at the same time providing
much more accurate solutions.}

\section{Introduction}
\label{section:intro}
In this report, we will discuss the results of a careful investigation of the performance of a variety of software packages applied to typical initial value ordinary differential equation (IVODEs) encountered in Covid-19 models. 

For any mathematical model, the accuracy requirements of the numerical solution should be determined by the quality of the model and the accuracy of the parameters that appear in the model. Numerical errors associated with the computational techniques that are used to obtain the approximate solution must always be negligible compared to the accuracy to which the model is defined. \emph{Researchers deserve to obtain numerically accurate solutions to the models that they are studying}. In this report, \emph{we will show that the straightforward use of standard IVODE solvers on typical Covid-19 models can lead to numerical solutions that have large errors, sometimes of the same order of magnitude as the solution itself.} Most of the IVODE solvers that we consider in this report allow the user to specify a parameter called a tolerance. The solvers use adaptive algorithms to attempt to compute an approximate solution with a corresponding error estimate that is approximately equal to the tolerance.

In Section $\ref{subsection:research_papers}$, we review examples of how IVODEs are used in epidemiology. In Section $\ref{subsection:SEIR_model}$, we define the SEIR models which we will consider throughout this report. In Section $\ref{subsection:exponential_growth}$, we discuss the numerical stability issues that arise in problems (such as Covid-19 modelling) with exponentially growing solutions. In Section $\ref{subsection:fixed_vs_control}$, we explain the difference between fixed step-size and error-controlled IVODE solvers. The IVODE software packages from programming environments that are typically used by researchers are described in Section $\ref{subsection:numerical_software_used}$. We also make a note of issues with evaluation of approximate solutions at output points that lead to inefficiencies for some of these solvers in Section $\ref{subsection:solution_output_points_impl}$. In Section $\ref{subsection:effect_of_discontinuity}$, we discuss the effects of problem discontinuities on the performance of these solvers.

In Section $\ref{subsection:naive_time_problem}$, we apply the solvers to a Covid-19 problem with a time-dependent discontinuity and show how, in some case, this results in numerical solutions with relative errors of the same magnitude as the solution being computed. In Section $\ref{subsection:time_disc_handling}$, we will use discontinuity handling to accurately solve the time-dependent discontinuity problem. In Section $\ref{subsection:time_tolerance_study}$, we will use a range of tolerances to discuss the effects of tolerance on the accuracy and efficiency of some of the solvers.

In Section $\ref{subsection:naive_state_problem}$, we apply the solvers to a Covid-19 problem with a state-dependent discontinuity and show how when using a straightforward implementation of the problem, none of the solvers are able to obtain accurate solutions. We will explain how even the use of very sharp tolerances is not sufficient to improve the computed solutions in Section $\ref{subsection:state_sharp_tol_failed}$ and show that a more effective way to solve this problem is through the use of event detection, which we will describe in Section $\ref{subsection:intro_event_detection}$. We then show an accurate solution to the state-dependent discontinuity problem in Section $\ref{subsection:state_with_event_detection}$ and perform a tolerance study on this problem in Section $\ref{subsection:state_tolerance_study}$.

In Section $\ref{section:fortran_inaccuracies}$, we examine implementation details for solvers with exceptionally poor solutions to investigate the cause of their errors. We conclude the report in Section $\ref{section:summary}$ with a summary and a discussion of the potential for future work projects.

\subsection{Epidemiological modelling}
\label{subsection:research_papers}
One common form of an epidemiological study is forecasting. Using previously obtained parameters, the researcher develops a mathematical model involving differential equations which are solved using an ODE solver. Often, the solver will be used to integrate over a large time period so that the researcher can examine how the disease will spread. In Section $\ref{subsection:exponential_growth}$, we discuss why it is unrealistic to attempt to compute a numerical solution for large time periods if the infection is still growing exponentially and how measures such as social distancing allow solvers to reduce errors so that reasonably accurate solutions can be computed over longer time periods.

A second type of epidemiology study involves parameter estimation. In this kind of study, data points are collected about the spread of a virus and we try to fit a mathematical model to that data. In so doing, we can estimate values for some modelling parameters that will minimize the error in the fit. An example of such a study can be found in Appendix $\ref{section:ebola_paper}$. Parameter estimation studies often involve using an ODE solver inside an optimization algorithm and thus the computing time, especially for large problems, can be significant. Since the computational cost is typically inversely proportional to the tolerance, we will investigate to what extent coarse tolerances can be employed in the computation of solutions to Covid-19 models.

\subsection{Detailed description of two specific models to be considered in this report.} 
\label{subsection:SEIR_model}
In this section, we explain how an IVODE problem is defined. We then describe the models that we are going to consider in this report. They involve typical SEIR models to which we add discontinuities.

An IVODE problem is defined by the equations and the initial conditions:
\begin{equation}
y'(t) = f(t, y(t)), \quad y(t_0) = y_0 \nonumber
\end{equation}
where $f(t, y(t))$ is a function that defines the derivative at time, t. A complete definition also includes the initial values of the solution components. Given $f(t, y(t))$ and $y(t_0)$, the goal is to find an approximation to  $y(t)$ using numerical methods. 

In this report, we consider the Covid-19 model:
\begin{equation}
\frac{\textit{d}S}{\textit{dt}} = \mu N - \mu S - \frac{\beta}{N}IS, \nonumber
\end{equation}

\begin{equation}
\frac{\textit{d}E}{\textit{dt}} = \frac{\beta}{N}IS - \alpha E - \mu E, \nonumber
\end{equation}

\begin{equation}
\frac{\textit{d}I}{\textit{dt}} = \alpha E - \gamma I - \mu I, \nonumber
\end{equation}

\begin{equation}
\frac{\textit{d}R}{\textit{dt}} = \gamma I - \mu R \nonumber
\end{equation} 

In this SEIR model, we describe the epidemic over time. S is the number of susceptible individuals, E is the number of exposed individuals, I is the number of infected individuals and R is the number of recovered individuals at a point in time. We also use N to represent the population size.
The other parameters in this model are as follows: $\alpha$ is such that $\alpha^{-1}$ is the average incubation period, $\beta$ is the transmission rate, $\gamma$ is the recovery rate and $\mu$ is the birth/death rate. In this report, we assume that all these parameters are known. Our goal is to investigate the performance of IVODE solvers on forms of this problem that has discontinuities. We will see that we can get approximate solutions that are not efficiently computed and/or that may have significant errors. This latter issue can have serious consequences as the computed solution will fail to show the actual impact of the virus corresponding to the actual epidemiology theories behind the mathematical models. These incorrect numerical solutions may lead epidemiologists into reaching incorrect conclusions and thus lead them into questioning the mathematical models themselves when, in fact, it is the solvers that are at fault.

The discontinuities we are going to consider involve the parameter $\beta$.
Before measures such as social distancing, masking, vaccinations, etc., are implemented, $\beta$ has a much higher value than after the measures are introduced. For the purpose of this study, we will use a large $\beta$ value equal to 0.9 before the measures and a small $\beta$ value equal to 0.005 after they are implemented, corresponding to a highly contagious variant and extreme shut down measures, respectively. These choices will come to highlight the different numerical issues as such an abrupt change in a modelling parameter introduces a discontinuity as we will show in Section $\ref{subsection:effect_of_discontinuity}$. We will consider two types of discontinuities. One depends only on $t$; the other depends on the value of one of the solution components. We will refer to the former as a time-dependent discontinuity and the latter as a state-dependent discontinuity.

For the time-dependent discontinuity, we will assume that at some point in time, measures are implemented that will lead to a reduction in the parameter $\beta$. We would like to solve the problem through this discontinuity but as we will show, this discontinuity introduces a numerical issue.

For the state-dependent discontinuity, we consider the following situation. If the population of exposed people reaches a certain maximum threshold, measures are introduced, which decreases the value of $\beta$. This introduces a discontinuity. Then, when the population of exposed people drops below a certain minimum threshold, the measures are relaxed, which increases $\beta$ back to its original value, which introduces another discontinuity. We will try to model this problem through multiple instances of shut-downs followed by periods where measures are relaxed. We consider a case where vaccines are not being used. This leads to setting $\beta$ back to its original value when the measures are removed. We note that each time we change the parameter $\beta$, a discontinuity is introduced and thus this problem is far more discontinuous than the previous one, which had only one discontinuity. For this problem, we show that all the solvers will fail.

The other parameters are assumed to be constant with N = 37,741,000 (the approximate Canadian population size), $\alpha$ = 1/8, $\gamma$ = 0.06, and $\mu$ = 0.01/365. The initial values are E(0) = 103, I(0) = 1, R(0) = 0 and S(0) = N - E(0) - I(0) - R(0). This gives us a complete system of IVODEs that is in a form that can be solved by typical software packages.

\subsection{Exponential growth and the issue of instability for ODEs}
\label{subsection:exponential_growth}
Some of the solution components of the SEIR model exhibit exponential growth over certain time periods. In this section, we discuss exponentially growing solutions and their impact on the accurate computation of a numerical solution. First of all, we give a quick overview of stability for ODEs. Then we will show that the SEIR model is unstable over certain time intervals and how measures such as social distancing can improve the stability. This is important as this essentially means that before measures are implemented, accurate models are for the most part very difficult to obtain but the addition of the measures such as social distancing can lead to the solvers being able to compute more accurate solutions.

The stability of an ODE is associated with the impact of small changes to the initial values of the solution to the problem. An ODE is unstable if a small change in the initial values results in a large change in the solution; otherwise, the ODE is said to be stable.

It is straightforward to see that problems with a solution component that exhibits exponential growth are unstable. As mentioned above, this is the case with some of the solution components of a Covid-19 model. The population of infected people, $I$, grows exponentially as long as no measures are introduced to reduce the spread of the virus. This means that ODE solvers will experience difficulties in obtaining accurate numerical solutions. 

In Figure $\ref{fig:unstability_of_exponential_growth}$, we show exponentially growing solutions corresponding to models with slightly different initial values for E(0). We can see that we get different solutions, that become even more different as time increases.

\begin{figure}[H]
\centering
\includegraphics[width=0.7\linewidth]{./figures/unstability_of_exponential_growth}
\caption{When a solution exhibits exponential growth, relatively small changes in the initial value can eventually lead to much different solution values. Here we assign the initial of only the E-component to 70, 80, .., 120.}
\label{fig:unstability_of_exponential_growth}
\end{figure}

However, when we introduce measures such as social distancing, which corresponds to a smaller $\beta$ value, the solution will exhibit slower exponential growth or can even show exponential decay. A slower exponential growth means that the solution will not be as sensitive to changes to the initial values. Exponential decay is even better as the solutions from different initial values will converge.

Epidemic modeling problems exhibit solutions with this type of behavior. At first, the problem is unstable but as measures are implemented, which lead to exponential decay rather than growth, the problem becomes stable. We show this in Figure $\ref{fig:regain_stability_after_measures}$ for the problem with the time-dependent discontinuity. At first, the solutions diverge when there is exponential growth, but the introduction of measures such as social distancing introduce exponential decay which makes them converge. Thus the measures not only save lives but also improve the capability of solvers to compute accurate solutions.

\begin{figure}[H]
\centering
\includegraphics[width=0.7\linewidth]{./figures/regain_stability_after_measures}
\caption{Unstable solutions in the region [0, 40] becomes stable in the region [40. 90] as measures are implemented.}
\label{fig:regain_stability_after_measures}
\end{figure}

\subsection{Brief overview of numerical software}
We start by explaining how typical solvers attempt to solve an IVODE problem. Given initial values (at the initial time, $t_0$), the solver will use an initial step size, $h$, to compute a solution at time, $t_1 (= t_0 + h)$. Similarly, the solver will attempt to take a sequence of steps until it reaches the end time. High-quality solvers will also employ an interpolation algorithm usually locally within each step to get a continuous numerical solution. We note that a solver is said to have order $p$ if the difference between the true solution and the computed solution is $O(h^p)$.

In the next section, we describe what a solver will attempt to do to improve the accuracy of the computed solution. We then discuss the numerical solvers we are going to use throughout our investigation. We will then provide an additional discussion on the implementation of interpolation to get a continuous numerical solution and how certain programming environments have not set up their ODE solvers to use interpolation (correctly).

\subsubsection{Fixed Step Size and Error Control Solvers}
\label{subsection:fixed_vs_control}
In this section, we explain the role of the tolerance and the difference between fixed step size and adaptive step-size error control solvers.

The tolerance is a measure of how accurate we want the solution computed by the solvers to be. A key point here is that solvers that can take a tolerance as input must have some way of computing an estimate of the error of the solution that they compute. Then that error estimate can be compared with the user-provided tolerance. Generally, an absolute tolerance means that we want the error estimate to be approximately equal to the tolerance, whereas a relative tolerance means that we want the ratio of the error estimate and the computed solution to be approximately equal to the tolerance. This is not always the case as some solvers will use a blended combination of the provided absolute and relative tolerances.

A solver is said to have a fixed step size if the solver begins with an initial step-size and this step-size is used throughout the whole integration. In this case, the solver will step from one point to the next and will not check if the numerical solution it obtains at the end of each step is sufficiently accurate. Thus, the distance between the points, i.e, the step size, is constant throughout the computation.

An error-controlled solver starts with an initial step size but as it takes a step, it will compute an error estimate and, based on the tolerances will repeat the computation with a smaller step-size if the error estimate is larger than the tolerance. It will repeat this process until the error estimate satisfies the given tolerance. Only then will it move to the next step. Thus it reduces the step-size as needed throughout the computation. We note that the error depends on the step-size and that a smaller step-size generally leads to a smaller error. However, a small step-size means that the computation is slower because more steps will be needed and thus if the error estimate is much smaller than the tolerance, a solver will increase the step-size for the next step. This allows it to make sure that the given tolerance is satisfied over the whole problem interval and that as large a step as possible is being taken to optimize the efficiency of the computation.

Error control is not simple to implement. Some researchers may be tempted to write their own solvers. based on a non-error control method like a simple fixed step-size Euler or Runge-Kutta method. We will show, using provided fixed step-size solvers in R, how these solvers simply cannot solve a Covid-19 model with reasonable accuracy. Without error control, these solvers cannot handle the discontinuity and stability issues that are present in these models and they will give very erroneous solutions, often without even a warning that the computed solutions should not be trusted.

\subsubsection{The ODE Solvers}
\label{subsection:numerical_software_used}
The ODE solvers are grouped in the following classes: Runge-Kutta methods, Runge-Kutta pairs, and multi-step methods.

A Runge-Kutta method is a one-step method that uses function evaluations, i.e, evaluations of $f(t, y(t))$, within the step. A solver based on this type of method integrates with a fixed step-size and has no error control. An example is the classical four-stage, fourth-order Runge-Kutta method.

A Runge-Kutta pair uses two Runge-Kutta methods. One of the methods is used to compute a solution and the second method is used to compute an error estimate. A solver that is based on a Runge-Kutta pair resizes the step based on the error estimate, as discussed previously. An example of such a solver is the DOPRI5 solver that uses a fifth-order method for the solution and a fourth-order method for the error estimate.

A multi-step method is a solver that will use a linear combination of the solutions and function values from the current and previous steps to take the next step. An example of such a solver is LSODA. Such solvers obtain an error estimate using two different multi-step methods. They use the error-estimate to control the step-size as discussed above.

\subparagraph{R packages}
Scientists who solve ODE models in R commonly use the deSolve package, \cite{soetaert2010solving}, and the $ode()$ function within it.
$ode()$ provides many numerical methods to solve a problem but we have focused our investigation only on the following popular choices: `lsoda', `daspk', `euler', `rk4', `ode45', `Radau', `bdf' and `adams'. The default method is `lsoda' and the default tolerances are $10^{-6}$ for both the absolute and relative tolerances. We also note that we did not consider the other integrators in the deSolve package like $rkMethod()$, which provides other Runge-Kutta methods, and the other methods which are called by the $ode()$ function itself.

The error control solvers are:
\begin{itemize}
\item `lsoda' calls the Fortran LSODA routine from ODEPACK. It can automatically detect stiffness and choose between a stiff (Backward Differentiation Formula, BDF) and a non-stiff (Adams) solver.

\item `daspk' calls the Fortran DAE solver of the same name.

\item `ode45' calls an implementation of Dormand-Prince (4)5 (DOPRI5) Runge-Kutta pair, written in C.

\item `Radau' calls the Fortran solver RADAU5 which implements the 3-stage RADAU IIA method.

\item `bdf' calls the stiff solver inside the Fortran LSODE package which is based on a family of BDF methods.

\item `adams' calls the non-stiff solver inside the Fortran LSODE package which is based on a family of Adams methods.
\end{itemize}

The fixed step-size solvers are:
\begin{itemize}
\item `euler' calls the classical Euler method which is implemented in C.
\item `rk4' uses the classical Runge-Kutta method of order 4 which is implemented in C. 
\end{itemize}

We will use these two methods to demonstrate what happens when non-error-controlled solvers are applied to Covid-19 models.

We next consider the R interface for handling output. The $ode()$ function is only given a vector of output points.  The function will use an interpolation by default but the interpolation schemes for all the solvers are not implemented in the most efficient way. As a result, the vector of output points affects the efficiency of the solver in a manner as described in Section $\ref{subsection:solution_output_points_impl}$.

\subparagraph{Python packages}
In Python, researchers can use the scipy.integrate package \cite{2020SciPy-NMeth}, and will normally use the $solve\_ivp()$ function due to its newer interface. It lets the user apply the following methods: `RK23', `RK45', `DOP853', `Radau', `BDF' and 'LSODA`. In this report, we will investigate all of these methods. The default solver in $solve\_ivp()$ is `RK45' and the default tolerance is $10^{-3}$ for the relative tolerance and $10^{-6}$ for the absolute tolerance. All of these solvers employ some form of error control:

\begin{itemize}
\item `RK23' uses an explicit Runge-Kutta pair of order 3(2), the Bogacki-Shampine pair of formulas. It is a Python implementation.

\item `RK45' uses the DOPRI5 pair of formulas, an explicit Runge-Kutta pair of order 5(4). It is a Python implementation.

\item `DOP853' uses an explicit Runge-Kutta triple of order 8(5, 3). It is a Python implementation.

\item `Radau' uses the implicit Radau IIA method of order 5. It is a Python implementation of the RADAU5 Fortran solver.

\item `BDF' uses a method based on BDF methods with the order varying automatically from 1 to 5. It is a Python implementation.

\item `LSODA' calls the Fortran LSODA routine from ODEPACK. It can automatically detect stiffness and choose between a stiff (BDF) and a non-stiff (Adams) solver.
\end{itemize}

We note that all solvers in $solve\_ivp()$ have error control and that only 'LSODA' is using the Fortran package itself; the others are a Python implementation and will likely be slower.

We next talk about Python's $solve\_ivp()$ interface. It can integrate using only the initial time and the final time and it will return the output at the end of each successful step. It can also take a $t\_eval$ vector of specified output points. The solver is allowed to take as big a step as needed and required solution approximations are obtained using interpolation. Thus it does not suffer from the inefficiencies described in Section $\ref{subsection:solution_output_points_impl}$. The interface also has a $dense\_output$ flag. This returns an interpolant for the solution over the whole time range.

\subparagraph{Scilab packages}
In Scilab, researchers solve differential equations using a method from the $ode()$ function, \cite{campbell2010modeling}, which has the following methods: `lsoda', `adams', `stiff', `rk', `rkf'. The default integrator is `lsoda'.
Default values for the tolerances are $10^{-5}$ for the relative tolerance and $10^{-7}$ for the absolute tolerance for all solvers used except `rkf' for which the relative tolerance is $10^{-3}$ and the absolute tolerance is $10^{-4}$. All of these solvers are error control solvers.

\begin{itemize}
\item `lsoda' calls the Fortran LSODA routine from ODEPACK. It can automatically detect stiffness and choose between a stiff (BDF) and a non-stiff (Adams) solver.

\item `stiff' calls the stiff solver inside the Fortran LSODE package which is based on a family of BDF methods.

\item `adams' calls the non-stiff solver inside the Fortran LSODE package which is based on a family of Adams methods.

\item `rk' calls an adaptive Runge-Kutta method of order 4. It uses Richardson extrapolation for the error estimation. It is implemented in Fortran in a program called `rkqc.f'.

\item `rkf' calls the Fortran program written by Shampine and Watts based on Fehlberg's Runge-Kutta pair of order 4 and 5 (RKF45) pair. It is implemented in a Fortran program called `rkf45.f'.
\end{itemize}

The $ode()$ function in Scilab takes a vector of time steps and the code uses interpolation or stops the integration at the output points as described in $\ref{subsection:solution_output_points_impl}$ based on the method used. For example Scilab's `rkf' is an interface to an old software package, `rkf45.f' which does not have any interpolation capabilities.

\subparagraph{Matlab packages}
In Matlab, researchers can solve differential equations with the $ode()$ $suite$ \cite{shampine1997matlab} of functions. We will consider two of these: $ode45()$ and $ode15s()$.
Default values for the tolerances are $10^{-3}$ for the relative tolerance and $10^{-6}$ for the absolute tolerance.

\begin{itemize}
\item Using $ode45()$ calls a Matlab implementation of DOPRI5.

\item Using $ode15s()$ employs an algorithm that is a variable-step, variable-order (VSVO) solver based on the numerical differentiation formulas (NDFs) of orders 1 to 5. Optionally, it can use BDF methods but these are usually less efficient.
\end{itemize}

Functions in the $ode$ $suite$ takes the initial and final time only and thus allows a solver to take as big a step as needed. All plots are then done using the plot() function. With such an interface, it does not suffer from the issues discussed in Section $\ref{subsection:solution_output_points_impl}$. 

\subparagraph{How the packages relate}
We tried to find connections across the programming environment where the solvers appear to be using the same source code.
Here is what we found:

In R, Python, and Scilab, the `lsoda' method is a wrapper around the Fortran LSODA code from ODEPACK.

The R `bdf' method is equivalent to the Scilab `stiff' method in that they both use the LSODE code from ODEPACK; however, the Python `BDF' method is a different implementation in Python itself.

The R `adams' method and the Scilab `adams' method are the same since they both use the LSODE code from ODEPACK.

The R and Python Runge Kutta 5(4) pairs are both implementations of DOPRI5 but they have different source code as the version in Python is implemented in Python while the R version is implemented in C. The $ode45()$ function in Matlab is a Matlab implementation of DOPRI5. The Scilab `rkf' method does not use the same pair; it uses the Shampine and Watts implementation of the Fehlberg's Runge-Kutta pair, not the Dormand-Prince pair. 

The Scilab `rk' method, which is of order 4, and the R `rk4' method are not the same solvers. The Scilab `rk' method is adaptive (error-controlled with Richardson extrapolation for the error estimate) whereas the R `rk4' method is a fixed step-size implementation of the classical 4-stage, 4$^{th}$ order Runge-Kutta method.

The R and Python `Radau' methods have different source code as Python implements its own version of RADAU5 while R calls the Fortran code for RADAU5 through a C interface.

\subsection{Observations on obtaining solution approximations at output points}
\label{subsection:solution_output_points_impl}
In this section, we discuss an issue that we encountered with some of the ODE solvers in R and Scilab when it comes to obtaining output. In an ideal scenario, the user's desired output points should not interfere with the efficiency of the solvers. However, in these two platforms, a method for handling output points is used which makes treating a lot of output points very inefficient.

A standard ODE solver works as follows. Using a default initial step-size, the solver will take a step. It will then accept or reject the step based on the tolerance and will adjust the step-size based on this to take the next step or retake the step. This process is repeated until the solver reaches the end of the interval. However, often the users of an ODE solver will require outputs at specific points and these points may lie at points that are internal to the steps. The current state-of-the-art approach to get solution approximations at these output points is to perform a high accuracy interpolation on the given step and to return the value of the interpolant at the required point. The interpolation error is usually at least of order $p$ if the numerical ODE solution is of order $p$. However some solvers use a lower order interpolant. This way the accuracy of the solution approximation at a point that is interior to a step should be comparable to the accuracy of the solution approximation at the end of the step.

Note that the standard ODE solvers only control the error at the end of the step. That is, an error estimate is generated for the solution approximation at the end of the step and the step is accepted if this error estimate satisfies the tolerance. It is hoped that the solution approximations obtained through the use of the interpolant will be of comparable accuracy to the solution approximation at the end of the step. It is typically the case that no error control is actually applied to the continuous solution approximation.

In R and Scilab, the above approach for handling output points is not used in all the solvers. Instead, R and Scilab will use the output points to dictate the step-size. An issue arises when many output points appear between the steps that would normally be taken by the solver. These solvers will use the difference between the current output point and the next output point to determine the step-size. We note that R solvers, like its `ode45' method, does have an interpolant but that their implementation still allows the user defined output points to affect the efficiency of the solver. 

In such approaches, the space between points will limit the maximum step-size that can be taken and will lead to additional function evaluations being performed because the solver needs to compute a solution approximation using the numerical method at each output point. This will lead to a considerable drop in efficiency as we will show; see for example Tables $\ref{tab:tolerance_time_discontinuity_rk45_R}$ and $\ref{tab:tolerance_time_discontinuity_rk45_further_R}$. These tables show that a problem that can be solved with 150 function evaluations will be solved with 500 function evaluations when there are many output points. 

This method of handling output points in which the solver steps to each output point and uses the numerical method itself to compute a solution approximation also means that the accuracy of the solution depends on the space between the output points. Thus, we get the unusual behavior that the accuracy is increased by putting the output points closer together and the accuracy is decreased by putting them further apart. We will point out these inconsistencies as they become relevant later in this report. We also note that spacing the points closer together is not a good way to control the accuracy as it is impossible to know beforehand how close the points should be in order to obtain a desired accuracy.

\begin{figure}[H]
\centering
\includegraphics[width=0.7\linewidth]{./figures/R_ode45_spacing_experiment}
\caption{R `ode45' output point spacing experiment}
\label{fig:ode45_spacing_experiment}
\end{figure}

Figure $\ref{fig:ode45_spacing_experiment}$ shows an experiment where we solve the time-dependent discontinuity Covid-19 problem using the R `ode45' method, which is an implementation of DOPRI5 which has error control, uses interpolation but allows the output points to affect the integration. We set both the absolute and relative tolerance to 0.1 and thus expect low accuracy but very good efficiency. However, the space between the output points becomes the limiting factor for the step-size. The computed solution has undue accuracy and is computed in a very inefficient manner considering the required tolerance. We recorded the number of function evaluations in Table $\ref{tab:ode45_spacing_experiment}$ and it can be seen that the solver is using a lot more function evaluations than are needed to satisfy such a coarse tolerance. In Table $\ref{tab:ode45_spacing_experiment}$, `spacing' refers to the distance between the output points and `nfev' is the number of function evaluations.

\begin{table}[h]
\caption {R DOPRI5 output point spacing experiment} \label{tab:ode45_spacing_experiment} 
\begin{center}
\begin{tabular}{ c c }
spacing & nfev \\ 
1 & 572 \\
3 & 188 \\
5 & 116 \\
7 & 80 \\
\end{tabular}
\end{center}
\end{table}

From Figure $\ref{fig:ode45_spacing_experiment}$ and Table $\ref{tab:ode45_spacing_experiment}$, we note that we did not ask the solver for an accurate solution but it is giving us a solution that is much more accurate than requested when the spacing between the output points is small. This excess accuracy comes at a price of around 500 more function evaluations. Accuracy should ideally be completely determined by the tolerance but using this method of skipping to the output points substantially interferes with this ideal. This results in the solver not being allowed to take as big a step as it should be based on the tolerance, and this leads to substantial inefficiency. 

It is important that users employ the interpolation option for an ODE solver whenever such an option is readily available so that the solvers can run as efficiently as possible. We also reiterate that the interpolant should have an interpolation error that is at least of order $p$ if the ODE solver gives a solution with an error that is of order $p$ so that the interpolation error does not interfere with the accuracy of the numerical solution.

\subsection{Discontinuities and their effects on solvers}
\label{subsection:effect_of_discontinuity}
The main purpose of this report is to discuss how to solve models with discontinuities and how these discontinuities affect the process of computing an accurate numerical solution to the model. In this section, we will show what happens when a solver encounters a discontinuity and how this discontinuity leads to inaccurate solutions.

We first note that one of the core assumptions for all the solvers is that the function $f(t, y(t))$ and a sufficient number of its higher derivatives are continuous. If the right-hand side function is discontinuous, this can have a major (negative) impact on the performance and accuracy of the solvers. 

We will see that discontinuities will have huge impacts on the accuracy and efficiency of the solvers, that some solvers, even with error control, will require an extremely sharp tolerance to step over the discontinuity in a way that allows them to obtain a reasonably accurate solution approximation, and that fixed-step solvers simply cannot solve these problems accurately. 

It is important to note that the step taken by a solver that first meets a discontinuity will almost always fail. This is because in order for the solver to step over a discontinuity, the step size needs to be much smaller than the one that is being used before the discontinuity. The solver will thus have to retake the step with a smaller step size and as long as the error estimate of the step is not small enough, it will need to continue reducing the step. This leads to high numbers of function evaluations near the discontinuity. 

\begin{figure}[h]
\centering
\includegraphics[width=0.7\linewidth]{./figures/lsoda_vs_discontinuity}
\caption{Function evaluations for the Python `LSODA' method for the time-dependent discontinuity problem with a discontinuity at t=27}
\label{fig:lsoda_vs_discontinuity}
\end{figure}

\begin{figure}[h]
\centering
\includegraphics[width=0.7\linewidth]{./figures/dop853_vs_discontinuity}
\caption{Function evaluations for the Python `DOP853' method for the time-dependent discontinuity problem with a discontinuity at t=27}
\label{fig:dop853_vs_discontinuity}
\end{figure}

In Figures $\ref{fig:lsoda_vs_discontinuity}$ and $\ref{fig:dop853_vs_discontinuity}$, we run `LSODA' and `DOP853' from Python on the time-dependent discontinuity problem where the discontinuity is introduced at t=27 and plot the time at which the $i^{th}$ function evaluation occurs. We thus show the spike in the number of function evaluations at the discontinuity as the solvers repeatedly retake the step with smaller and smaller step-sizes.

Following from the above discussion, we also recommend researchers carry out a manual discontinuity detection experiment to see if their model has any discontinuity. For the case where it is not known if a discontinuity is present, a trivial experiment is done by collecting at what time the solver made the $i^{th}$ call to the function that evaluates the right hand side of the ODE. When a plot of the time against the cumulative count of the function calls gives an almost straight vertical line, it indicates that the function was called repeatedly at a specific time and thus that the solver repeatedly changed the step-size in this region to step over a discontinuity. In the remainder of this report, we will outline the ways to accurately and efficiently solve problems with such discontinuities.


\section{Time Dependent discontinuity problem}
\label{section:time_problem}
In the time dependent disconinuity problem, we change the value of the parameter $\beta$ from 0.9 to 0.005 at time 27. This introduces a discontinuity in the problem. We will show that this leads to inaccuracies, especially with fixed step solvers. We then introduce a form of discontinuity handling using cold starts to show an efficient way to solve time dependent discontinuity problems.

\subsection{Naive treatment of Covid-19 time discontinuity models}
\label{subsection:naive_time_problem}
A naive implementation of the problem is to use an if-statement inside the right hand side function, $f(t, y)$, to implement the changes in $\beta$ as measures are implemented. An if-statement makes the function f(t, y) and its derivatives discontinuous. This introduces problems as outlined in Section $\ref{subsection:effect_of_discontinuity}$.

For this report, we will add a time-dependent discontinuity by using an if statement at time 27, where the parameter $\beta$ will change from 0.9 to 0.005, indicating that measures are implemented. In pseudo code, this looks like:

\begin{minipage}{\linewidth}
\begin{lstlisting}[language=Python]
function model_with_if(t, y)
    (S, E, I, R) = y
    beta = 0.005
    if t < 27:
        beta = 0.9
           	
    // code to get (dSdt, dEdt, dIdt, dRdt)
    return (dSdt, dEdt, dIdt, dRdt)
\end{lstlisting}
\end{minipage}

Also, to stay true to a naive treatment, we will always use the default tolerances in this section. Discrepancies across the programming environments that can be due to tolerance issues are investigate in Section $\ref{subsection:time_tolerance_study}$.
 
\subsubsection{Time discontinuity in R}
\begin{figure}[h]
	\centering
	\includegraphics[width=0.7\linewidth]{./figures/time_discontinuity_R}
	\caption{Time Discontinuity in R}
	\label{fig:time_discontinuity_R}
\end{figure}
From Figure $\ref{fig:time_discontinuity_R}$, we can see that all the methods except 'euler' and 'rk4' are on the same line. 'rk4' is somehow close to the actual solution but 'euler' was completely wrong. We note that all the other methods had error control while these two are fixed step-size solvers.

We also note that 'rk4' is doing better than 'euler' for this specific problem as it has a higher order. But the way it is performing is still better than expected. We prove in another experiment that this is entirely because of the outputting problem discussed in Section $\ref{subsection:solution_output_points_impl}$. In fact, if we use a bigger step-size, 'rk4' gives results which are as bad as 'euler'. Figure $\ref{fig:rk4_messing_up_no_event_R}$ shows that experiment with 'rk4' at different initial step-size (space between the output points) plotted against a graphically accurate solution in red. We can see that as soon as we change the step-size for 'rk4', it does not give good results at all. Analysing the source for 'rk4' and 'euler' shows that it picks the step size using the output points requested. Spacing out the output points affect the step-size which affects the accuracy of the fixed step-size solver.

A solution to this problem of wanting to use 'rk4', 'euler' or any user-programmed solver would be to have a small initial step-size but we cannot know beforehand how small is small enough. A better solution would be to not use fixed-step size solvers, be it from the packages or user-implemented ones. Reputable methods with error control should be preferred as we have shown that these solvers can step over one discontinuity by resizing the step repeatedly, as we have seen is needed in Section $\ref{subsection:effect_of_discontinuity}$.

\begin{figure}[h]
	\centering
	\includegraphics[width=0.7\linewidth]{./figures/rk4_messing_up_no_event_R}
	\caption{R's rk4 with bigger step-size}
	\label{fig:rk4_messing_up_no_event_R}
\end{figure}


\subsubsection{Time discontinuity in Python}
\begin{figure}[h]
	\centering
	\includegraphics[width=0.7\linewidth]{./figures/time_discontinuity_py}
	\caption{Time Discontinuity in Python}
	\label{fig:time_discontinuity_py}
\end{figure}
From Figure $\ref{fig:time_discontinuity_py}$, we can see that all the methods in Python's $solve\_ivp()$ work correctly. There is some blurring at the peak but all the methods are along the same line. Python only provides error-controlled packages and thus we can see that error-control is all that is needed to step over one discontinuity. This observation also leads us to another conclusion that sharp tolerance with an error-control solution is what is required to step over one discontinuity.

\subsubsection{Time discontinuity in Scilab}
\begin{figure}[h]
	\centering
	\includegraphics[width=0.7\linewidth]{./figures/time_discontinuity_scilab}
	\caption{Time Discontinuity in Scilab}
	\label{fig:time_discontinuity_scilab}
\end{figure}
From Figure $\ref{fig:time_discontinuity_scilab}$, in Scilab, all the methods are on the same line except for 'rkf'. This is more interesting as we know that 'rkf' should also have error control. This is explained by noting that 'rkf' has smaller default absolute and relative tolerances in Scilab. We will show during a tolerance analysis in Section $\ref{subsection:time_tolerance_study}$ that with a sharp enough tolerance, it also provides the right solution.

The other methods are all error-controlled and are on the same line as expected. We note that all of the other methods have a higher default tolerance than 'rkf' and thus this result is not surprising.

This also points to us that an error control solver with a sharp tolerance is able to step over a discontinuity.
  

\subsection{Better way to treat discontinuities in the time models}
\label{subsection:time_disc_handling}
A better way to solve the time dependent discontinuity problem is to make use of cold starts so that we integrate before and after the discontinuity with different (separate) calls to the solver. Cold starting at a time dependent discontinuity improves the accuracy as we will see in this and the next section. It also improves the efficiency as less function calls are required since we do not have the spike in function calls due to repeated step-size resizing described in Section $\ref{subsection:effect_of_discontinuity}$.

A cold start will entail using new unfilled data structures with no values from previous computations corrupting the new integration. It will also use new initial step sizes and for methods of varying order like the BDF and Adams, we start anew with the default order.  Each call thus do not overlap and the solver thus integrates a continuous subinterval with each call and will not have to step over a discontinuity.

To solve the time discontinuity problem, we will integrate from time 0 to the time that measures are implemented, 27, with one call to the solver and use its solution values at time 27 as the initial values to make another call that will integrate from then to the end. The pseudo-code is as follows:

\begin{minipage}{\linewidth}
\begin{lstlisting}[language=Python]
initial_values = (S0, E0, I0, R0)
tspan_before = [0, 27]
solution_before = ode(intial_values, model_before_measures,
 tspan_before)

initial_values_after = extract_last_row(solution_before)
tspan_after = [27, 95]
solution_after = ode(intial_values_after, 
model_after_measures, tspan_after)

solution = concatenate(solution_before, solution_after)
\end{lstlisting}
\end{minipage}

This technique can be applied by epidemiologists for any problem where they know when the discontinuity is introduced or any time they feel that a time dependent if-statement should be added to their model.

 \subsubsection{Solving time discontinuity in R} 
\begin{figure}[h]
	\centering
	\includegraphics[width=0.7\linewidth]{./figures/solve_time_discontinuity_R}
	\caption{Solving Time Discontinuity in R}
	\label{fig:solve_time_discontinuity_R}
\end{figure}
From Figure $\ref{fig:solve_time_discontinuity_R}$, the 'euler' method still fails even with the discontinuity handling. This is as expected as it has no error control and thus it still suffers from the stability issues and will require smaller steps to integrate even continuous problems.

We see that breaking the problem into two makes 'rk4' perform better. The method has a higher order, meaning that it does not need as small an initial step size as 'euler' to solve the two continuous problems but this exceptionally well performance is still unexpected. We will show in Figure $\ref{fig:rk4_messing_up_with_event_R}$ that this is only due to a very small initial step size and the problem with the old method of outputting points as described in Section $\ref{subsection:solution_output_points_impl}$.

\begin{figure}[h]
	\centering
	\includegraphics[width=0.7\linewidth]{./figures/rk4_messing_up_with_event_R}
	\caption{R's rk4 with bigger step-size with discontinuity handling}
	\label{fig:rk4_messing_up_with_event_R}
\end{figure}

Thus our recommendation to avoid fixed step size solvers still holds as for any particular problem, researchers will never know how small the step size should be without knowing the actual solution.

We also note again, that all the error-controlled solvers perform well. We will see, from the efficiency data, that using cold starts is more efficient. Using cold starts, the error control solvers do not have to step over a discontinuity and we will not have the rise in the number of function evaluation as we discussed in $\ref{subsection:effect_of_discontinuity}$. Table $\ref{tab:time_discontinuity_R}$ shows that discontinuity handling reduces the number of function evaluations. 

\begin{table}[h]
\caption {R Time Discontinuity problem efficiency data} \label{tab:time_discontinuity_R}
\begin{center}
\begin{tabular}{ c c c } 
 method & nfev & event's nfev \\ 
euler & 96  & 97  \\
rk4   & 381 & 382 \\ 
lsoda & 332 & 272 \\
ode45 & 735 & 599 \\
radau & 679 & 585 \\
bdf   & 423 & 263 \\
adams & 210 & 176 \\
daspk & 517 & 521
\end{tabular}
\end{center}
\end{table}

Our analysis of the efficiency data in Table $\ref{tab:time_discontinuity_R}$ starts by noting that the non-error controlled solvers in the 'euler' and 'rk4' methods have the same number of function evaluations, the additional one being due to integrating twice at time 27. This indicates that they are just stepping from output point to output point using the same fixed step-size both with and without the discontinuity handling.

Next, we note the drastic decreases in the number of function evaluations from all the remaining solvers except 'daspk'. These reductions in the number of function evaluations will have a significant impact on the CPU time if the problem were harder. This is entirely explained in $\ref{subsection:effect_of_discontinuity}$ where the error controlled solvers have to repeatedly resize the step-size as they encounter a discontinuity. Not having to integrate through a discontinuity means that there is no need to perform the 'crash' into a discontinuity.

VI =======================
Finally, we explain the almost constant value of dapsk's number of function evaluations through the fact that it may have some inherent discontinuity handling and thus even without the treatment we did, it could still detect discontinuity and integrate as usual. IT COULD ALSO BE BECAUSE daspk DEPENDS ON THE SPACING BETWEEN THE POINTS.
======================== VI

In Section $\ref{subsection:time_tolerance_study}$, we will see that this discontinuity handling also allows us to use coarser tolerances.

\subsubsection{Solving time discontinuity in Python} 
\begin{figure}[h]
	\centering
	\includegraphics[width=0.7\linewidth]{./figures/solve_time_discontinuity_py}
	\caption{Solving Time Discontinuity in Python}
	\label{fig:solve_time_discontinuity_py}
\end{figure}
Python did not have a problem even with the if-statement inside. This is because all the available methods use error control in Python and its default tolerances were sharp enough. From Figure $\ref{fig:solve_time_discontinuity_py}$, we can see that Python again gave the correct results. Furthermore, the slight blurring at the peak has disappeared. The addition of discontinuity handling will also drastically reduce the number of function evaluations as seen in Table $\ref{tab:time_discontinuity_Py}$.

\begin{table}[h]
\caption {Python Time Discontinuity problem efficiency data} \label{tab:time_discontinuity_Py} 
\begin{center}
\begin{tabular}{ c c c }
 method & nfev & event nfev \\ 
lsoda  & 162 & 124 \\
rk45   & 134 & 130 \\
bdf    & 202 & 146 \\
radau  & 336 & 220 \\
dop853 & 329 & 181 \\
rk23   & 152 & 127 \\
\end{tabular}
\end{center}
\end{table}

We note that we are not using $dense\_output$ here. However, Python does not seem to allow the space between the points affect the accuracy. It seems to be doing some form of local interpolation in a step. $dense\_output$ is used to do a global interpolation.

From Table $\ref{tab:time_discontinuity_Py}$, we see that across the board, the methods take less function evaluations. There are some huge changes for 'BDF', 'DOP853' and 'Radau'. There are slight decreases in 'LSODA' and 'RK23' and only a very small decrease in 'RK45'. In Section $\ref{subsection:time_tolerance_study}$, we will see that this discontinuity handling also allows us to use coarser tolerances.

\subsubsection{Solving time discontinuity in Scilab} 
\begin{figure}[h]
	\centering
	\includegraphics[width=0.7\linewidth]{./figures/solve_time_discontinuity_scilab}
	\caption{Solving Time Discontinuity in Scilab}
	\label{fig:solve_time_discontinuity_scilab}
\end{figure}
We can see from Figure $\ref{fig:solve_time_discontinuity_scilab}$ that all the methods lie on a single line and thus the time discontinuity has been solved. The 'rkf' method is also on that line and is correctly solving the solver. This is despite the fact that 'rkf' has a coarser default tolerance. In Section $\ref{subsection:time_tolerance_study}$, we will see that this discontinuity handling also allows us to use coarser tolerances and thus explains why the default tolerance 'rkf' is also solving the problem correctly.

The addition of discontinuity handling will also drastically reduce the number of function evaluations as seen in Table $\ref{tab:time_discontinuity_scilab}$.

\begin{table}[h]
\caption {Scilab Time Discontinuity problem efficiency data} 
    \label{tab:time_discontinuity_scilab} 
\begin{center}
\begin{tabular}{ c c c }
 method & nfev &  disc. handling nfev \\ 
    lsoda & 346  & 292 \\
    stiff & 531  & 362 \\
    rkf   & 589  & 590 \\
    rk    & 1649 & 1473 \\
    adams & 304  & 221  \\
\end{tabular}
\end{center}
\end{table}

From Table $\ref{tab:time_discontinuity_sci}$, we see that across the board, the methods take less function evaluations. We see substantial decreases in the number of function evaluations for lsoda, stiff, rk and adams.

The odd values of 'rkf' whereby the number of function evaluations does not decrease is because 'rkf' is using the old method for outputting points as outlined in Section $\ref{subsection:solution_output_points_impl}$. The results when we space out the points more are 208 without discontinuity handling and 288 with discontinuity handling. We see that the number of function evaluations increased.

We note that the high number of of function evaluations in 'rk' with and without discontinuity handling is because it is using Richardson extrapolation to get an error estimate.

VI ==================
rkf spaced out had number of function evaluations increased??
=================== VI

\subsection{Efficiency data and tolerance study for the time discontinuous problem}
\label{subsection:time_tolerance_study}
It is not uncommon for researchers to use the ODE in a loop or within an optimisation algorithm so that they can get models with different parameters. In so doing, some may be tempted to coarsen the tolerances whenever the experiment they are performing is taking too long. In this Section, we investigate how coarse we can set the tolerance while keeping accurate results. 

We investigate 'lsoda' across R, Python and Scilab as they all appear to use the same source code. We use this experiment to show that the discontinuity handling allow us to use coarser tolerances.

We will also investigate 'rkf' in Scilab as it has a smaller default tolerance and will prove that it can solve the problem without discontinuity handling only for sharper tolerances than its default. We also investigate Runge-Kutta pairs of the same order in the other programming environments. R and Python have a version of DOPRI5 but do not share the same source code, the DOPRI5 in Python being a Python implementation and the one in R being a Fortran implementation. We also note that R is not using DOPRI5.f but another Fortran implementation of DOPRI5.

\subsubsection{Comparing LSODA across platforms for time discontinuous problem}

In this Section, we run R's LSODA solver with multiple tolerances with and without discontinuity handling. We will set both the relative and absolute tolerances to particular values and see how coarse we can keep the tolerance while still having accurate results. We also look at efficiency data to see the decreases in the number of function evaluations, which would lead to significant decreases in computation times.

\subparagraph{Time discontinuity LSODA tolerance study in R}
\begin{figure}[h]
	\centering
	\includegraphics[width=0.7\linewidth]{./figures/tolerance_time_lsoda_no_event_R}
	\caption{Time discontinuity tolerance study on R's LSODA without a cold start}
	\label{fig:tolerance_time_lsoda_no_event_R}
\end{figure}

\begin{figure}[h]
	\centering
	\includegraphics[width=0.7\linewidth]{./figures/tolerance_time_lsoda_with_event_R}
	\caption{Time discontinuity tolerance study on R's LSODA with a cold start}
	\label{fig:tolerance_time_lsoda_with_event_R}
\end{figure}

From Figures $\ref{fig:tolerance_time_lsoda_no_event_R}$ and $\ref{fig:tolerance_time_lsoda_with_event_R}$, we can see that the addition of discontinuity handling lets us use a coarser tolerance and still get the required answer; we need $10^{-3}$ and sharper tolerances without discontinuity handling but can use $10^{-2}$ and sharper with it. This solidifies that a form of discontinuity handling when coding a discontinuous problem will improve the accuracy of the solution. Also, using coarser tolerances gives us more efficiency, as we will see in Table $\ref{tab:tolerance_time_discontinuity_lsoda_R}$. This allows researchers to coarsen the tolerances of their experiment when the latter is running too slow.

\begin{table}[h]
\caption {R LSODA Time Discontinuity tolerance study} \label{tab:tolerance_time_discontinuity_lsoda_R} 
\begin{center}
\begin{tabular}{ c c c c c }
tolerance & nfev & nsteps &  disc. handling nfev &  disc. handling nsteps \\ 
1e-01 & 197 &  98 & 200 &  99 \\
1e-02 & 214 & 104 & 206 & 101 \\
1e-03 & 264 & 122 & 212 & 105 \\
1e-04 & 264 & 123 & 224 & 111 \\
1e-05 & 317 & 145 & 244 & 121 \\
1e-06 & 332 & 154 & 272 & 133 \\
1e-07 & 393 & 185 & 298 & 146 \\
\end{tabular}
\end{center}
\end{table}

From Table $\ref{tab:tolerance_time_discontinuity_lsoda_R}$, we see that for the coarser tolerances, the number of function evaluations is roughly the same. But with sharper tolerances, a lot more function evaluations are required and thus if we had a user-provided function which took longer to run, we will see clear drops in computation times.

The similar number of function evaluations for the coarser tolerances should not distract from the fact that the code without discontinuity at these tolerances are not as accurate as the code with. The small differences of 3 function evaluations for the 0.1 tolerance case and 8 function evaluations in the 0.01 case do not excuse that the solutions are wrong.

\subparagraph{Time discontinuity LSODA tolerance study in Python}
In this Section, we run Python's LSODA solver with multiple tolerances with and without discontinuity handling. We note that Python was working correctly in both cases apart from some blurring but we will see how coarse we can keep the tolerance while still having correct results. We set both the relative and absolute tolerances to particular values. We also look at efficiency data to see the decreases in the number of function evaluations.

\begin{figure}[h]
	\centering
	\includegraphics[width=0.7\linewidth]{./figures/tolerance_time_lsoda_no_event_py}
	\caption{Time discontinuity tolerance study on Python's LSODA without a cold start}
	\label{fig:tolerance_time_lsoda_no_event_py}
\end{figure}

\begin{figure}[h]
	\centering
	\includegraphics[width=0.7\linewidth]{./figures/tolerance_time_lsoda_with_event_py}
	\caption{Time discontinuity tolerance study on Python's LSODA with a cold start}
	\label{fig:tolerance_time_lsoda_with_event_py}
\end{figure}

From Figures $\ref{fig:tolerance_time_lsoda_with_event_py}$ and $\ref{fig:tolerance_time_lsoda_no_event_py}$, the addition of the discontinuity handling lets us use a coarser tolerance, as $10^{-2}$ was enough to get the correct answer with the discontinuity handling whereas $10^{-3}$ was needed without. This essentially tells us that using discontinuity handling will improve our results for a more complex time-dependent discontinuity problem.

In turn, the use of coarser tolerances give us more efficiency. (See Table $\ref{tab:tolerance_time_discontinuity_lsoda_py}$.)

We also note that the results using LSODA in Python and R are very similar which stems from the fact that they are using the same source code.

\begin{table}[h]
\caption {Python LSODA Time Discontinuity tolerance study} \label{tab:tolerance_time_discontinuity_lsoda_py} 
\begin{center}
\begin{tabular}{ c c c }
tolerance & nfev  & disc. handling nfev \\ 
0.1    & 79.0  & 86.0  \\
0.01   & 98.0  & 93.0  \\
0.001  & 156.0 & 116.0 \\
0.0001 & 185.0 & 146.0 \\
1e-05  & 259.0 & 186.0 \\
1e-06  & 283.0 & 228.0 \\
1e-07  & 361.0 & 272.0 \\
\end{tabular}
\end{center}
\end{table}
Again, in Table $\ref{tab:tolerance_time_discontinuity_lsoda_py}$, we see that that at coarse tolerances, the number of function evaluations is roughly the same. This similar number of function evaluations does not excuse the fact that the coarser tolerances are giving erroneous values. 

At sharp tolerance, where the comparison is fair, the number of function evaluations is much smaller with the discontinuity handling than without; we make 40 less function evaluations at 0.001 and 0.0001 but we do much less function evaluations for sharper tolerances. We note that if the user-provided function was more time-consuming, this reduced number of function evaluations will cause a decrease in the CPU times. This reduced number of function evaluations stems from the facts discussed in Section $\ref{subsection:effect_of_discontinuity}$. 

\subparagraph{Time discontinuity LSODA tolerance study in Scilab}
In this Section, we run Scilab's LSODA solver with multiple tolerances with and without discontinuity handling. We will set both the relative and absolute tolerances to particular values and see how coarse we can keep the tolerance while still getting correct results.

\begin{figure}[h]
	\centering
	\includegraphics[width=0.7\linewidth]{./figures/tolerance_time_lsoda_no_event_sci}
	\caption{Time discontinuity tolerance study on Scilab's lsoda without a cold start}
	\label{fig:tolerance_time_lsoda_no_event_sci}
\end{figure}

\begin{figure}[h]
	\centering
	\includegraphics[width=0.7\linewidth]{./figures/tolerance_time_lsoda_with_event_sci}
	\caption{Time discontinuity tolerance study on Scilab's lsoda with a cold start}
	\label{fig:tolerance_time_lsoda_with_event_sci}
\end{figure}

From Figures $\ref{fig:tolerance_time_lsoda_no_event_sci}$ and $\ref{fig:tolerance_time_lsoda_with_event_sci}$ we can see that at $10^{-1}$ to $10^{-4}$, Scilab's LSODA without discontinuity handling fails but we seem to be able to use $10^{-3}$ with the discontinuity handling. 

It is interesting to see how far off the solution without discontinuity handling is at a tolerance of $10^{-1}$. We also note that this behaviour is different from R's and Python's LSODA but this may be due to the way Scilab processes the tolerances before handing it to the source code.

\begin{table}[h]
\caption {Scilab LSODA Time Discontinuity tolerance study} 
    \label{tab:tolerance_time_discontinuity_lsoda_scilab} 
\begin{center}
\begin{tabular}{ c c c }
tolerance & nfev &  disc. handling nfev \\ 
    0.1   & 80.  & 82.   \\
    0.01  & 98.  & 92.   \\
    0.001 & 156. & 116.  \\
    1e-4  & 185. & 146.  \\
    1e-5  & 255. & 186.  \\
    1e-6  & 280. & 228.  \\
    1e-7  & 361. & 272.  \\
\end{tabular}
\end{center}
\end{table}
Again, in Table $\ref{tab:tolerance_time_discontinuity_lsoda_scilab}$, we see that the number of function evaluations is roughly the same at coarser tolerances but that at sharp tolerances, where both give accurate solution and thus allow far comparison, the code with the discontinuity handling performs better than the code without. We can use up to 90 less function evaluations through discontinuity handling.  

\subsubsection{Comparing Runge-Kutta pairs across platforms for the time discontinuous problem}
\subparagraph{Time discontinuity tolerance study on R's version of DOPRI5}

In this Section, we use R's version of DOPRI5, which is the 'ode45' method of the $ode()$ function, with multiple tolerances with and without discontinuity handling. We will set both the relative and absolute tolerances to particular values and see how coarse we can keep the tolerance while still getting correct results. We also look at efficiency data to see the decreases in the number of function evaluations.

\begin{figure}[h]
	\centering
	\includegraphics[width=0.7\linewidth]{./figures/tolerance_time_rk45_no_event_R}
	\caption{Time Discontinuity tolerance study on R's version of dopri5 without discontinuity handling}
	\label{fig:tolerance_time_rk45_no_event_R}
\end{figure}

\begin{figure}[h]
	\centering
	\includegraphics[width=0.7\linewidth]{./figures/tolerance_time_rk45_with_event_R}
	\caption{Time Discontinuity tolerance study on R's version of dopri5 with discontinuity handling}
	\label{fig:tolerance_time_rk45_with_event_R}
\end{figure}

From Figures $\ref{fig:tolerance_time_rk45_no_event_R}$ and $\ref{fig:tolerance_time_rk45_with_event_R}$, the addition of discontinuity handling lets us use a smaller tolerance and still get the required answer. Without discontinuity handling, we had to use $10^{-4}$ for both the absolute and relative tolerance but without, we seem to be able to use $10^{-1}$. 

However as we will see in the Python's version of DOPRI5, the results from $\ref{fig:tolerance_time_rk45_no_event_R}$ and $\ref{fig:tolerance_time_rk45_with_event_R}$ are suspicious and stem from the fact that R is not using interpolation to produce the results. It is using an old method that depends on the selected output points which affects efficiency and accuracy.


\begin{table}[h]
\caption {R dopri5 Time Discontinuity tolerance study} \label{tab:tolerance_time_discontinuity_rk45_R} 
\begin{center}
\begin{tabular}{ c c c c c }
tolerance & nfev & nsteps & disc. handling nfev & disc. handling nsteps \\ 
1e-01 & 572 &  95 & 574 &  95 \\
1e-02 & 572 &  95 & 574 &  95 \\
1e-03 & 572 &  95 & 574 &  95 \\
1e-04 & 612 & 101 & 574 &  95 \\
1e-05 & 692 & 113 & 587 &  97 \\
1e-06 & 735 & 120 & 599 &  99 \\
1e-07 & 926 & 150 & 702 & 116 \\
\end{tabular}
\end{center}
\end{table}

Table $\ref{tab:tolerance_time_discontinuity_rk45_R}$ also confirms our suspicions as at coarser tolerances, 1e-01 to 1e-3, the number of function evaluations does not change at all. This indicates that something else, not the tolerance nor the discontinuity, is the limiting factor for the number of function evaluations and that this other factor requires 572 or 574 function evaluations.

We suspect that R's DOPRI5 version is not using interpolation or some other dense output technique to produce its solutions and that it is integrating using the sampling points. That is, even though DOPRI5 is an algorithm that does not have a fixed step-size, R is forcing it to step from one output point to the other output point and thus our set of sampling points is a limiting factor. We then do the following experiment where we give R a smaller set of output points with the points are further away from each other and see what happens.

\begin{figure}[h]
	\centering
	\includegraphics[width=0.7\linewidth]{./figures/tolerance_time_rk45_further_no_event_R}
	\caption{Time Discontinuity tolerance study on R's version of dopri5 without discontinuity handling and output points more space out}
	\label{fig:tolerance_time_rk45_further_no_event_R}
\end{figure}

\begin{figure}[h]
	\centering
	\includegraphics[width=0.7\linewidth]{./figures/tolerance_time_rk45_further_with_event_R}
	\caption{Time Discontinuity tolerance study on R's version of dopri5 with discontinuity handling and output points more space out}
	\label{fig:tolerance_time_rk45_further_with_event_R}
\end{figure}

From Figures $\ref{fig:tolerance_time_rk45_further_no_event_R}$ and $\ref{fig:tolerance_time_rk45_further_with_event_R}$, we can now see a more drastic change in the solution from the codes the output points further spaced out. Also, see table $\ref{tab:tolerance_time_discontinuity_rk45_further_R}$ where we will see the number of function evaluations actually change.

Using these two figures, we also see that discontinuity handling is allowing us to use coarser tolerances. We are able to use even $10^{-1}$ with discontinuity handling while getting an accurate result whereas without it, we need to use $10^{-3}$ and sharper tolerances to get the required answer.

\begin{table}[h]
\caption {R DOPRI5 Time Discontinuity tolerance study with spaced out points} \label{tab:tolerance_time_discontinuity_rk45_further_R} 
\begin{center}
\begin{tabular}{ c c c c c }
tolerance & nfev & nsteps & disc. handling nfev & disc. handling nsteps \\ 
1e-01 & 116 &  19 & 112 & 18 \\
1e-02 & 142 &  23 & 125 & 20 \\
1e-03 & 168 &  27 & 131 & 21 \\
1e-04 & 246 &  39 & 162 & 26 \\
1e-05 & 352 &  56 & 235 & 38 \\
1e-06 & 614 &  97 & 349 & 57 \\
1e-07 & 796 & 128 & 542 & 89 \\
\end{tabular}
\end{center}
\end{table}

Our analysis of Table $\ref{tab:tolerance_time_discontinuity_rk45_further_R}$ begins by noting that the set of output points is not longer a limiting factor. We can see the number of function evaluation change with the tolerance now and this indicates that the tolerance is affecting the step-size.This confirms our suspicions that R's implementation of DOPRI5 is not using a dense output mode or some form of interpolation. Instead it makes a steps to and stops at every required output points. This is the old way of giving function values at specified output points. 

Regarding the accuracy of the solver as we coarsen the tolerance we can see from Figures $\ref{fig:tolerance_time_rk45_further_no_event_R}$ and $\ref{fig:tolerance_time_rk45_further_with_event_R}$ that even at $10^{-1}$, the code with the discontinuity handling is still able to produce accurate solutions whereas it requires $10^{03}$ for the one without the discontinuity handling.

The new table, Table $\ref{tab:tolerance_time_discontinuity_rk45_further_R}$, does offer some more insights. Again we can see that at coarser tolerances, the decrease in the number of function evaluations is small but as the tolerance is sharpened, the number of function evaluations decreases significantly. The relatively similar number of function evaluations at the coarser tolerances does not excuse that the code without discontinuity handling is not getting the correct answer. 

\subparagraph{Time discontinuity tolerance study on Python's version of DOPRI5}
In this section, we run Python's version of DOPRI5, which is aliased under 'RK45' from the $solver\_ivp()$ function, with multiple tolerances with and without discontinuity handling. We will set both the relative and absolute tolerances to particular values and see how coarse we can keep the tolerance while still having correct results. We also look at efficiency data to see the decreases in the number of function evaluations.

\begin{figure}[h]
	\centering
	\includegraphics[width=0.7\linewidth]{./figures/tolerance_time_rk45_no_event_py}
	\caption{Time Discontinuity tolerance study on Python's version of dopri5 without discontinuity handling}
	\label{fig:tolerance_time_rk45_no_event_py}
\end{figure}

\begin{figure}[h]
	\centering
	\includegraphics[width=0.7\linewidth]{./figures/tolerance_time_rk45_with_event_py}
	\caption{Time Discontinuity tolerance study on Python's version of dopri5 with discontinuity handling}
	\label{fig:tolerance_time_rk45_with_event_py}
\end{figure}

From Figures $\ref{fig:tolerance_time_rk45_with_event_py}$ and $\ref{fig:tolerance_time_rk45_no_event_py}$, we can see clearer differences at the different tolerance values. This is in contrast with the first tolerance study on R's DOPRI5. From studying Python's $solve\_ivp$ interface and source code, we note that Python is definitely using dense output/interpolation. This  concludes the discussion on R's DOPRI5 and we can explain R's DOPRI4 performance entirely because it does not use interpolation by default but instead stops at every output point.

We then compare Python's DOPRI5 with and without discontinuity handling. We can see that the use of discontinuity handling allowed us to use coarser tolerances in Python while keeping accurate results. We see that we need $10^{-5}$ and sharper to get the correct solutions without discontinuity handling in Python while $10^{-2}$ was enough with discontinuity handling. We will also see in Table $\ref{tab:tolerance_time_discontinuity_rk45_py}$ that the discontinuity handling code is much more efficient as well.


\begin{table}[h]
\caption {Python DOPRI5 Time Discontinuity tolerance study} \label{tab:tolerance_time_discontinuity_rk45_py} 
\begin{center}
\begin{tabular}{ c c c }
tolerance & nfev & disc. handling nfev \\ 
0.1   & 68.0  & 70.0  \\
0.01  & 86.0  & 88.0  \\
0.001 & 146.0 & 124.0 \\
0.0001& 224.0 & 172.0 \\
1e-05 & 326.0 & 250.0 \\
1e-06 & 488.0 & 370.0 \\
1e-07 & 752.0 & 568.0 \\
\end{tabular}
\end{center}
\end{table}

From Table $\ref{tab:tolerance_time_discontinuity_rk45_py}$, we see that at coarser tolerances, the number of function evaluations is lower in Python without the discontinuity handling than wit. But we should also point out that in Python, DOPRI5 at coarse tolerances gives very erroneous results and these do not excuse the small gain in efficiency.

At sharper tolerance where we get accurate results both with and without discontinuity handling and thus a fair comparison can be done, we can see that the code with discontinuity handling performs much better. At $10^{-7}$, the drop in the number of function evaluations is very significant and would lead to much faster execution times whereas at $10^{-5}$ and sharper, the decrease in the number of function evaluations is 75 or more.

\subparagraph{Time discontinuity tolerance study on Scilab's version of RKF45}
In this Section we run Scilab's RKF45 aliased as 'rkf' in the $ode()$ function with different tolerances. We note that the default tolerance for Scilab's 'rkf' was not enough to solve the problem without discontinuity handling but using cold starts did solve the problem even with that default tolerance. 

By running 'rkf' at various tolerances, we will show that it can also come up with the correct solutions at sharper tolerances without a discontinuity. Thus the anomaly we saw in section $\ref{subsection:naive_time_problem}$ was entirely because the solver has a coarser default tolerance in Scilab than the other methods.

We will also see that using the discontinuity handling lets us use less function evaluations which, given a more complex problem, will be a significant improvement in computation times.

\begin{figure}[h]
	\centering
	\includegraphics[width=0.7\linewidth]{./figures/tolerance_time_rk45_no_event_sci}
	\caption{Time Discontinuity tolerance study on Scilab's RKF45 without discontinuity handling}
	\label{fig:tolerance_time_rk45_no_event_sci}
\end{figure}

\begin{figure}[h]
	\centering
	\includegraphics[width=0.7\linewidth]{./figures/tolerance_time_rk45_with_event_sci}
	\caption{Time Discontinuity tolerance study on Scilab's RKF45 with discontinuity handling}
	\label{fig:tolerance_time_rk45_with_event_sci}
\end{figure}

We see from Figure $\ref{fig:tolerance_time_rk45_no_event_sci}$ that $10^{-4}$ for both the absolute and the relative tolerance give accurate answers and that anything coarser does not work. We then remember that the relative tolerance defaults to $10^{-3}$ and the absolute tolerance defaults to $10^{-4}$ for 'rkf' which is slightly coarser than what was needed to get this correct solution.

Figure $\ref{fig:tolerance_time_rk45_with_event_sci}$ is also interesting as it seems to indicate that a $10^{-1}$ is enough to get the correct solution with discontinuity handling. This is also surprising but is consistent with our observations in R and Python's Runge-Kutta pairs.

\begin{table}[h]
\caption {Scilab RKF45 Time Discontinuity tolerance study} 
\label{tab:tolerance_time_discontinuity_rk45_scilab} 
\begin{center}
\begin{tabular}{ c c c }
tolerance & nfev  & disc. handling nfev \\ 
    0.1  & 577.  & 584. \\
    0.01  & 577. & 584. \\
    0.001  & 583. &  584. \\
    1e-4  & 641.  &  590. \\
    1e-5  & 674.  &  608. \\
    1e-6 &  847.  &  764. \\
    1e-7  & 924.  & 830.   \\
\end{tabular}
\end{center}
\end{table}
We can see from table $\ref{tab:tolerance_time_discontinuity_rk45_scilab}$ that Scilab's 'rkf' is not using interpolation. We can say this because even at extremely low tolerances, it is still using the same number of function evaluation. There are also no difference with and without discontinuity handling. We also note that the tolerance did not change the number of function evaluations and thus something else is regulating the number of function evaluation. Doing the same experiment with the points further spaced out shows us that it was the spacing that was a problem.

We start by replicating the experiments in the previous sections
\begin{figure}[h]
	\centering
	\includegraphics[width=0.7\linewidth]{./figures/tolerance_time_rkf_further_no_event_sci}
	\caption{Time Discontinuity tolerance study on Scilab's RKF45 without discontinuity handling}
	\label{fig:tolerance_time_rkf_further_no_event_sci}
\end{figure}

\begin{figure}[h]
	\centering
	\includegraphics[width=0.7\linewidth]{./figures/tolerance_time_rkf_further_with_event_sci}
	\caption{Time Discontinuity tolerance study on Scilab's RKF45 with discontinuity handling}
	\label{fig:tolerance_time_rkf_further_with_event_sci}
\end{figure}

Figures $\ref{fig:tolerance_time_rkf_further_no_event_sci}$ and $\ref{fig:tolerance_time_rkf_further_with_event_sci}$ shows a clearer distinction in why discontinuity handling is important. We can see that without, we need a tolerance of tolerance of $10^{-3}$ to get accurate results but with the discontinuity handling, we can use $10^{-1}$. We note that such coarse tolerance may mean that we still did not space the points enough but the number of function evaluation study, shown in Table $\ref{tab:tolerance_time_discontinuity_rk45_spaced_out_scilab}$, is clear.

\begin{table}[h]
\caption {Scilab RKF45 Spaced Out Time Discontinuity tolerance study} 
\label{tab:tolerance_time_discontinuity_rk45_spaced_out_scilab} 
\begin{center}
\begin{tabular}{ c c c }
tolerance & nfev  & disc. handling nfev \\ 
   0.1   & 133. & 134. \\
   0.01  & 166. & 152. \\
   0.001 & 208. & 176. \\
   1e-4  & 322. & 254. \\
   1e-5  & 417. & 338. \\
   1e-6  & 606. & 482. \\
   1e-7  & 864. & 704. \\
\end{tabular}
\end{center}
\end{table}

Table $\ref{tab:tolerance_time_discontinuity_rk45_spaced_out_scilab}$ shows how the number of function evaluation with discontinuity handling is lesser. We also note that at coarse tolerance the number of function evaluation are similar but that at those tolerances, the code without discontinuity handling was not obtaining the correct results. We can thus conclude that using discontinuity handling lets us use lower tolerances and less number of function evaluations while improving the accuracy.


\section{State dependent discontinuity problem}
In this section, we consider the state-dependent discontinuity problem. We start by noting that this problem cannot be solved with the form of discontinuity handling used in the previous problem as we do not know when the discontinuity arises. Also, this problem will be harder than the time-dependent discontinuity problem as the parameter $\beta$ will be changed more than once as we attempt to model the waves of imposition of Covid-19 measures followed by periods where these measures are removed. 

As in Section $\ref{section:time_problem}$, changes in the modelling parameter $\beta$ introduce discontinuities in the function $f(t, y(t))$ and thus some solvers will ``thrash" when trying to solve the problem (as described in Section $\ref{subsection:effect_of_discontinuity}$). We will show that the presence of several discontinuities makes the problem hard enough that all the ODE solvers we considered, even at very sharp tolerances, will not be able to solve the problem with reasonable accuracy.

The problem uses the state variable, E, which is the number of Exposed people, to determine when to change the parameter $\beta$. When the number of exposed people is greater than 25000, measures will be introduced and thus $\beta$ will change from 0.9 to 0.005. When the number of exposed people drops to 10000, the measures will be relaxed and $\beta$ is set to 0.9. We run this model over a longer time period toggling the parameter $\beta$ back and forth to model the waves of alternating the imposition and relaxing of the measures. This scenario corresponds to the case of an unvaccinated population where the only means of controlling the spread of the virus is through measures such as social isolation, masking, etc... The ability of the virus to infect people is not diminished as time progresses, and when measures to stop the spread of the virus are removed, the infection rate of the virus returns to its original value.

We start with a naive treatment of the problem with if-statements applied inside the function that defines the right-hand side of the ODE system. We proceed to show how the problem cannot be solved this way even at sharp tolerances and finally, we will introduce a way to efficiently and accurately solve the problem using event detection.

\subsection{Naive treatment of Covid-19 state dependent discontinuity model}
\label{subsection:naive_state_problem}
The naive treatment of this problem is to use global variables for tracking when measures are implemented and relaxed and to toggle these global variables as we reach the required thresholds. Global variables are needed because we need to know if the number of Exposed people is going up or down to know whether we need to check for the maximum or the minimum threshold. We then have an if-statement that will choose the value of parameter $\beta$ based on whether measures are being implemented. The pseudo-code for this algorithm is as follows:

\begin{minipage}{\linewidth}
\begin{lstlisting}[language=Python]
measures_implemented = False
direction = "up"

function model_with_if(_, y):
    // ...
    global measures_implemented, direction
    if (direction == "up"):
        if (E > 25000):
            measures_implemented = True
            direction = "down"
    else:
        if (E < 10000):
            measures_implemented = False
            direction = "up"

    if measures_implemented:
        beta = 0.005 
    else:
        beta = 0.9
    // ...
    return (dSdt, dEdt, dIdt, dRdt)
\end{lstlisting}
\end{minipage}

\subsubsection{Solution to the state dependent discontinuity model in R}
\begin{figure}[H]
\centering
\includegraphics[width=0.7\linewidth]{./figures/state_discontinuity_R}
\caption{State dependent discontinuity model in R}
\label{fig:state_discontinuity_R}
\end{figure}
Figure $\ref{fig:state_discontinuity_R}$ shows how difficult this problem is with a naive treatment. We note that none of the solutions are aligned and that none of the solvers get the accurate solution (described in Section $\ref{subsection:state_with_event_detection}$) as none of the computed solutions cleanly oscillate between 10000 and 25000 with clear peaks and troughs.

We note that all the solvers, even the error-controlled ones, did not issue a warning about the integration and thus users may be tempted to think that their code has solved the problem to within reasonable accuracy. Having no warning also tells us that the error estimation and error control algorithms employed by all the solvers did not detect anything abnormal; the solvers return with an indication that the provided solutions are accurate to within the requested tolerance.

As we are modeling E, we expect that each graph should go from 25000 to 10000 and back to 25000 repeatedly but none of these graphs do so in the required pattern. We would also expect the solvers with error control to repeatedly reduce the step-size to satisfy the tolerance and compute solutions that align with each other but Figure $\ref{fig:state_discontinuity_R}$ shows that this is not the case.

We also note that the result for `euler' is especially poor as it reaches a maximum of 40000. This is again as expected as `euler' has no error control; `rk4', the other fixed step-size method, is also performing poorly as we see the solution it computes reach approximately 30000 in its third peak. This is happening even though the space between the output points is as small as it was when we were investigating the time-dependent discontinuity problem. Because of this, we will not run any spacing of output points experiments in this section. The step-size for these fixed-step solvers is not small enough and further step-size reductions are needed.
`
Another important fact to note is how poorly `Radau', as shown in Figure $\ref{fig:state_discontinuity_radau_R}$, is performing. This is not a problem in the R programming environment as similar results will be seen in Python in the next section and in the Fortran code in Section $\ref{section:fortran_inaccuracies}$. The solution grows exponentially even after the parameter $\beta$ should be switched to 0.005 which should begin a decay.

We perform an analysis with the Fortran code in $\ref{section:fortran_inaccuracies}$ to show that $\beta$ is indeed 0.005 while this exponential growth is happening. 

\begin{figure}[h]
\centering
\includegraphics[width=0.7\linewidth]{./figures/state_discontinuity_radau_R}
\caption{State dependent discontinuity model of Radau in R}
\label{fig:state_discontinuity_radau_R}
\end{figure}

We then proceed to show that sharp tolerances are not enough to solve this problem as was the case for the time-dependent discontinuity problem. We repeated the experiment at the sharpest tolerance usable before some of the solvers failed. This was at $10^{-13}$ in the R environment. We set both the absolute and relative tolerance to that value and show the results in Figure $\ref{fig:state_discontinuity_sharp_R}$.

\begin{figure}[H]
\centering
\includegraphics[width=0.7\linewidth]{./figures/state_discontinuity_sharp_R}
\caption{State dependent discontinuity model in R at high tolerances}
\label{fig:state_discontinuity_sharp_R}
\end{figure}

We can see from Figure $\ref{fig:state_discontinuity_sharp_R}$ that the situation has only marginally improved. None of the solvers give solutions that are in agreement and none of them cleanly oscillate between 10000 and 25000. We note that the error-controlled solvers are following the correct pattern and that until about time 20-30, some of them give solutions that are in agreement, showing that sharp tolerance error-control can step over one state-dependent discontinuity. (See the comparison against the final solution in Section $\ref{subsubsection:state_solution_comparison}$ to see that even this sharp tolerance solution is not accurate enough.)

The fixed step-size method `euler' and `rk4' are the same as in Figure $\ref{fig:state_discontinuity_R}$ since the codes do not employ a tolerance.

We can also point out that at such sharp tolerances, `Radau' longer computes solutions exhibiting the abnormal behavior we saw previously. From Figure $\ref{fig:state_discontinuity_radau_sharp_R}$, we can see that it oscillates approximately between 10000 and 25000. From supplementary experiments, we observe that `Radau' starts performing at a level that is comparable to the other solvers at a tolerance of $10^{-9}$.

\begin{figure}[H]
\centering
\includegraphics[width=0.7\linewidth]{./figures/state_discontinuity_sharp_radau_R}
\caption{State dependent discontinuity model of Radau in R at high tolerances}
\label{fig:state_discontinuity_radau_sharp_R}
\end{figure}

\subsubsection{Solution to the state dependent discontinuity model in Python}
\begin{figure}[H]
\centering
\includegraphics[width=0.7\linewidth]{./figures/state_discontinuity_py}
\caption{State dependent discontinuity model in Python}
\label{fig:state_discontinuity_python}
\end{figure}
Figure $\ref{fig:state_discontinuity_python}$ shows what happens when the problem is coded with global variables and if-statements in Python. We can see that the results are similar to those in R. This happens even though all solvers in Python have error control.

We note that all the solvers except `RK23' give solutions that at least oscillate between 10000 and 25000, though in completely dissimilar patterns. The solutions have peaks and troughs at different times and no warnings were given by the solvers.

The `RK23' solver, in purple, computes a solution with a completely different pattern than the other solvers. It never reaches 25000 and only oscillates between around 10000 and 15000. 

Again, as shown in Figure $\ref{fig:state_discontinuity_radau_py}$, `Radau' computes a solution that has E grow exponentially even though the parameter $\beta$ is eventually set to 0.005 which leads to a solution with an exponential decay in the E component with all other solvers.

\begin{figure}[h]
\centering
\includegraphics[width=0.7\linewidth]{./figures/state_discontinuity_radau_py}
\caption{State dependent discontinuity model of Radau in Python}
\label{fig:state_discontinuity_radau_py}
\end{figure}

We then used very sharp tolerances to solve the problem but, as is the case in the R environment, none of the solvers obtained a reasonably accurate solution. The highest tolerance we could use in Python without any one method failing was $10^{-12}$. Both the absolute and relative tolerances were set to this value and Figure $\ref{fig:state_discontinuity_sharp_python}$ shows the results from this sharp tolerance experiment.

\begin{figure}[H]
\centering
\includegraphics[width=0.7\linewidth]{./figures/state_discontinuity_sharp_py}
\caption{State dependent discontinuity model in Python at sharp tolerances}
\label{fig:state_discontinuity_sharp_python}
\end{figure}

Figure $\ref{fig:state_discontinuity_sharp_python}$ shows that the results did improve. However, the solvers give solutions that are not in agreement. We note that none of the solvers are oscillating beyond 25000 as was the case with the fixed-step solvers in R. At sharp tolerances, the solutions are aligned for the first few discontinuities with only some blurring until about t=25 when the solvers give substantially different solutions. Though the pattern is correct, none of the solvers give solutions that are in agreement telling us that none got the accurate solution that we present in Section $\ref{subsection:state_with_event_detection}$. (See the comparison against the final solution in Section $\ref{subsubsection:state_solution_comparison}$ to see that even this sharp tolerance solution is not accurate enough.)

We note that `RK23' is now following the correct pattern in that it oscillates between 10000 and 25000 whereas it only reached 15000 at the default tolerances. 

\begin{figure}[H]
\centering
\includegraphics[width=0.7\linewidth]{./figures/state_discontinuity_sharp_radau_py}
\caption{State dependent discontinuity model of Radau in Python at sharp tolerances}
\label{fig:state_discontinuity_sharp_radau_py}
\end{figure}

Again, as shown in Figure $\ref{fig:state_discontinuity_sharp_radau_py}$, `Radau' begins to give reasonable solutions at these sharp tolerances; those solutions follows the pattern we are expecting but as we will show in Section $\ref{subsection:state_with_event_detection}$, they are still not sufficiently accurate solutions. `Radau' starts reasonably performing well at around a tolerance of $10^{-10}$. We also note that the R and Python implementation of `Radau' are different. The `Radau' solver in Python is implemented in Python with the NumPy library whereas R uses the Fortran code. Thus we eliminate the possibility of a bug in the code as well as any problem stemming from the interface from R to Fortran or from Python to NumPy. The problem is simply in how the Radau algorithm interacts with this naive implementation of the state-dependent discontinuity. In our experiments with the Radau Fortran code, in Section $\ref{section:fortran_inaccuracies}$, the same behavior is observed.

\subsubsection{Solution to the state dependent discontinuity model in Scilab}
\begin{figure}[H]
\centering
\includegraphics[width=0.7\linewidth]{./figures/state_discontinuity_scilab}
\caption{State dependent discontinuity model in Scilab}
\label{fig:state_discontinuity_scilab}
\end{figure}

Figure $\ref{fig:state_discontinuity_scilab}$ shows the same issues that we saw before. None of the solvers give solutions that are aligned which prompts us to conclude that none of them are getting an accurate solution. All of the solvers in Scilab have error control and we can also see that their solutions all follow the correct pattern of oscillating between 10000 and 25000. However, as we will discuss in Section $\ref{subsection:state_with_event_detection}$, none of the solutions are very accurate. We note that the spacing between output points is not important in this analysis as at the current spacing, even the solvers that depend on the spacing are getting inaccurate answers.

We then repeat the experiment at sharp tolerances. The Scilab rkf' method does not allow the use of very sharp tolerance as it has a cap of 3000 derivatives so it was omitted from this experiment. The sharpest tolerance we can use in Scilab before the other methods fail is $10^{-13}$; the results are shown in Figure $\ref{fig:state_discontinuity_sharp_scilab}$.

\begin{figure}[H]
\centering
\includegraphics[width=0.7\linewidth]{./figures/state_discontinuity_sharp_sci}
\caption{State dependent discontinuity model in Scilab with sharp tolerances}
\label{fig:state_discontinuity_sharp_scilab}
\end{figure}

Again, in Figure $\ref{fig:state_discontinuity_sharp_scilab}$ we can see that the use of sharp tolerances is not enough to force the solvers to compute accurate solutions. The solutions did improve as all the solvers follow the correct pattern but none oscillate between 10000 and 25000 with clear peaks and troughs at those values. For the time period between 0 to 30, the solutions all seem to show reasonable agreement but as we go further in time, all of the solutions diverge. We also note that none of the solvers compute solutions in reasonable agreement with the solution discussed in Section $\ref{subsection:state_with_event_detection}$. (See the comparison against the final solution in Section $\ref{subsubsection:state_solution_comparison}$ to see that even these sharp tolerance solutions are not accurate enough.)


\subsubsection{Solution to the state dependent discontinuity model in Matlab}
\begin{figure}[H]
\centering
\includegraphics[width=0.7\linewidth]{./figures/state_discontinuity_matlab}
\caption{State dependent discontinuity model in Matlab}
\label{fig:state_discontinuity_matlab}
\end{figure}
We see the same incorrect solutions in Matlab at the default tolerances in Figure $\ref{fig:state_discontinuity_matlab}$. The solvers do not even consistently reach 25000. We then use a sharper tolerance to see how the solvers act.

\begin{figure}[h]
\centering
\includegraphics[width=0.7\linewidth]{./figures/state_discontinuity_sharp_matlab}
\caption{State dependent discontinuity model in Matlab with sharp tolerances}
\label{fig:state_discontinuity_sharp_matlab}
\end{figure}

Figure $\ref{fig:state_discontinuity_sharp_matlab}$ shows the results of the experiment at sharp tolerances. We get surprisingly good solutions compared to the solutions in the previous environments. However, as we will see in Section $\ref{subsection:state_with_event_detection}$, these solutions are computed extremely inefficiently and they are not as accurate as the solution presented in Section $\ref{subsection:state_with_event_detection}$, especially for later time periods. (See the comparison against the final solution in Section $\ref{subsubsection:state_solution_comparison}$ to see that even these sharp tolerance solutions are not accurate enough.)


\subsubsection{State dependent discontinuity model - solution comparison}
\label{subsubsection:state_solution_comparison}
In all the previous subsections, we have maintained that even the sharp tolerance solutions, though more in agreement, are not accurate. Here, we present a comparison between LSODA in Python at default tolerance, at the sharpest tolerance, and the final solution we will present shortly. We can see from Figure $\ref{fig:comparison_state_default_sharp_event}$ that the solution both at default and the sharp tolerance does not agree with the accurate solution. 

\begin{figure}[H]
\centering
\includegraphics[width=0.7\linewidth]{./figures/comparison_state_default_sharp_event}
\caption{State dependent discontinuity model solutions comparison}
\label{fig:comparison_state_default_sharp_event}
\end{figure}

We also note that at the default tolerance, the solver uses 2357 function evaluations; at the sharpest tolerance, the solver uses 4282 evaluations; while for the solution obtained using event detection, the solver uses 535 function evaluations.

\subsection{Why the solvers fail even with sharp tolerances}
\label{subsection:state_sharp_tol_failed}
In this section we discuss why sharp tolerances were not enough to force the solvers to solve the problem in the naive way it is coded, i.e, using global variables and if-statements. 

Whenever there is a change in the value of $\beta$, the step that first encounters the ensuing discontinuity will almost always fail. As discussed in Section $\ref{subsection:effect_of_discontinuity}$, the step-size at a discontinuity will always have to be much smaller than the step-size on a continuous region. Thus the first encounter of a solver with any discontinuity will always be in the context of a failed step.

During this failed step, the value of the E will cross the threshold. The global variables will thus be toggled. But then, when the solver attempts to retake the step using a smaller step-size, to the left of the discontinuity, it will be using the wrong $\beta$ value. 

This observation is crucial as it allows us to conclude that just before the discontinuity, the function evaluations should be based on the previous $\beta$ value but they are in fact using the new $\beta$ value. There is no trivial way to code this behavior in the ODE function, $f(t, y(t))$, if we do not know the time of the discontinuity. 

The problem, in summary, is that the solvers need to figure out how to step up to the discontinuity such that to the left of the discontinuity, the step employs function evaluations that use the previous $\beta$, and then after the discontinuity, the solver employs function evaluations that use the new $\beta$ value. This cannot be coded in a straightforward way using the interfaces available in our programming environments.

At extremely sharp tolerances, the first step that encounters the discontinuity can also fail. The solver will still have to retake the step but, as discussed before, it will use the wrong $\beta$ value. In the next few sections, we will present the correct way to code problems with state-dependent discontinuities so that we get accurate solutions efficiently.

\subsection{Introducing event detection}
\label{subsection:intro_event_detection}
For the time-dependent discontinuity problem, we saw that if we used error-controlled software, then the solvers can work through one discontinuity at sufficiently high tolerances. We also showed that this was not the most efficient way for them to solve the problem. For the state-dependent discontinuity problem, we showed in the previous section why the solvers, using even sharp tolerances, will not be able to solve this problem with much accuracy. Because we do not know when the discontinuities occur, we cannot use the discontinuity handling technique, involving a cold restart, that we used to solve the time-dependent discontinuity problem. However, the idea that we developed in Section $\ref{subsection:time_disc_handling}$ about integrating continuous sub-problems separately and combining them into a final solution can still be applied here. 

To integrate continuous sub-problems, we need a way to detect that a threshold has been met, and then as soon as we reach such a point, we can perform a cold start. This will make the solver integrate the problem one continuous subinterval at a time. In this section, we will explain the capability of modern solvers to detect events and we will show how to encode the E(t) thresholds (either E(t)=25000 or E(t)=10000) as events so that the times at which they occur can be determined. We can then perform a cold start at these times.

To perform event detection, an ODE solver will require two functions from the user: the usual ODE right-hand side function, $f(t, y(t))$ and another function which we will call the root function (commonly denoted by $g(t, y(t))$), that determines the events.

The root function is a function that, given the value of the solution to the ODE at the current step will return a real number. The ODE solution is said to have a root whenever the value of the root function is zero. The key idea is that each event must be written so that it occurs at the root of a root function.

The solver calls the root function at the end of each successful step that it takes and will record its value. It will then compare the value of the root function with the corresponding value from the previous step to see if there has been a change of sign. If the value of the root-function has changed sign, the solver raises a flag to say that it has detected a root and will then run a root-finding algorithm on that step to find the point where the root-function equals zero. Most solvers will then return, allowing us to perform a cold start.

Using event detection thus entails defining a function that takes the value of the ODE solution at the current point and returns a real number which is zero whenever we want it to detect an event. For example, if we want to detect when x is 100, it is sufficient to define (x - 100) to be the root function. In the next section, we will elaborate on how to use event detection to accurately and efficiently solve the state-dependent discontinuity problem.

We also mention that many modern solvers have event detection built-in. Thus users should be able to use event-detection solvers from their preferred programming environments without any additional software.

\subsection{Solving the state dependent discontinuity model using event detection}
\label{subsection:state_with_event_detection}
As mentioned earlier, each toggling between the values of the parameter $\beta$ introduces a discontinuity. As none of the provided solvers are designed to solve discontinuous problems, we get the erroneous solutions reported in $\ref{subsection:naive_state_problem}$. We have seen that although sharp tolerances do result in somewhat better solutions being computed, none of the solvers were able to obtain an accurate solution. The use of such sharp tolerances leads to inefficiencies as well. We will now present an approach using event detection that is both accurate and efficient.

The solution is to use the thresholds that we have defined in our model to define events and integrate up to each threshold using the event detection capability of the solver. We can then cold start from there and repeat the process with another right-hand side function corresponding to the new $\beta$ value and with a different root function that encodes the next threshold we are looking for. We repeat this process until we reach the end of the time interval. This approach allows the solvers to integrate continuous sub-problems, one at a time, and these sub-problems can then be combined into a final solution.

For our specific problem, event detection is used as follows:
We start by solving the problem with $\beta$=0.9 and with a root function that detects when E is equal to 25000. Once we detect the time at which E(t)=25000, we do a cold start. We extract the solution of the solver at the time of the event and use that solution as the initial value for our next call to the solver. This next call will have $\beta$ at 0.005 and a root function that detects a root when E(t)=10000. We again integrate up to that new threshold and cold start when we reach it. The new cold start will have $\beta$=0.9 and the root function looking for E(t)=25000 as the event. This is repeated until we reach the desired end time. The pseudo-code is as follows:

\begin{minipage}{\linewidth}
\centering
\begin{lstlisting}[language=Python]
function model_no_measures(t, y):
    beta = 0.9
    // code to get dSdt, dEdt, dIdt, dRdt
    return (dSdt, dEdt, dIdt, dRdt)

function root_25000(t, y):
    E = y[1]
    return E - 25000

function model_with_measures(t, y):
    beta = 0.005
    // code to get dSdt, dEdt, dIdt, dRdt
    return (dSdt, dEdt, dIdt, dRdt)

function root_10000(t, y):
    E = y[1]
    return E - 10000

res = array()
t_initial = 0
y_initial = (S0, E0, I0, R0)
while t_initial < 180:
    tspan = [t_initial, 180]
    if (measures_implemented):
        sol = ode(model_with_measures, tspan, y_initial,
            events=root_10000)
        measures_implemented = False
    else:
        sol = ode(model_no_measures, tspan, y_initial,
            events=root_25000)
        measures_implemented = True
    t_initial = extract_last_t_from_sol(sol)
    y_initial = extract_last_row_from_sol(sol)
    res = concatenate(res, sol)

// use res as the final solution
\end{lstlisting}
\end{minipage}

Some programming environments, such as Python, by default, do not stop the integration when the first event is detected. To do a cold start, we need the solver to stop at events, and to make this happen, in some programming environments we need to set appropriate flags. 

\subsubsection{Solving the state-dependent discontinuity model in R using event detection}
\begin{figure}[H]
\centering
\includegraphics[width=0.7\linewidth]{./figures/solve_state_discontinuity_R}
\caption{Solving state discontinuity model in R}
\label{fig:solve_state_discontinuity_R}
\end{figure}
Several of the solvers in R have event detection capabilities. These are: `adams', `bdf', `lsoda', `Radau', and they will be used in this section to solve the model using the approach described in the previous subsection. From Figure $\ref{fig:solve_state_discontinuity_R}$, we can see that all the solvers give solutions that are in agreement except `Radau'. This is in contrast with what happened previously when we were integrating a discontinuous problem, even at sharp tolerances. 

The case of `Radau' is interesting as it was giving a poor quality solution at the default tolerances, without event detection but it is now giving at least a solution that is exhibiting a correct pattern. We note that at high tolerances `Radau' with event detection approach the results from the other solvers, as shown in Figure $\ref{fig:solve_state_discontinuity_sharp_R}$. We will also note the poor performance of Radau in Table $\ref{tab:state_discontinuity_R}$. We also note that `Radau' Fortran code does not have built-in detection and that the event detection has been added through the C interface, which may explain the slight disparity.

\begin{figure}[H]
\centering
\includegraphics[width=0.7\linewidth]{./figures/solve_state_discontinuity_sharp_R}
\caption{Solving state discontinuity model at sharp tolerances in R}
\label{fig:solve_state_discontinuity_sharp_R}
\end{figure}

We will show in Table $\ref{tab:state_discontinuity_R}$ that introducing event detection also made the solvers significantly more efficient while giving us better results.

We note that it is unfair to compare the efficiency of the solvers at the default tolerances with the efficiency of the solvers when they use event detection as the results for the former are inaccurate.

\begin{table}[h]
\caption {R state discontinuity model} 
\label{tab:state_discontinuity_R}
\begin{center}
\begin{tabular}{ c c c c c } 
method & no event & no event-sharp tol. & with event & with event-sharp tol.\\ 
lsoda & 2135 & 4658 & 1248 & 3435 \\
radau & 1002 & 21835 & 2151 & 14681\\
bdf & 3300 & 9803 & 1678 & 7963\\
adams & 1368 & 3467 & 817 & 2689\\
\end{tabular}
\end{center}
\end{table}

We can see from Table $\ref{tab:state_discontinuity_R}$ that with event detection we are gaining an improvement of around 1000 function evaluations for `lsoda', 7000 in `Radau' (sharp tol comparison), 2000 in `bdf', and 500 in `adams' while having more accuracy. This significant decrease in the number of function evaluations will lead to much faster CPU times, especially when the right-hand side function, $f(t, y)$ is more complex.

Also, we can see from the table that the solvers use fewer function evaluations compared with event detection than without event detection at the default tolerances. When comparing the values at the sharp tolerances, the use of event detection also decreased the respective number of function evaluations.

We also note that the Fortran code for `Radau' does not have event detection and that event detection was added through the R interface.

\subsubsection{Solving the state-dependent discontinuity model in Python using event detection}
\begin{figure}[H]
\centering
\includegraphics[width=0.7\linewidth]{./figures/solve_state_discontinuity_py}
\caption{Solving state dependent discontinuity model in Python}
\label{fig:solve_state_discontinuity_py}
\end{figure}
All the solvers in Python have event detection and thus all will be used in this part of the study. In Python, $solve\_ivp()$ does not stop when an event is detected by default. We thus need to set the terminal flag of the root functions.
(Example: $root\_10000.terminal = True$).
Again, Figure $\ref{fig:solve_state_discontinuity_py}$ shows that all the solvers give solutions that are in agreement, suggesting that this is the correct solution. This is different from our results at sharp tolerances when event detection was not employed. We will also see that this is a much more efficient approach across all the solvers. The $solve\_ivp()$ implementation of `Radau' is in Python itself and thus it is different from the R implementation. We note that we did not have to provide it with a sharp tolerance to make it align with the other solvers, suggesting that the issue in R may be due to the C implementation of event detection.

As is the case with R, we cannot compare the default tolerance efficiency data to the event detection efficiency data as the former corresponds to inaccurate results. So, in Table $\ref{tab:state_discontinuity_Py}$, we compare the sharp tolerance efficiency data with the data from the event detection computation.

\begin{table}[h]
\caption {Python state discontinuity model} \label{tab:state_discontinuity_Py}
\begin{center}
\begin{tabular}{ c c c c } 
method & no event & no event with sharp tol. & with event detection \\ 
lsoda & 2357 & 4282 & 535 \\
bdf & 2301 & 11794 & 808 \\
radau & 211 & 74723 & 990 \\
rk45 & 1484 & 17648 & 674 \\
dop853 & 11129 & 21131 & 1514 \\
rk23 & 4307 & 246644 & 589 \\
\end{tabular}
\end{center}
\end{table}

Table $\ref{tab:state_discontinuity_Py}$ shows that the number of function evaluations when the solvers use event detection is far less when they do not; `LSODA' used around 3000 fewer function evaluations, `BDF' used 11000 less, `Radau' used 74000 less, `RK45' used 17000 less, `DOP853' used 20000 less and `RK23' used 246000 less. The reduction in CPU times from this will be significant across all the solvers, especially with a more complex right-hand side function.

\subsubsection{Solving the state-dependent discontinuity model in Scilab using event detection}
\begin{figure}[H]
\centering
\includegraphics[width=0.7\linewidth]{./figures/solve_state_discontinuity_scilab}
\caption{Solving state discontinuity model in Scilab}
\label{fig:solve_state_discontinuity_scilab}
\end{figure}
There is only one solver with root functionality in Scilab; it is `lsodar', the root-finding version of `lsoda'. Judging from the solutions we obtained from Python and R, it seems that `lsodar' gave a correct solution as well. It oscillates in the correct pattern and goes sharply between 10000 and 25000.

\begin{table}[h]
\caption {Scilab state discontinuity model} \label{tab:state_discontinuity_scilab}
\begin{center}
\begin{tabular}{ c c c c } 
method & no event & no event with sharp tol. & with event detection \\ 
lsoda & 2794 & 4636 & 1327 \\
\end{tabular}
\end{center}
\end{table}

From Table $\ref{tab:state_discontinuity_scilab}$, we can see that the root-finding code uses fewer function evaluations that `lsoda' both at sharp and default tolerances.

\subsubsection{Solving the state-dependent discontinuity model in Matlab using event detection}
\begin{figure}[H]
\centering
\includegraphics[width=0.7\linewidth]{./figures/solve_state_discontinuity_matlab}
\caption{Solving state discontinuity model in Matlab}
\label{fig:solve_state_discontinuity_matlab}
\end{figure}
Both $ode45()$ and $ode15s()$ have an event detection capability. (The root functions need to set that the root is terminal to perform a cold start.) We applied event detection to solve the problem with the solvers in the Matlab environment and the results are shown in Figure $\ref{fig:solve_state_discontinuity_matlab}$. We remember that the solutions in Matlab without event detection were surprisingly accurate but were in disagreement with each other at points further in time. We can see that with event detection, the solutions are all in agreement at the default tolerances even at points further in time. We also see, in Table $\ref{tab:state_discontinuity_matlab}$, that the use of event detection is also more efficient than the computation without event detection.

\begin{table}[h]
\caption {Matlab state discontinuity model problem} \label{tab:state_discontinuity_matlab}
\begin{center}
\begin{tabular}{ c c c c } 
method & no event & no event with sharp tol. & with event detection \\ 
ode45 & 2023 & 22411 & 859 \\
ode15s & 1397 & 11550 & 620 \\
\end{tabular}
\end{center}
\end{table}

We can see in Table $\ref{tab:state_discontinuity_matlab}$ that the computation with event detection uses fewer function evaluations than the code without event detection at default and sharp tolerances. We see that the computations with sharp tolerances, although they give acceptable solutions, use 20000 more function evaluations in $ode45$ than the computation with event detection and 11000 in the case of $ode15s$ than the computation with event detection.

\subsection{Efficiency data and tolerance study for the state dependent discontinuity problem}
\label{subsection:state_tolerance_study}
In this section, we will investigate how sharpening the tolerance improves the results in the case of the non-event detection experiment. We will also investigate coarsening the tolerance with event detection to show how coarse a tolerance we can use while getting acceptable results.

We will perform this analysis on LSODA across R, Python, and Scilab, as they appear to use the same source code, and with R and Python versions of DOPRI5 which do not use the same code but do use the same Runge-Kutta pair and with the Scilab version of RKF45 which is not the same code, nor the same pair but is a Runge-Kutta pair of the same order. We also use $ode45$ in Matlab as it is an implementation of DOPRI5 in Matlab. 

\subsubsection{Comparing LSODA across platforms for the state discontinuous problem}
\subparagraph{State dependent discontinuity LSODA tolerance study in R}
In this section, we use the R version of LSODA at multiple tolerances. We set both the relative and the absolute tolerance to various values and analyze the solution.

We know that without event detection, LSODA does not give accurate results even at very sharp tolerances. We will also examine how coarse we can set the tolerance to still have the event detection computation yield accurate results.

\begin{figure}[h]
\centering
\includegraphics[width=0.7\linewidth]{./figures/tolerance_state_lsoda_no_event_R}
\caption{State dependent discontinuity model tolerance study on the R version of LSODA without event detection}
\label{fig:tolerance_state_lsoda_no_event_R}
\end{figure}

\begin{figure}[h]
\centering
\includegraphics[width=0.7\linewidth]{./figures/tolerance_state_lsoda_with_event_R}
\caption{State dependent discontinuity model tolerance study on the R version of LSODA with event detection}
\label{fig:tolerance_state_lsoda_with_event_R}
\end{figure}

Figure $\ref{fig:tolerance_state_lsoda_no_event_R}$ shows that LSODA applied to the same problem at different tolerances gives vastly different results. We would expect the solutions at the sharper tolerances to be along very similar curves but that is not the case. The computation is suffering from the fact that the first step that encounters a discontinuity fails and switches the global variables. This further supports our statement that for any state-dependent discontinuity, we cannot get reasonable results simply by sharpening the tolerance.

From Figures $\ref{fig:tolerance_state_lsoda_with_event_R}$ and $\ref{fig:tolerance_state_lsoda_no_event_R}$, we can see the clear advantage of using event detection. Event detection even allows us to use very coarse tolerances while solving the problem to a reasonable accuracy. Event detection allows us to use tolerances of $10^{-3}$ and sharper to get reasonable results while the computation without event detection failed even at a tolerance of $10^{-13}$. We also analyze the differences in efficiency between the two codes in Table $\ref{tab:tolerance_state_discontinuity_lsoda_R}$.

\begin{table}[h]
\caption {R version of LSODA applied to state discontinuity model tolerance study} \label{tab:tolerance_state_discontinuity_lsoda_R} 
\begin{center}
\begin{tabular}{ c c c }
tolerance & no event detection & with event detection \\
1e-01 & 675 & 560 \\
1e-02 & 1856 & 522 \\
1e-04 & 1863 & 752 \\
1e-06 & 2135 & 1248 \\
1e-07 & 2676 & 1874 \\
1e-08 & 2730 & 2060 \\
1e-10 & 3337 & 2604 \\
1e-11 & 3603 & 3054 \\
\end{tabular}
\end{center}
\end{table}

Table $\ref{tab:tolerance_state_discontinuity_lsoda_R}$ shows a decrease in the number of function evaluations across all tolerances which will translate into faster CPU times when the right-hand side function is more complex. We note that the comparison is unfair as the computations without event detection do not give a reasonably accurate answer. Furthermore, the latter computation uses more function evaluations. This supports our conclusion that event detection is the appropriate way to solve state-dependent discontinuity problems when the discontinuity can be characterized in terms of an event.

\subparagraph{State dependent discontinuity model LSODA tolerance study in Python}
In this section, we use the Python version of LSODA at multiple tolerances to see how it performs. We note that LSODA without event detection even at very sharp tolerances in Python was still not giving accurate results but we will see how the solutions change as the tolerance is increased. We will also show that coarse tolerances can be used with the computation that uses event detection. 

\begin{figure}[h]
\centering
\includegraphics[width=0.7\linewidth]{./figures/tolerance_state_lsoda_no_event_py}
\caption{State dependent discontinuity model tolerance study on the Python version of LSODA without event detection}
\label{fig:tolerance_state_lsoda_no_event_py}
\end{figure}

\begin{figure}[h]
\centering
\includegraphics[width=0.7\linewidth]{./figures/tolerance_state_lsoda_with_event_py}
\caption{State dependent discontinuity model tolerance study on the Python version of LSODA with event detection}
\label{fig:tolerance_state_lsoda_with_event_py}
\end{figure}

Again Figure $\ref{fig:tolerance_state_lsoda_no_event_py}$ exposes that LSODA applied to the same problem at different tolerances give substantially different results. We would expect the computations at the sharper tolerances to give quite similar results but this is not the case.

From Figures $\ref{fig:tolerance_state_lsoda_with_event_py}$ and $\ref{fig:tolerance_state_lsoda_no_event_py}$, we can see that the addition of event detection allows for the use of a coarser tolerance. We also note that the computations with event detection blur as we go further in time. This is because the coarser tolerance computations are not giving a sufficiently accurate solution. In Python, it is at a tolerance of $10^{-4}$ and sharper that we get reasonably accurate results. 

We analyse the efficiency of the computations in Table $\ref{tab:tolerance_state_discontinuity_lsoda_py}$. We must note that this analysis is unfair as the computation without event detection does not give an accurate solution to the problem. Still, we will see that the event detection computation uses fewer function evaluations while getting a more accurate answer.

\begin{table}[h]
\caption {Python version of LSODA applied to state discontinuity model tolerance study} \label{tab:tolerance_state_discontinuity_lsoda_py} 
\begin{center}
\begin{tabular}{ c c c }
tolerance & no event detection & with event detection \\
0.1 & 1207 & 425 \\
0.01 & 1627 & 454 \\
0.0001 & 1968 & 689 \\
1e-06 & 2122 & 1305 \\
1e-07 & 2684 & 1807 \\
1e-08 & 2730 & 2099 \\
1e-10 & 3337 & 2639 \\
1e-11 & 3603 & 3098 \\
\end{tabular}
\end{center}
\end{table}

\subparagraph{State dependent discontinuity model LSODA tolerance study in Scilab}

We perform the same experiment in Scilab. We set the absolute and relative tolerance to the same values as in the other experiments and run the solvers. For the different tolerance values, we plot the solutions and analyze how the solutions computed without event detection change as the tolerance is sharpened; we also examine how coarse a tolerance we can use with the event detection solver.

\begin{figure}[h]
\centering
\includegraphics[width=0.7\linewidth]{./figures/tolerance_state_lsoda_no_event_sci}
\caption{State dependent discontinuity model tolerance study on the Scilab version of LSODA without event detection}
\label{fig:tolerance_state_lsoda_no_event_sci}
\end{figure}

\begin{figure}[h]
\centering
\includegraphics[width=0.7\linewidth]{./figures/tolerance_state_lsoda_with_event_sci}
\caption{State dependent discontinuity model tolerance study on the Scilab version of LSODA with event detection}
\label{fig:tolerance_state_lsoda_with_event_sci}
\end{figure}

Again, Figure $\ref{fig:tolerance_state_lsoda_no_event_sci}$ exposes the behavior whereby the same solver applied to the same problem at different tolerances gives substantially different results. We would expect the code at the sharper tolerances to give very similar curves but clearly, LSODA even at sharp tolerances does not.

From Figure $\ref{fig:tolerance_state_lsoda_with_event_sci}$, we can see that the use of the event detection allows us to use a smaller tolerance. We can use a tolerance of $10^{-3}$ and still get an accurate answer whereas, without event detection, even tolerance of $10^{-12}$ is not sufficient.

\begin{table}[h]
\caption {Scilab version of LSODA applied to state discontinuity model tolerance study} \label{tab:tolerance_state_discontinuity_lsoda_scilab} 
\begin{center}
\begin{tabular}{ c c c }
tolerance & no event detection & with event detection \\
0.1 & 1141 & 287 \\
0.01 & 1606 & 262 \\
0.0001 & 1968 & 523 \\
0.000001 & 2122 & 983 \\
0.0000001 & 2684 & 1307 \\
1.000D-08 & 2730 & 1567 \\
1.000D-10 & 3380 & 1963 \\
1.000D-11 & 3603 & 2331 \\
\end{tabular}
\end{center}
\end{table}

\subsubsection{Comparing Runge-Kutta pairs across platforms for state discontinuous problem}
In this section, we consider solvers based on Runge-Kutta pairs of the same order: DOPRI5 in R aliased as `ode45', DOPRI5 in Python aliased as `RK45', DOPRI5 in Matlab through the $ode45$ function, and RKF45 in Scilab aliased as `rkf'.

We recall that without event detection, none of these solvers across the platforms solved the problem correctly even with sharp tolerances. We will show what happens to these solvers as the tolerance is sharpened. We also coarsen the tolerance for the case where solvers use event detection where that is possible to see how coarse the tolerance can be while still obtaining sufficient accuracy.

\subparagraph{Tolerance study on state discontinuity using the R version of DOPRI5}
The R version of DOPRI5 does not have event detection but we still perform the experiment on this solver without event detection. We pick several values for the absolute and relative tolerances and run the solvers. In so doing we see how the code performs as the tolerance is sharpened. 

\begin{figure}[h]
\centering
\includegraphics[width=0.7\linewidth]{./figures/tolerance_state_rk45_no_event_R}
\caption{State dependent discontinuity model tolerance study on the R version DOPRI5 without event detection}
\label{fig:tolerance_state_rk45_no_event_R}
\end{figure}

From Figure $\ref{fig:tolerance_state_rk45_no_event_R}$, we see that DOPRI5 applied to the same problem with different tolerances, gives significantly different solutions.

We then report on the efficiency data for this case in Table $\ref{tab:tolerance_state_discontinuity_rk45_R}$. 

\begin{table}[h]
\caption {R version of DOPRI5 state discontinuity model tolerance study} \label{tab:tolerance_state_discontinuity_rk45_R} 
\begin{center}
\begin{tabular}{ c c }
tolerance & no event detection \\
1e-01 & 1082 \\
1e-02 & 1142 \\
1e-04 & 2014 \\
1e-06 & 2027 \\
1e-07 & 2193 \\
1e-08 & 2919 \\
1e-10 & 5194 \\
1e-11 & 7690 \\
\end{tabular}
\end{center}
\end{table}

\subparagraph{Tolerance study on state discontinuity using the Python version of DOPRI5}
We perform the same experiment in Python. The absolute and relative tolerances are set to a range of values and the solver is run both with and without event detection. We report on how the code performs as the tolerance is increased in the case without event detection. Since the Python version of DOPRI5 has event detection, we will see how coarse the tolerance can be set while still giving us a reasonably accurate solution. We note that the solver crashes if we ask for a tolerance of 0.1.

\begin{figure}[h]
\centering
\includegraphics[width=0.7\linewidth]{./figures/tolerance_state_rk45_no_event_py}
\caption{State dependent discontinuity model tolerance study on the Python version of DOPRI5 without event detection}
\label{fig:tolerance_state_rk45_no_event_py}
\end{figure}

\begin{figure}[h]
\centering
\includegraphics[width=0.7\linewidth]{./figures/tolerance_state_rk45_with_event_py}
\caption{State dependent discontinuity model tolerance study on the Python version of DOPRI5 with event detection}
\label{fig:tolerance_state_rk45_with_event_py}
\end{figure}

In Figure $\ref{fig:tolerance_state_rk45_no_event_py}$, we can see that even at sharp tolerances, the solver is not able to compute a reasonably accurate solution.

In contrast, when using event detection, the code can use very coarse tolerances. We can see that a tolerance of $10^{-4}$ is sharp enough to solve the given problem accurately; the blurring that occurs is due to the coarser tolerances. We present the efficiency data in Table $\ref{tab:tolerance_state_discontinuity_rk45_py}$ to show how the code with event detection is also far more efficient.

\begin{table}[h]
\caption {The Python version of DOPRI5 state discontinuity model tolerance study} \label{tab:tolerance_state_discontinuity_rk45_py} 
\begin{center}
\begin{tabular}{ c c c }
tolerance & no event detection & with event detection \\
0.01 & 1400.0 & 664.0 \\
0.0001 & 8462.0 & 806.0 \\
1e-06 & 6248.0 & 1232.0 \\
1e-07 & 6848.0 & 1754.0 \\
1e-08 & 7082.0 & 2354.0 \\
1e-10 & 10262.0 & 5066.0 \\
1e-11 & 13058.0 & 7688.0 \\
\end{tabular}
\end{center}
\end{table}

We can see in Table $\ref{tab:tolerance_state_discontinuity_rk45_py}$ that across all the different tolerances, the solver with event detection requires fewer function evaluations, around several thousand fewer for the sharper tolerances. 

\subparagraph{State dependent discontinuity RKF45 tolerance study in Scilab}
Scilab uses RKF45 which is a different Runge-Kutta pair from what is used in DOPRI5 but the pairs have the same order. It does not have event detection but we can still perform the experiment on the solver without event detection. We pick several values for the absolute and relative tolerances and run the solvers. In so doing we see how the solver performs as the tolerance is sharpened. 

The Scilab version of `rkf' can only integrate up to time 90 as it has a hard cap of 3000 derivative evaluations but this is enough to see that even at sharper tolerances, the solutions are not in agreement. Figure $\ref{fig:tolerance_state_rk45_no_event_sci}$ shows that the problem cannot be solved by simply using sharper tolerances. We can conclude that event detection is required. 

\begin{figure}[h]
\centering
\includegraphics[width=0.7\linewidth]{./figures/tolerance_state_rk45_no_event_sci}
\caption{State dependent discontinuity model tolerance study on the Scilab version of RKF45 without event detection}
\label{fig:tolerance_state_rk45_no_event_sci}
\end{figure}

\begin{table}[h]
\caption {The Scilab version of RKF45 State Discontinuity tolerance study} \label{tab:tolerance_state_discontinuity_rk45_scilab} 
\begin{center}
\begin{tabular}{ c c }
tolerance & no event detection \\ 
0.1 & 547 \\
0.01 & 732 \\
0.001 & 1294 \\
1e-4 & 1956 \\
1e-5 & 2364 \\
1e-6 & 2662 \\
1e-7 & 2802 \\
\end{tabular}
\end{center}
\end{table}

\subparagraph{Tolerance study on state discontinuity using the Matlab version of DOPRI5}
We apply different tolerances to the state problem with and without event detection on the $ode45$ function which is a Matlab implementation of DOPRI5.

\begin{figure}[h]
\centering
\includegraphics[width=0.7\linewidth]{./figures/tolerance_state_rk45_no_event_matlab}
\caption{State dependent discontinuity model tolerance study on the Matlab version of DOPRI5 without event detection}
\label{fig:tolerance_state_rk45_no_event_matlab}
\end{figure}

From Figure $\ref{fig:tolerance_state_rk45_no_event_matlab}$, we can see that the solution obtained with a tolerance of 0.1 is of poor quality without event detection. It does not follow the correct pattern of oscillating between 10000 and 25000. The computations of the other tolerances follow the correct pattern but are not in agreement.

\begin{figure}[h]
\centering
\includegraphics[width=0.7\linewidth]{./figures/tolerance_state_rk45_with_event_matlab}
\caption{State dependent discontinuity Model tolerance study on the Matlab version of DOPRI5 with event detection}
\label{fig:tolerance_state_rk45_with_event_matlab}
\end{figure}
In Figure $\ref{fig:tolerance_state_rk45_with_event_matlab}$, we can see that the computations corresponding to most tolerances give solutions that are in agreement. A tolerance of 0.1 now follows the correct pattern but is not in agreement with the other tolerances at further points in time. For tolerances of $10^{-2}$ and sharper, we get accurate solutions. We also see how event detection allows us to use fewer function evaluations.

\begin{table}[h]
\caption {Matlab DOPRI5 state discontinuity model tolerance study} \label{tab:tolerance_state_discontinuity_rk45_matlab} 
\begin{center}
\begin{tabular}{ c c c }
tolerance & no event detection & with event detection \\
0.1 & 415 & 650 \\
0.01 & 1339 & 661 \\
0.0001 & 4891 & 901 \\
1e-06 & 5803 & 1411 \\
1e-07 & 7225 & 1873 \\
1e-09 & 9739 & 4039 \\
1e-10 & 12385 & 6043 \\
1e-11 & 16357 & 9277 \\
\end{tabular}
\end{center}
\end{table}

Table $\ref{tab:tolerance_state_discontinuity_rk45_matlab}$, although being an unfair comparison since the solver without event detection did not give accurate solutions, shows that this way of solving the problem is also less efficient. At the tolerance of 0.1, the smaller number of function evaluations for the solver without event detection is not relevant since the solution at a tolerance of 0.1 is very inaccurate. At all the other tolerances, the code with event detection is both more accurate and more efficient, usually using less than half the number of function evaluations.



\section{Investigation of the cause of inaccuracies that arise for some of the numerical software packages when they are applied to the models}
\label{section:fortran_inaccuracies}
\subsection{Radau}

\subsection{may have a section on dopri5.f and rkf45.f}



\section{Summary, Conclusions, and Future Work}
\label{section:summary}
\subsection{Summary and Conclusions}
Our starting assumption for both models is a reasonable implementation that might typically be employed by a computational scientist. This includes fixed-step size solvers as well as implementations based on the introduction of if-else statements into the functions that define the ODE systems. 
We reported on the stability and discontinuity issues associated with SEIR models. We showed how stability affects our solutions even if there is a small change in the initial values. We showed how discontinuities reduce the efficiency of the solvers and presented a straightforward way to detect that the problem at hand is discontinuous.

We then used ODE software packages in R, Python, and Scilab to model two Covid-19 problems, one with a time-dependent discontinuity and one with a state-dependent discontinuity.

For the time-dependent discontinuity problem, we have shown that error-controlled ODE solvers can step over one discontinuity with sufficiently sharp tolerances while fixed step-size solvers cannot. We have shown that although error-controlled solvers can solve the problem, the use of discontinuity handling in the form of cold starts leads to more efficient solutions that allow us to use coarser tolerances. 

For the state-dependent discontinuity problem, we have shown that even error control solvers cannot successfully step over multiple discontinuities. We have shown that if the discontinuity is state-dependent, we cannot straightforwardly implement the model using the model function $f(t, y)$. We then introduced event detection and showed how it can be used to model state-dependent discontinuity problems by encoding the thresholds as events and applying cold starts. Using event detection provides an efficient and accurate way to solve such problems.

From the usage of the different packages, we also found a certain inconsistency. We note that R and Scilab do not use the interpolation capabilities for the solver by default. We would advise software implementers to use the capabilities of the solver's interpolation. Using the method of forcing the solver to integrate exactly to given output points reduces the efficiency of the algorithm. The algorithm is no longer allowed to take as big a step as it should.
We also recommend against using fixed step-size solvers.

We recommend using some form of discontinuity handling rather than introducing an if-statement into the right-hand side function that defines the ODE.

When a researcher has a problem that has a time-dependent discontinuity that occurs at a known time, they should use the form of discontinuity handling presented in this report. Using cold starts allows the researcher to integrate continuous subintervals of the problem in separate calls leading, to efficient and accurate solutions.

When a researcher has a problem that has a state-dependent discontinuity, they should map out the thresholds at which these discontinuities occur and look to use event detection with these thresholds as events. They can then cold start at each event and integrate continuous subintervals of the problem in separate calls to the solvers. This leads to efficiency and accuracy that is not possible using a naive treatment. 

\subsection{Future Work}
\label{subsection:future_work}
In Section $\ref{subsection:naive_state_problem}$, we see that `Radau' exhibits unusual behavior when solving the state-dependent problem. Further analysis needs to be done on the algorithm itself as two different implementations of the algorithm gave similarly poor quality solutions.

We also propose to do the same discontinuity analysis on Covid-19 PDE models to see how error-controlled and non-error-controlled PDE solvers differ. We can also use BACOLIKR or other root-finding capable software to analyze how they improve the solutions to discontinuous PDE problems.



\bibliographystyle{ieeetran}
\bibliography{bibliography}
\appendix
\section{Appendix: Parameter fitting in an SEIR Model}
\label{section:ebola_paper}
In \cite{althaus2014estimating}, the epidemiologist uses data from the Ebola spread in three different West African countries to understand the impact of the implemented control measures. To do this, the researcher needed to estimate parameters like the basic and effective reproduction number of the virus.

These parameters are estimated by doing a best fit optimization on the parameters applied to an SEIR model. The experiment is to use an ODE model with certain values of these parameters and calculate the error of these models based on real-life data. The model with the minimum error is the `best fit' model and the corresponding parameters' values are the optimal choices for these parameters. These optimal parameters are then used to understand the spread of the virus. 

We note that the ODE model is run inside an optimization algorithm and thus its efficiency is critical as the algorithm will need to solve the ODE model with each different set of parameters.

The following is the pseudo-code for our attempt at replicating the experiment reported in \cite{althaus2014estimating}:

\begin{minipage}{\linewidth}
\begin{lstlisting}[language=Python]
data = read_csv("ebola_data.csv")

function model(t, y, parms):
    // define the SEIR model
    return (dSdt, dEdt, dIdt, dRdt)

function ssq(parms):
    // get the model
    out = ode(model, initial_value, times, parms)
    // calculate the error from the data points as such:
    ssq = abs(out.C - data.C) + abs(out.D - data.D)
    return ssq

parms = c(beta=0.27, f=0.74, k=0.0023)
// give error function and manipulatable parameters
// to an optimisation algorithm
fit = optimise(par=parms, errorFunc=ssq)

// fit will contain the optimal parameter values...
\end{lstlisting}
\end{minipage}

The figure that was reported in  \cite{althaus2014estimating} is shown in Figure $\ref{fig:original_figure_SEIR_paper}$. Our results are as shown in Figures $\ref{fig:my_fit_Gui}$, $\ref{fig:my_fit_SL}$ and $\ref{fig:my_fit_Lib}$. We see good agreement between our results and those reported in  \cite{althaus2014estimating}.

\begin{figure}[h]
\centering
\includegraphics[width=1\linewidth]{./figures/original_figure_SEIR_paper}
\caption{Original figure in Ebola paper}
\label{fig:original_figure_SEIR_paper}
\end{figure}

\begin{figure}[h]
\centering
\includegraphics[width=0.7\linewidth]{./figures/my_fit_Gui}
\caption{Our Guinea Figure}
\label{fig:my_fit_Gui}
\end{figure}

\begin{figure}[h]
\centering
\includegraphics[width=0.7\linewidth]{./figures/my_fit_SL}
\caption{Our Sierra Leone Figure}
\label{fig:my_fit_SL}
\end{figure}

\begin{figure}[h]
\centering
\includegraphics[width=0.7\linewidth]{./figures/my_fit_Lib}
\caption{Our Liberia Figure}
\label{fig:my_fit_Lib}
\end{figure}



\end{document}
