\documentclass{article}
\usepackage{amssymb}
\usepackage{graphicx}
\usepackage{caption}
\usepackage{subcaption}
\usepackage{listings}
\usepackage{float} %figure inside minipage
\graphicspath{ {./images/} }
\usepackage[export]{adjustbox}
\usepackage{cite}
\usepackage{amsmath}
\usepackage{hyperref}

% gives the \cites command for multiple citations
\makeatletter
\newcommand{\citecomment}[2][]{\citen{#2}#1\citevar}
\newcommand{\citeone}[1]{\citecomment{#1}}
\newcommand{\citetwo}[2][]{\citecomment[,~#1]{#2}}
\newcommand{\citevar}{\@ifnextchar\bgroup{;~\citeone}{\@ifnextchar[{;~\citetwo}{]}}}
\newcommand{\citefirst}{\@ifnextchar\bgroup{\citeone}{\@ifnextchar[{\citetwo}{]}}}
\newcommand{\cites}{[\citefirst}
\makeatother

\begin{document}
\begin{titlepage}
\author{Humaid Agowun and Paul Muir} 
\title{Efficient defect control for IVODES based on multistep Hermite-Birkhoff interpolants
} 
\date{\today} 
\maketitle
\end{titlepage}

\begin{center}
    \textbf{Abstract}
\end{center}

In this report, we consider the concept of `defect control'. We discuss its importance and the efficiency issues associated with ODE solvers based on Runge-Kutta methods \cite{MR3822086} that control the maximum defect of a continuous approximate solution. Standard approaches typically makes use of continuous Runge-Kutta methods \cite{MR3822086} to perform defect control which typically involves performing several evaluations of the right hand side function of the ODE, $f(t, y(t))$. In this chapter, we consider an approach to perform defect control using a multistep Hermite interpolant \cite{MR3822086} that requires no additional function evaluations. We will augment $4^{th}$, $6^{th}$ and $8^{th}$ order Runge Kutta methods with a Hermite cubic, with a sixth order Hermite-Birkhoff interpolant and an eighth order Hermite-Birkhoff interpolant and show that high quality interpolants can be obtained using no extra function evaluations. In its simplest form, a numerical method typically solves an ODE by stepping from the initial time to the final time, using stepsize $h$. It computes a solution approximation at the end of each step. A method is said to be of order $p$, if the error associated with the solution approximations it computes behaves like O($h^p$). We will discuss challenges associated with this approach and how these challenges can be addressed. We then conclude with suggestions for additional work that can be done on this project.



\section{Introduction}
\label{section:intro}
In this report, we will discuss the results of a careful investigation of the performance of a variety of software packages applied to typical initial value ordinary differential equation (IVODEs) encountered in Covid-19 models. 

For any mathematical model, the accuracy requirements of the numerical solution should be determined by the quality of the model and the accuracy of the parameters that appear in the model. Numerical errors associated with the computational techniques that are used to obtain the approximate solution must always be negligible compared to the accuracy to which the model is defined. \emph{Researchers deserve to obtain numerically accurate solutions to the models that they are studying}. In this report, \emph{we will show that the straightforward use of standard IVODE solvers on typical Covid-19 models can lead to numerical solutions that have large errors, sometimes of the same order of magnitude as the solution itself.} Most of the IVODE solvers that we consider in this report allow the user to specify a parameter called a tolerance. The solvers use adaptive algorithms to attempt to compute an approximate solution with a corresponding error estimate that is approximately equal to the tolerance.

In Section $\ref{subsection:research_papers}$, we review examples of how IVODEs are used in epidemiology. In Section $\ref{subsection:SEIR_model}$, we define the SEIR models which we will consider throughout this report. In Section $\ref{subsection:exponential_growth}$, we discuss the numerical stability issues that arise in problems (such as Covid-19 modelling) with exponentially growing solutions. In Section $\ref{subsection:fixed_vs_control}$, we explain the difference between fixed step-size and error-controlled IVODE solvers. The IVODE software packages from programming environments that are typically used by researchers are described in Section $\ref{subsection:numerical_software_used}$. We also make a note of issues with evaluation of approximate solutions at output points that lead to inefficiencies for some of these solvers in Section $\ref{subsection:solution_output_points_impl}$. In Section $\ref{subsection:effect_of_discontinuity}$, we discuss the effects of problem discontinuities on the performance of these solvers.

In Section $\ref{subsection:naive_time_problem}$, we apply the solvers to a Covid-19 problem with a time-dependent discontinuity and show how, in some case, this results in numerical solutions with relative errors of the same magnitude as the solution being computed. In Section $\ref{subsection:time_disc_handling}$, we will use discontinuity handling to accurately solve the time-dependent discontinuity problem. In Section $\ref{subsection:time_tolerance_study}$, we will use a range of tolerances to discuss the effects of tolerance on the accuracy and efficiency of some of the solvers.

In Section $\ref{subsection:naive_state_problem}$, we apply the solvers to a Covid-19 problem with a state-dependent discontinuity and show how when using a straightforward implementation of the problem, none of the solvers are able to obtain accurate solutions. We will explain how even the use of very sharp tolerances is not sufficient to improve the computed solutions in Section $\ref{subsection:state_sharp_tol_failed}$ and show that a more effective way to solve this problem is through the use of event detection, which we will describe in Section $\ref{subsection:intro_event_detection}$. We then show an accurate solution to the state-dependent discontinuity problem in Section $\ref{subsection:state_with_event_detection}$ and perform a tolerance study on this problem in Section $\ref{subsection:state_tolerance_study}$.

In Section $\ref{section:fortran_inaccuracies}$, we examine implementation details for solvers with exceptionally poor solutions to investigate the cause of their errors. We conclude the report in Section $\ref{section:summary}$ with a summary and a discussion of the potential for future work projects.

\subsection{Epidemiological modelling}
\label{subsection:research_papers}
One common form of an epidemiological study is forecasting. Using previously obtained parameters, the researcher develops a mathematical model involving differential equations which are solved using an ODE solver. Often, the solver will be used to integrate over a large time period so that the researcher can examine how the disease will spread. In Section $\ref{subsection:exponential_growth}$, we discuss why it is unrealistic to attempt to compute a numerical solution for large time periods if the infection is still growing exponentially and how measures such as social distancing allow solvers to reduce errors so that reasonably accurate solutions can be computed over longer time periods.

A second type of epidemiology study involves parameter estimation. In this kind of study, data points are collected about the spread of a virus and we try to fit a mathematical model to that data. In so doing, we can estimate values for some modelling parameters that will minimize the error in the fit. An example of such a study can be found in Appendix $\ref{section:ebola_paper}$. Parameter estimation studies often involve using an ODE solver inside an optimization algorithm and thus the computing time, especially for large problems, can be significant. Since the computational cost is typically inversely proportional to the tolerance, we will investigate to what extent coarse tolerances can be employed in the computation of solutions to Covid-19 models.

\subsection{Detailed description of two specific models to be considered in this report.} 
\label{subsection:SEIR_model}
In this section, we explain how an IVODE problem is defined. We then describe the models that we are going to consider in this report. They involve typical SEIR models to which we add discontinuities.

An IVODE problem is defined by the equations and the initial conditions:
\begin{equation}
y'(t) = f(t, y(t)), \quad y(t_0) = y_0 \nonumber
\end{equation}
where $f(t, y(t))$ is a function that defines the derivative at time, t. A complete definition also includes the initial values of the solution components. Given $f(t, y(t))$ and $y(t_0)$, the goal is to find an approximation to  $y(t)$ using numerical methods. 

In this report, we consider the Covid-19 model:
\begin{equation}
\frac{\textit{d}S}{\textit{dt}} = \mu N - \mu S - \frac{\beta}{N}IS, \nonumber
\end{equation}

\begin{equation}
\frac{\textit{d}E}{\textit{dt}} = \frac{\beta}{N}IS - \alpha E - \mu E, \nonumber
\end{equation}

\begin{equation}
\frac{\textit{d}I}{\textit{dt}} = \alpha E - \gamma I - \mu I, \nonumber
\end{equation}

\begin{equation}
\frac{\textit{d}R}{\textit{dt}} = \gamma I - \mu R \nonumber
\end{equation} 

In this SEIR model, we describe the epidemic over time. S is the number of susceptible individuals, E is the number of exposed individuals, I is the number of infected individuals and R is the number of recovered individuals at a point in time. We also use N to represent the population size.
The other parameters in this model are as follows: $\alpha$ is such that $\alpha^{-1}$ is the average incubation period, $\beta$ is the transmission rate, $\gamma$ is the recovery rate and $\mu$ is the birth/death rate. In this report, we assume that all these parameters are known. Our goal is to investigate the performance of IVODE solvers on forms of this problem that has discontinuities. We will see that we can get approximate solutions that are not efficiently computed and/or that may have significant errors. This latter issue can have serious consequences as the computed solution will fail to show the actual impact of the virus corresponding to the actual epidemiology theories behind the mathematical models. These incorrect numerical solutions may lead epidemiologists into reaching incorrect conclusions and thus lead them into questioning the mathematical models themselves when, in fact, it is the solvers that are at fault.

The discontinuities we are going to consider involve the parameter $\beta$.
Before measures such as social distancing, masking, vaccinations, etc., are implemented, $\beta$ has a much higher value than after the measures are introduced. For the purpose of this study, we will use a large $\beta$ value equal to 0.9 before the measures and a small $\beta$ value equal to 0.005 after they are implemented, corresponding to a highly contagious variant and extreme shut down measures, respectively. These choices will come to highlight the different numerical issues as such an abrupt change in a modelling parameter introduces a discontinuity as we will show in Section $\ref{subsection:effect_of_discontinuity}$. We will consider two types of discontinuities. One depends only on $t$; the other depends on the value of one of the solution components. We will refer to the former as a time-dependent discontinuity and the latter as a state-dependent discontinuity.

For the time-dependent discontinuity, we will assume that at some point in time, measures are implemented that will lead to a reduction in the parameter $\beta$. We would like to solve the problem through this discontinuity but as we will show, this discontinuity introduces a numerical issue.

For the state-dependent discontinuity, we consider the following situation. If the population of exposed people reaches a certain maximum threshold, measures are introduced, which decreases the value of $\beta$. This introduces a discontinuity. Then, when the population of exposed people drops below a certain minimum threshold, the measures are relaxed, which increases $\beta$ back to its original value, which introduces another discontinuity. We will try to model this problem through multiple instances of shut-downs followed by periods where measures are relaxed. We consider a case where vaccines are not being used. This leads to setting $\beta$ back to its original value when the measures are removed. We note that each time we change the parameter $\beta$, a discontinuity is introduced and thus this problem is far more discontinuous than the previous one, which had only one discontinuity. For this problem, we show that all the solvers will fail.

The other parameters are assumed to be constant with N = 37,741,000 (the approximate Canadian population size), $\alpha$ = 1/8, $\gamma$ = 0.06, and $\mu$ = 0.01/365. The initial values are E(0) = 103, I(0) = 1, R(0) = 0 and S(0) = N - E(0) - I(0) - R(0). This gives us a complete system of IVODEs that is in a form that can be solved by typical software packages.

\subsection{Exponential growth and the issue of instability for ODEs}
\label{subsection:exponential_growth}
Some of the solution components of the SEIR model exhibit exponential growth over certain time periods. In this section, we discuss exponentially growing solutions and their impact on the accurate computation of a numerical solution. First of all, we give a quick overview of stability for ODEs. Then we will show that the SEIR model is unstable over certain time intervals and how measures such as social distancing can improve the stability. This is important as this essentially means that before measures are implemented, accurate models are for the most part very difficult to obtain but the addition of the measures such as social distancing can lead to the solvers being able to compute more accurate solutions.

The stability of an ODE is associated with the impact of small changes to the initial values of the solution to the problem. An ODE is unstable if a small change in the initial values results in a large change in the solution; otherwise, the ODE is said to be stable.

It is straightforward to see that problems with a solution component that exhibits exponential growth are unstable. As mentioned above, this is the case with some of the solution components of a Covid-19 model. The population of infected people, $I$, grows exponentially as long as no measures are introduced to reduce the spread of the virus. This means that ODE solvers will experience difficulties in obtaining accurate numerical solutions. 

In Figure $\ref{fig:unstability_of_exponential_growth}$, we show exponentially growing solutions corresponding to models with slightly different initial values for E(0). We can see that we get different solutions, that become even more different as time increases.

\begin{figure}[H]
\centering
\includegraphics[width=0.7\linewidth]{./figures/unstability_of_exponential_growth}
\caption{When a solution exhibits exponential growth, relatively small changes in the initial value can eventually lead to much different solution values. Here we assign the initial of only the E-component to 70, 80, .., 120.}
\label{fig:unstability_of_exponential_growth}
\end{figure}

However, when we introduce measures such as social distancing, which corresponds to a smaller $\beta$ value, the solution will exhibit slower exponential growth or can even show exponential decay. A slower exponential growth means that the solution will not be as sensitive to changes to the initial values. Exponential decay is even better as the solutions from different initial values will converge.

Epidemic modeling problems exhibit solutions with this type of behavior. At first, the problem is unstable but as measures are implemented, which lead to exponential decay rather than growth, the problem becomes stable. We show this in Figure $\ref{fig:regain_stability_after_measures}$ for the problem with the time-dependent discontinuity. At first, the solutions diverge when there is exponential growth, but the introduction of measures such as social distancing introduce exponential decay which makes them converge. Thus the measures not only save lives but also improve the capability of solvers to compute accurate solutions.

\begin{figure}[H]
\centering
\includegraphics[width=0.7\linewidth]{./figures/regain_stability_after_measures}
\caption{Unstable solutions in the region [0, 40] becomes stable in the region [40. 90] as measures are implemented.}
\label{fig:regain_stability_after_measures}
\end{figure}

\subsection{Brief overview of numerical software}
We start by explaining how typical solvers attempt to solve an IVODE problem. Given initial values (at the initial time, $t_0$), the solver will use an initial step size, $h$, to compute a solution at time, $t_1 (= t_0 + h)$. Similarly, the solver will attempt to take a sequence of steps until it reaches the end time. High-quality solvers will also employ an interpolation algorithm usually locally within each step to get a continuous numerical solution. We note that a solver is said to have order $p$ if the difference between the true solution and the computed solution is $O(h^p)$.

In the next section, we describe what a solver will attempt to do to improve the accuracy of the computed solution. We then discuss the numerical solvers we are going to use throughout our investigation. We will then provide an additional discussion on the implementation of interpolation to get a continuous numerical solution and how certain programming environments have not set up their ODE solvers to use interpolation (correctly).

\subsubsection{Fixed Step Size and Error Control Solvers}
\label{subsection:fixed_vs_control}
In this section, we explain the role of the tolerance and the difference between fixed step size and adaptive step-size error control solvers.

The tolerance is a measure of how accurate we want the solution computed by the solvers to be. A key point here is that solvers that can take a tolerance as input must have some way of computing an estimate of the error of the solution that they compute. Then that error estimate can be compared with the user-provided tolerance. Generally, an absolute tolerance means that we want the error estimate to be approximately equal to the tolerance, whereas a relative tolerance means that we want the ratio of the error estimate and the computed solution to be approximately equal to the tolerance. This is not always the case as some solvers will use a blended combination of the provided absolute and relative tolerances.

A solver is said to have a fixed step size if the solver begins with an initial step-size and this step-size is used throughout the whole integration. In this case, the solver will step from one point to the next and will not check if the numerical solution it obtains at the end of each step is sufficiently accurate. Thus, the distance between the points, i.e, the step size, is constant throughout the computation.

An error-controlled solver starts with an initial step size but as it takes a step, it will compute an error estimate and, based on the tolerances will repeat the computation with a smaller step-size if the error estimate is larger than the tolerance. It will repeat this process until the error estimate satisfies the given tolerance. Only then will it move to the next step. Thus it reduces the step-size as needed throughout the computation. We note that the error depends on the step-size and that a smaller step-size generally leads to a smaller error. However, a small step-size means that the computation is slower because more steps will be needed and thus if the error estimate is much smaller than the tolerance, a solver will increase the step-size for the next step. This allows it to make sure that the given tolerance is satisfied over the whole problem interval and that as large a step as possible is being taken to optimize the efficiency of the computation.

Error control is not simple to implement. Some researchers may be tempted to write their own solvers. based on a non-error control method like a simple fixed step-size Euler or Runge-Kutta method. We will show, using provided fixed step-size solvers in R, how these solvers simply cannot solve a Covid-19 model with reasonable accuracy. Without error control, these solvers cannot handle the discontinuity and stability issues that are present in these models and they will give very erroneous solutions, often without even a warning that the computed solutions should not be trusted.

\subsubsection{The ODE Solvers}
\label{subsection:numerical_software_used}
The ODE solvers are grouped in the following classes: Runge-Kutta methods, Runge-Kutta pairs, and multi-step methods.

A Runge-Kutta method is a one-step method that uses function evaluations, i.e, evaluations of $f(t, y(t))$, within the step. A solver based on this type of method integrates with a fixed step-size and has no error control. An example is the classical four-stage, fourth-order Runge-Kutta method.

A Runge-Kutta pair uses two Runge-Kutta methods. One of the methods is used to compute a solution and the second method is used to compute an error estimate. A solver that is based on a Runge-Kutta pair resizes the step based on the error estimate, as discussed previously. An example of such a solver is the DOPRI5 solver that uses a fifth-order method for the solution and a fourth-order method for the error estimate.

A multi-step method is a solver that will use a linear combination of the solutions and function values from the current and previous steps to take the next step. An example of such a solver is LSODA. Such solvers obtain an error estimate using two different multi-step methods. They use the error-estimate to control the step-size as discussed above.

\subparagraph{R packages}
Scientists who solve ODE models in R commonly use the deSolve package, \cite{soetaert2010solving}, and the $ode()$ function within it.
$ode()$ provides many numerical methods to solve a problem but we have focused our investigation only on the following popular choices: `lsoda', `daspk', `euler', `rk4', `ode45', `Radau', `bdf' and `adams'. The default method is `lsoda' and the default tolerances are $10^{-6}$ for both the absolute and relative tolerances. We also note that we did not consider the other integrators in the deSolve package like $rkMethod()$, which provides other Runge-Kutta methods, and the other methods which are called by the $ode()$ function itself.

The error control solvers are:
\begin{itemize}
\item `lsoda' calls the Fortran LSODA routine from ODEPACK. It can automatically detect stiffness and choose between a stiff (Backward Differentiation Formula, BDF) and a non-stiff (Adams) solver.

\item `daspk' calls the Fortran DAE solver of the same name.

\item `ode45' calls an implementation of Dormand-Prince (4)5 (DOPRI5) Runge-Kutta pair, written in C.

\item `Radau' calls the Fortran solver RADAU5 which implements the 3-stage RADAU IIA method.

\item `bdf' calls the stiff solver inside the Fortran LSODE package which is based on a family of BDF methods.

\item `adams' calls the non-stiff solver inside the Fortran LSODE package which is based on a family of Adams methods.
\end{itemize}

The fixed step-size solvers are:
\begin{itemize}
\item `euler' calls the classical Euler method which is implemented in C.
\item `rk4' uses the classical Runge-Kutta method of order 4 which is implemented in C. 
\end{itemize}

We will use these two methods to demonstrate what happens when non-error-controlled solvers are applied to Covid-19 models.

We next consider the R interface for handling output. The $ode()$ function is only given a vector of output points.  The function will use an interpolation by default but the interpolation schemes for all the solvers are not implemented in the most efficient way. As a result, the vector of output points affects the efficiency of the solver in a manner as described in Section $\ref{subsection:solution_output_points_impl}$.

\subparagraph{Python packages}
In Python, researchers can use the scipy.integrate package \cite{2020SciPy-NMeth}, and will normally use the $solve\_ivp()$ function due to its newer interface. It lets the user apply the following methods: `RK23', `RK45', `DOP853', `Radau', `BDF' and 'LSODA`. In this report, we will investigate all of these methods. The default solver in $solve\_ivp()$ is `RK45' and the default tolerance is $10^{-3}$ for the relative tolerance and $10^{-6}$ for the absolute tolerance. All of these solvers employ some form of error control:

\begin{itemize}
\item `RK23' uses an explicit Runge-Kutta pair of order 3(2), the Bogacki-Shampine pair of formulas. It is a Python implementation.

\item `RK45' uses the DOPRI5 pair of formulas, an explicit Runge-Kutta pair of order 5(4). It is a Python implementation.

\item `DOP853' uses an explicit Runge-Kutta triple of order 8(5, 3). It is a Python implementation.

\item `Radau' uses the implicit Radau IIA method of order 5. It is a Python implementation of the RADAU5 Fortran solver.

\item `BDF' uses a method based on BDF methods with the order varying automatically from 1 to 5. It is a Python implementation.

\item `LSODA' calls the Fortran LSODA routine from ODEPACK. It can automatically detect stiffness and choose between a stiff (BDF) and a non-stiff (Adams) solver.
\end{itemize}

We note that all solvers in $solve\_ivp()$ have error control and that only 'LSODA' is using the Fortran package itself; the others are a Python implementation and will likely be slower.

We next talk about Python's $solve\_ivp()$ interface. It can integrate using only the initial time and the final time and it will return the output at the end of each successful step. It can also take a $t\_eval$ vector of specified output points. The solver is allowed to take as big a step as needed and required solution approximations are obtained using interpolation. Thus it does not suffer from the inefficiencies described in Section $\ref{subsection:solution_output_points_impl}$. The interface also has a $dense\_output$ flag. This returns an interpolant for the solution over the whole time range.

\subparagraph{Scilab packages}
In Scilab, researchers solve differential equations using a method from the $ode()$ function, \cite{campbell2010modeling}, which has the following methods: `lsoda', `adams', `stiff', `rk', `rkf'. The default integrator is `lsoda'.
Default values for the tolerances are $10^{-5}$ for the relative tolerance and $10^{-7}$ for the absolute tolerance for all solvers used except `rkf' for which the relative tolerance is $10^{-3}$ and the absolute tolerance is $10^{-4}$. All of these solvers are error control solvers.

\begin{itemize}
\item `lsoda' calls the Fortran LSODA routine from ODEPACK. It can automatically detect stiffness and choose between a stiff (BDF) and a non-stiff (Adams) solver.

\item `stiff' calls the stiff solver inside the Fortran LSODE package which is based on a family of BDF methods.

\item `adams' calls the non-stiff solver inside the Fortran LSODE package which is based on a family of Adams methods.

\item `rk' calls an adaptive Runge-Kutta method of order 4. It uses Richardson extrapolation for the error estimation. It is implemented in Fortran in a program called `rkqc.f'.

\item `rkf' calls the Fortran program written by Shampine and Watts based on Fehlberg's Runge-Kutta pair of order 4 and 5 (RKF45) pair. It is implemented in a Fortran program called `rkf45.f'.
\end{itemize}

The $ode()$ function in Scilab takes a vector of time steps and the code uses interpolation or stops the integration at the output points as described in $\ref{subsection:solution_output_points_impl}$ based on the method used. For example Scilab's `rkf' is an interface to an old software package, `rkf45.f' which does not have any interpolation capabilities.

\subparagraph{Matlab packages}
In Matlab, researchers can solve differential equations with the $ode()$ $suite$ \cite{shampine1997matlab} of functions. We will consider two of these: $ode45()$ and $ode15s()$.
Default values for the tolerances are $10^{-3}$ for the relative tolerance and $10^{-6}$ for the absolute tolerance.

\begin{itemize}
\item Using $ode45()$ calls a Matlab implementation of DOPRI5.

\item Using $ode15s()$ employs an algorithm that is a variable-step, variable-order (VSVO) solver based on the numerical differentiation formulas (NDFs) of orders 1 to 5. Optionally, it can use BDF methods but these are usually less efficient.
\end{itemize}

Functions in the $ode$ $suite$ takes the initial and final time only and thus allows a solver to take as big a step as needed. All plots are then done using the plot() function. With such an interface, it does not suffer from the issues discussed in Section $\ref{subsection:solution_output_points_impl}$. 

\subparagraph{How the packages relate}
We tried to find connections across the programming environment where the solvers appear to be using the same source code.
Here is what we found:

In R, Python, and Scilab, the `lsoda' method is a wrapper around the Fortran LSODA code from ODEPACK.

The R `bdf' method is equivalent to the Scilab `stiff' method in that they both use the LSODE code from ODEPACK; however, the Python `BDF' method is a different implementation in Python itself.

The R `adams' method and the Scilab `adams' method are the same since they both use the LSODE code from ODEPACK.

The R and Python Runge Kutta 5(4) pairs are both implementations of DOPRI5 but they have different source code as the version in Python is implemented in Python while the R version is implemented in C. The $ode45()$ function in Matlab is a Matlab implementation of DOPRI5. The Scilab `rkf' method does not use the same pair; it uses the Shampine and Watts implementation of the Fehlberg's Runge-Kutta pair, not the Dormand-Prince pair. 

The Scilab `rk' method, which is of order 4, and the R `rk4' method are not the same solvers. The Scilab `rk' method is adaptive (error-controlled with Richardson extrapolation for the error estimate) whereas the R `rk4' method is a fixed step-size implementation of the classical 4-stage, 4$^{th}$ order Runge-Kutta method.

The R and Python `Radau' methods have different source code as Python implements its own version of RADAU5 while R calls the Fortran code for RADAU5 through a C interface.

\subsection{Observations on obtaining solution approximations at output points}
\label{subsection:solution_output_points_impl}
In this section, we discuss an issue that we encountered with some of the ODE solvers in R and Scilab when it comes to obtaining output. In an ideal scenario, the user's desired output points should not interfere with the efficiency of the solvers. However, in these two platforms, a method for handling output points is used which makes treating a lot of output points very inefficient.

A standard ODE solver works as follows. Using a default initial step-size, the solver will take a step. It will then accept or reject the step based on the tolerance and will adjust the step-size based on this to take the next step or retake the step. This process is repeated until the solver reaches the end of the interval. However, often the users of an ODE solver will require outputs at specific points and these points may lie at points that are internal to the steps. The current state-of-the-art approach to get solution approximations at these output points is to perform a high accuracy interpolation on the given step and to return the value of the interpolant at the required point. The interpolation error is usually at least of order $p$ if the numerical ODE solution is of order $p$. However some solvers use a lower order interpolant. This way the accuracy of the solution approximation at a point that is interior to a step should be comparable to the accuracy of the solution approximation at the end of the step.

Note that the standard ODE solvers only control the error at the end of the step. That is, an error estimate is generated for the solution approximation at the end of the step and the step is accepted if this error estimate satisfies the tolerance. It is hoped that the solution approximations obtained through the use of the interpolant will be of comparable accuracy to the solution approximation at the end of the step. It is typically the case that no error control is actually applied to the continuous solution approximation.

In R and Scilab, the above approach for handling output points is not used in all the solvers. Instead, R and Scilab will use the output points to dictate the step-size. An issue arises when many output points appear between the steps that would normally be taken by the solver. These solvers will use the difference between the current output point and the next output point to determine the step-size. We note that R solvers, like its `ode45' method, does have an interpolant but that their implementation still allows the user defined output points to affect the efficiency of the solver. 

In such approaches, the space between points will limit the maximum step-size that can be taken and will lead to additional function evaluations being performed because the solver needs to compute a solution approximation using the numerical method at each output point. This will lead to a considerable drop in efficiency as we will show; see for example Tables $\ref{tab:tolerance_time_discontinuity_rk45_R}$ and $\ref{tab:tolerance_time_discontinuity_rk45_further_R}$. These tables show that a problem that can be solved with 150 function evaluations will be solved with 500 function evaluations when there are many output points. 

This method of handling output points in which the solver steps to each output point and uses the numerical method itself to compute a solution approximation also means that the accuracy of the solution depends on the space between the output points. Thus, we get the unusual behavior that the accuracy is increased by putting the output points closer together and the accuracy is decreased by putting them further apart. We will point out these inconsistencies as they become relevant later in this report. We also note that spacing the points closer together is not a good way to control the accuracy as it is impossible to know beforehand how close the points should be in order to obtain a desired accuracy.

\begin{figure}[H]
\centering
\includegraphics[width=0.7\linewidth]{./figures/R_ode45_spacing_experiment}
\caption{R `ode45' output point spacing experiment}
\label{fig:ode45_spacing_experiment}
\end{figure}

Figure $\ref{fig:ode45_spacing_experiment}$ shows an experiment where we solve the time-dependent discontinuity Covid-19 problem using the R `ode45' method, which is an implementation of DOPRI5 which has error control, uses interpolation but allows the output points to affect the integration. We set both the absolute and relative tolerance to 0.1 and thus expect low accuracy but very good efficiency. However, the space between the output points becomes the limiting factor for the step-size. The computed solution has undue accuracy and is computed in a very inefficient manner considering the required tolerance. We recorded the number of function evaluations in Table $\ref{tab:ode45_spacing_experiment}$ and it can be seen that the solver is using a lot more function evaluations than are needed to satisfy such a coarse tolerance. In Table $\ref{tab:ode45_spacing_experiment}$, `spacing' refers to the distance between the output points and `nfev' is the number of function evaluations.

\begin{table}[h]
\caption {R DOPRI5 output point spacing experiment} \label{tab:ode45_spacing_experiment} 
\begin{center}
\begin{tabular}{ c c }
spacing & nfev \\ 
1 & 572 \\
3 & 188 \\
5 & 116 \\
7 & 80 \\
\end{tabular}
\end{center}
\end{table}

From Figure $\ref{fig:ode45_spacing_experiment}$ and Table $\ref{tab:ode45_spacing_experiment}$, we note that we did not ask the solver for an accurate solution but it is giving us a solution that is much more accurate than requested when the spacing between the output points is small. This excess accuracy comes at a price of around 500 more function evaluations. Accuracy should ideally be completely determined by the tolerance but using this method of skipping to the output points substantially interferes with this ideal. This results in the solver not being allowed to take as big a step as it should be based on the tolerance, and this leads to substantial inefficiency. 

It is important that users employ the interpolation option for an ODE solver whenever such an option is readily available so that the solvers can run as efficiently as possible. We also reiterate that the interpolant should have an interpolation error that is at least of order $p$ if the ODE solver gives a solution with an error that is of order $p$ so that the interpolation error does not interfere with the accuracy of the numerical solution.

\subsection{Discontinuities and their effects on solvers}
\label{subsection:effect_of_discontinuity}
The main purpose of this report is to discuss how to solve models with discontinuities and how these discontinuities affect the process of computing an accurate numerical solution to the model. In this section, we will show what happens when a solver encounters a discontinuity and how this discontinuity leads to inaccurate solutions.

We first note that one of the core assumptions for all the solvers is that the function $f(t, y(t))$ and a sufficient number of its higher derivatives are continuous. If the right-hand side function is discontinuous, this can have a major (negative) impact on the performance and accuracy of the solvers. 

We will see that discontinuities will have huge impacts on the accuracy and efficiency of the solvers, that some solvers, even with error control, will require an extremely sharp tolerance to step over the discontinuity in a way that allows them to obtain a reasonably accurate solution approximation, and that fixed-step solvers simply cannot solve these problems accurately. 

It is important to note that the step taken by a solver that first meets a discontinuity will almost always fail. This is because in order for the solver to step over a discontinuity, the step size needs to be much smaller than the one that is being used before the discontinuity. The solver will thus have to retake the step with a smaller step size and as long as the error estimate of the step is not small enough, it will need to continue reducing the step. This leads to high numbers of function evaluations near the discontinuity. 

\begin{figure}[h]
\centering
\includegraphics[width=0.7\linewidth]{./figures/lsoda_vs_discontinuity}
\caption{Function evaluations for the Python `LSODA' method for the time-dependent discontinuity problem with a discontinuity at t=27}
\label{fig:lsoda_vs_discontinuity}
\end{figure}

\begin{figure}[h]
\centering
\includegraphics[width=0.7\linewidth]{./figures/dop853_vs_discontinuity}
\caption{Function evaluations for the Python `DOP853' method for the time-dependent discontinuity problem with a discontinuity at t=27}
\label{fig:dop853_vs_discontinuity}
\end{figure}

In Figures $\ref{fig:lsoda_vs_discontinuity}$ and $\ref{fig:dop853_vs_discontinuity}$, we run `LSODA' and `DOP853' from Python on the time-dependent discontinuity problem where the discontinuity is introduced at t=27 and plot the time at which the $i^{th}$ function evaluation occurs. We thus show the spike in the number of function evaluations at the discontinuity as the solvers repeatedly retake the step with smaller and smaller step-sizes.

Following from the above discussion, we also recommend researchers carry out a manual discontinuity detection experiment to see if their model has any discontinuity. For the case where it is not known if a discontinuity is present, a trivial experiment is done by collecting at what time the solver made the $i^{th}$ call to the function that evaluates the right hand side of the ODE. When a plot of the time against the cumulative count of the function calls gives an almost straight vertical line, it indicates that the function was called repeatedly at a specific time and thus that the solver repeatedly changed the step-size in this region to step over a discontinuity. In the remainder of this report, we will outline the ways to accurately and efficiently solve problems with such discontinuities.


\section{Multistep interpolants for zero-cost defect control for a Runge Kutta method}
\label{section:equipping_rk4_with_HBs}
In this section we will consider a multistep interpolant approach, built on the classical $4^{th}$ order Runge Kutta method (RK4), that allows for defect control. We will augment the discrete Runge-Kutta solution with interpolants of $4^{th}$, $6^{th}$ and $8^{th}$ orders respectively and explain the challenges and the efficiency and accuracy of each interpolant. 

We first note that Runge-Kutta methods are very convenient in that they are one step methods. At any point, when taking the next step, the method does not have to take into consideration the size of the previous steps and this is convenient as the solver can choose the size of the next step in the most optimal way based only on how the error estimate compares with the tolerance for the current step. 

The first interpolant that we discuss is a single step interpolant. It is the classical Hermite cubic of order 4. We will show how this interpolant can be used to perform defect control and discuss the limitations of this interpolant. 

We then show how a Hermite-Birkhoff interpolant of $6^{th}$ order can address the issues that we identified for the $4^{th}$ order interpolant. We will give an overview of how a $6^{th}$ order interpolant is derived and show the results of applying defect control using this interpolant to solve the three test problems. 

We will then use a similar approach to derive an $8^{th}$ order Hermite-Birkhoff interpolant and show why the approach used for the $6^{th}$ order interpolant needs to be modified for the $8^{th}$ order. We then discuss another approach to derive an $8^{th}$ order interpolant that addresses the issue with the previous one and show the results of using this $8^{th}$ order interpolant as the basis for computing defect controlled numerical solutions of the three test problems.


\subsection{The Classical $4^{th}$ order RK method with a $4^{th}$ order Hermite Cubic Interpolant}
The Hermite cubic spline is a very widely used interpolant. The basic idea is that we can use the derivative values and not just the solution values to get more data to fit an interpolant. Given points $(t_i, y_i)$ and $(t_{i + 1}, y_{i + 1})$ with derivatives $f_i$ and $f_{i + 1}$ (here $f_i = f(t_i, y_i)$ and $f_{i+1}=f(t_{i+1}, y_{i+1})$), respectively, the interpolant across the step $[t_i, t_{i + 1}]$ of size $h_i$ is defined as:

\begin{equation}
\label{eqn:HB4}
u(t_i + \theta h) = h_{00}(\theta)y_i +  h_ih_{10}(\theta)f_i + h_{01}(\theta)y_{i + 1} + h_ih_{11}(\theta)f_{i + 1}, 
\end{equation}
and its derivative is:
\begin{equation}
u'(t_i + \theta h) = h_{00}'(\theta)y_i/h_i +  h_{10}'(\theta)f_i + h_{01}'(\theta)y_{i + 1}/h_i + h_{11}'(\theta)f_{i + 1}. 
\end{equation}

The quantity $\theta$ is:
\begin{equation}
\label{eqn:HB4_theta}
\theta = (t - t_i) / h_i.
\end{equation}

The functions $h_{00}(\theta)$, $h_{01}(\theta)$, $h_{10}(\theta)$ and $h_{11}(\theta)$ are each cubics defined such that $u(t_i)= y_i$, $u'(t_i) = f_i$, $u(t_{i+1}) = y_{i + 1}$ and $u'(t_{i + 1}) = f_{i + 1}$. When $\theta$ is 0, $t_i + \theta h_i$ is $t_i$ and thus only $h_{00}(0)$ should be 1 and all the others cubic should evaluate to 0. Also only $h_{10}'(0)$ should be 1 and the derivatives of all the other cubics should evaluate to 0. When $\theta$ is 1, $t_i + \theta h_i$ is $t_{i + 1}$ and thus only $h_{01}(1)$ should be 1 and all the other cubics should evaluate to 0. Also only $h_{11}'(1)$ should be 1 and the derivatives of all the other cubics should be 0.

We will assume that each of the cubics has the form $a\theta^3 + b\theta^2 + c\theta + d$, where $a, b, c$ and $d$ are coefficients to be determined, and note that for each cubic we know its value for $\theta$ at 0 and 1 and the values of its derivatives for $\theta$ at 0 and 1. We thus have 4 equations for each cubic. Thus we can solve the system for each cubic to get the values of a, b, c and d for each cubic.

We now note that from equations $\ref{eqn:HB4}$ and $\ref{eqn:HB4_theta}$, we can evaluate both the interpolant and its derivative for any $\theta$ in $[t_i, t_{i + 1}]$ and therefore we can form $\delta(t + \theta h_i) = u'(t_i + \theta h_i) - f(t_i + \theta h_i, u(t_i + \theta h_i))$ which can be used to sample the defect for any $\theta$.

We will show below experimentally that the maximum defect occurs consistently approximately either at $x_i + 0.2h_i$ or at $x_i + 0.8h_i$ so that we just need to sample the defect twice in order to obtain an estimate of the maximum defect on the step.

We now note that the interpolant comes at no additional cost. We only need $(x_i, y_i, f_i)$ and $(x_{i + 1}, y_{i + 1}, f_{i + 1})$ to be stored as the discrete solution approximation are computed by the RK method. No additional stages or function evaluations are required for the construction of the interpolant itself. 

For the remainder of this chapter, we will refer to the Hermite cubic interpolant as `HB4'.

\paragraph{Problem 1 results}
Figures $\ref{fig:rk4_with_hb4_p1_global_defect}$, $\ref{fig:rk4_with_hb4_p1_global_error}$ and $\ref{fig:rk4_with_hb4_p1_scaled_defects}$ shows the results of using RK4 with HB4 on Problem 1. We note that an absolute tolerance of $10^{-6}$ is applied on the maximum defect estimate within the step and this can be observed to occur at $0.2h$ and $0.8h$ for a step of size h. See Figure $\ref{fig:rk4_with_hb4_p1_scaled_defects}$, to see the scaled defect reaching a maximum near these points. (The figure shows the shape of the defect for each of the steps that were taken to solve Problem 1 using RK4 with HB4. All the defects were scaled vertically to be in the range [0, 1] and scaled horizontally so that they map onto [0, 1]. We see that over all steps and problems, the defect has two clear peaks at $0.2h$ and $0.8h$.) We note that we are able to successfully control the defect of the continuous numerical solution using this approach; see Figure $\ref{fig:rk4_with_hb4_p1_global_defect}$. To obtain these results, we sampled the defect at many points within each step.

\begin{figure}[H]
\centering
\includegraphics[width=0.7\linewidth]{./figures/rk4_with_hb4_p1_global_defect}
\caption{Defect across the entire domain for RK4 with HB4 on problem 1 at an absolute tolerance of $10^{-6}$.}
\label{fig:rk4_with_hb4_p1_global_defect}
\end{figure}

\begin{figure}[H]
\centering
\includegraphics[width=0.7\linewidth]{./figures/rk4_with_hb4_p1_global_error}
\caption{Global Error for RK4 with HB4 on problem 1 at an absolute tolerance of $10^{-6}$.}
\label{fig:rk4_with_hb4_p1_global_error}
\end{figure}

\begin{figure}[H]
\centering
\includegraphics[width=0.7\linewidth]{./figures/rk4_with_hb4_p1_scaled_defects}
\caption{Scaled defects over all steps for RK4 with HB4 on problem 1 at an absolute tolerance of $10^{-6}$ mapped onto $[0, 1]$. The defect has the same shape on every step.}
\label{fig:rk4_with_hb4_p1_scaled_defects}
\end{figure}

\paragraph{Problem 2 results}
Figures $\ref{fig:rk4_with_hb4_p2_global_defect}$, $\ref{fig:rk4_with_hb4_p2_global_error}$ and $\ref{fig:rk4_with_hb4_p2_scaled_defects}$ shows the results of using RK4 with HB4 on Problem 2. We note that an absolute tolerance of $10^{-6}$ is applied on the maximum defect estimate within the step and this can be observed to occur at $0.2h$ and $0.8h$ for a step of size, h. See Figure $\ref{fig:rk4_with_hb4_p2_scaled_defects}$, to see the scaled defect reaching a maximum near these points. We note that we are able to successfully control the defect of the continuous numerical solution using this approach; see Figure $\ref{fig:rk4_with_hb4_p2_global_defect}$. For Problem 2, the defect is noisy on small steps and we do not get two clean peaks. However, we note that we quite consistently get the maximum defects at $0.2h$ and $0.8h$ and thus we only require two defect samplings.

\begin{figure}[H]
\centering
\includegraphics[width=0.7\linewidth]{./figures/rk4_with_hb4_p2_global_defect}
\caption{Defect across the entire domain for RK4 with HB4 on problem 2 at an absolute tolerance of $10^{-6}$.}
\label{fig:rk4_with_hb4_p2_global_defect}
\end{figure}

\begin{figure}[H]
\centering
\includegraphics[width=0.7\linewidth]{./figures/rk4_with_hb4_p2_global_error}
\caption{Global Error for RK4 with HB4 on problem 2 at an absolute tolerance of $10^{-6}$. There is more variation in the shape of the defect for this problem but the maximum defect occurs near either $0.2h$ or $0.8h$.}
\label{fig:rk4_with_hb4_p2_global_error}
\end{figure}

\begin{figure}[H]
\centering
\includegraphics[width=0.7\linewidth]{./figures/rk4_with_hb4_p2_scaled_defects}
\caption{Scaled defects over all steps taken for RK4 with HB4 on problem 2 at an absolute tolerance of $10^{-6}$ mapped onto $[0, 1]$. There is more variation in the shape of the defect for this problem but the maximum defect occurs near either $0.2h$ or $0.8h$. }
\label{fig:rk4_with_hb4_p2_scaled_defects}
\end{figure}

\begin{figure}[H]
\centering
\includegraphics[width=0.7\linewidth]{./figures/rk4_with_hb4_p2_scaled_defects_small_steps}
\caption{Scaled defects small steps taken RK4 with HB4 on problem 2 at an absolute tolerance of $10^{-6}$ mapped onto $[0, 1]$.}
\label{fig:rk4_with_hb4_p2_scaled_defects_small_steps}
\end{figure}

\paragraph{Problem 3 results}
Figures $\ref{fig:rk4_with_hb4_p3_global_defect}$, $\ref{fig:rk4_with_hb4_p3_global_error}$ and $\ref{fig:rk4_with_hb4_p3_scaled_defects}$ shows the results of using RK4 with HB4 on Problem 3. We note that an absolute tolerance of $10^{-6}$ is applied on the maximum defect within the step and this can be shown to occur at $0.2h$ and $0.4h$ along a step of size, h. See Figure $\ref{fig:rk4_with_hb4_p3_scaled_defects}$, to see the scaled defect reaching a maximum near these points. We note that we are able to successfully control the defect of the continuous numerical solution using this approach, see Figure $\ref{fig:rk4_with_hb4_p3_global_defect}$. 

\begin{figure}[H]
\centering
\includegraphics[width=0.7\linewidth]{./figures/rk4_with_hb4_p3_global_defect}
\caption{Defect across the entire domain for RK4 with HB4 on problem 3 at an absolute tolerance of $10^{-6}$.}
\label{fig:rk4_with_hb4_p3_global_defect}
\end{figure}

\begin{figure}[H]
\centering
\includegraphics[width=0.7\linewidth]{./figures/rk4_with_hb4_p3_global_error}
\caption{Global Error of RK4 with HB4 for problem 3 at an absolute tolerance of $10^{-6}$.}
\label{fig:rk4_with_hb4_p3_global_error}
\end{figure}

\begin{figure}[H]
\centering
\includegraphics[width=0.7\linewidth]{./figures/rk4_with_hb4_p3_scaled_defects}
\caption{Scaled defects over all steps taken for RK4 with HB4 on problem 3 at an absolute tolerance of $10^{-6}$ mapped onto $[0, 1]$. The shape of the defect stays relatively similar but there are two peaks, one at $0.2h$ and another at $0.8h$.}
\label{fig:rk4_with_hb4_p3_scaled_defects}
\end{figure}

\paragraph{Efficiency data and discussion of the interpolation error}
We now present the number of steps that were taken by the solver to solve each problem along with the number of successful steps.

\begin{table}[h]
\caption {Number of steps taken by RK4 using defect control with HB4.} \label{tab:rk4_with_hb4_nsteps}
\begin{center}
\begin{tabular}{ c c c } 
Problem & successful steps & total steps \\ 
1       & 88                         & 88 \\ 
2       & 59                         & 62 \\
3       & 225                        & 232 \\
\end{tabular}
\end{center}
\end{table}

The Hermite cubic, HB4, is an interpolant of $4^{th}$ order. However, to perform defect control we need to calculate the derivative of this interpolant. Since we are differentiating the interpolant, the order of the derivative is 3. The numerical solution is of order 4 as we are using RK4 and thus the derivative of the interpolant is less accurate than the ODE solution. We need a way to get an interpolant whose derivative is at least of order 4 so that the error of the derivative of the interpolant is not larger than the error of the discrete numerical solution obtained from the RK4 method.

To do that, in the next section, we will introduce a new interpolation scheme based on a Hermite-Birkhoff interpolant which is of $6^{th}$ order and will thus have a derivative of order 5.

\subsection{RK4 with a $6^{th}$ order Hermite-Birkhoff interpolant}
We first start by noting that this interpolant is a multistep interpolant as it depends on the previous step.

Suppose that the step taken by a solver to go from $t_i$ to $t_{i + 1}$ is of size $h$. We can define the size of the step from $t_{i - 1}$ to $t_i$ by using a weight $\alpha$ such that the size of the step from $t_{i - 1}$ to $t_i$ is $\alpha$h. Then given the solution values and the derivative values at all the three points, i.e, $(t_{i-1}, y_{i - 1}, f_{i - 1})$, $(t_i, y_i, f_i)$ and $(t_{i + 1}, y_{i + 1}, f_{i + 1})$, we can fit a two-step quintic interpolant of order 6 defined as such:
\begin{equation}
\begin{split}
u(t_i + \theta h) = d_{0}(\theta) y_{i-1} +  h_id_{1}(\theta)f_{i-1} \\
+ d_{2}(\theta)y_i     +  h_id_{3}(\theta)f_i
+ d_{4}(\theta)y_{i + 1} + h_id_{5}(\theta)f_{i + 1}, \\
\end{split}
\end{equation}
and its derivative is
\begin{equation}
\begin{split}
u'(t_i + \theta h) = d_{0}'(\theta)y_{i-1}/h_i +  d_{1}'(\theta)f_{i-1} \\
+ d_{2}'(\theta)y_i/h_i     +  d_{3}'(\theta)f_i
+ d_{4}'(\theta)y_{i + 1}/h_i + d_{5}'(\theta)f_{i + 1}. \\
\end{split}
\end{equation}
As before, $\theta$ is:
\begin{equation}
\theta = (t - t_i) / h_i.
\end{equation}

This time $\theta$ is allowed to vary between -$\alpha$ and 1 such that $t_i + \theta$h is $t_{i - 1}$ when $\theta$ is $-\alpha$, $t_i$ when $\theta$ is 0 and $t_{i + 1}$ when $\theta$ is 1.

Each of $d_0(\theta)$, $d_1(\theta)$, $d_2(\theta)$, $d_3(\theta)$, $d_4(\theta)$, and $d_5(\theta)$ is a quintic of the form $a\theta^5 + b\theta^4 + c\theta^3 + d\theta^2 + e\theta + f$ where the six coefficients for each can be found by solving a linear system of 6 equations in terms of $\alpha$. The six equations are obtained as follows. First, for $\theta = -\alpha$, only $d_0(\theta)$ evaluates to 1 and all the other quintic polynomials evaluate to 0 as $u(t_i - \alpha h) = u(t_{i - 1}) = y_{i - 1}$. Also at this $\theta$, only the derivative of $d_1(\theta)$ evaluates to 1 and all the other quintic polynomials' derivatives evaluate to 0 as $u'(t_i - \alpha h) = u'(t_{i - 1}) = f_{i - 1}$. When $\theta$ is 0, only $d_2(\theta)$ evaluates to 1 and all the other polynomials evaluate to 0 as $u(t_i - 0(h)) = u(t_i) = y_i$. Also at this $\theta$ value, only the derivative of $d_3(\theta)$ evaluates to 1 and all the other quintic polynomial derivatives evaluate to 0 as $u'(t_i - 0(h)) = u'(t_i) = f_i$. When $\theta$ is 1, only $d_4(\theta)$ evaluates to 1 and all the other polynomials evaluate to 0 as $u(t_i + 1(h)) = u(t_{i+1}) = y_{i+1}$. Also at this $\theta$, only the derivative of $d_5(\theta)$ evaluates to 1 and all the other quintic polynomial derivatives evaluate to 0 as $u'(t_i - 1(h)) = u'(t_{i+1}) = f_{i+1}$. With these six conditions, we can get six equations for each quintic in terms of $\alpha$ and, using a symbolic management package, we can solve all of these to find the six quintic polynomials. (Their coefficients will be given in terms of $\alpha$.)

We again note that as the solver is stepping across the problem domain, these interpolants are constructed for no additional cost in terms of evaluation of $f(t, y(t))$. We just need to store $(t_{i-1}, y_{i - 1}, f_{i - 1})$, $(t_i, y_i, f_i)$ and $(t_{i + 1}, y_{i + 1}, f_{i + 1})$ as these quantities are being computed by the RK method. We will also observe that the defect peaks at two positions within the new step, $[t_i, t_{i+1}]$, and thus, we can find the maximum defect by only sampling the defect twice. This technique provides an efficient defect control of a continuous approximate solution.

The interpolant defined as above will be referred to as `HB6' for the remainder of this chapter. We note that it is of order 6 and its derivative is of order 5.

We now note that for RK4 as the solution values are only accurate to $4^{th}$ order. We want the derivative of the interpolant to be of order 4 or higher so that interpolation error is relatively negligible. This scheme satisfies this condition and we will see below how this allows us to take fewer time steps to solve a given problem than when HB4 was used with RK4.

\paragraph{Problem 1 results}
Figures $\ref{fig:rk4_with_hb6_p1_global_defect}$, $\ref{fig:rk4_with_hb6_p1_global_error}$ and $\ref{fig:rk4_with_hb6_p1_scaled_defects}$ shows the results of using RK4 with HB6 on Problem 1. We note that an absolute tolerance of $10^{-6}$ is applied on the maximum defect within the step and this can be shown to occur at $0.3h$ and $0.8h$ along a step of size, h. See Figure $\ref{fig:rk4_with_hb6_p1_scaled_defects}$, to see the scaled defect reaching a maximum near these points. We note that we are able to successfully control the defect of the continuous numerical solution using this approach; see Figure $\ref{fig:rk4_with_hb6_p1_global_defect}$. 

\begin{figure}[H]
\centering
\includegraphics[width=0.7\linewidth]{./figures/rk4_with_hb6_p1_global_defect}
\caption{Defect across the entire domain for RK4 with HB6 on problem 1 at an absolute tolerance of $10^{-6}$.}
\label{fig:rk4_with_hb6_p1_global_defect}
\end{figure}

\begin{figure}[H]
\centering
\includegraphics[width=0.7\linewidth]{./figures/rk4_with_hb6_p1_global_error}
\caption{Global Error for RK4 with HB6 on problem 1 at an absolute tolerance of $10^{-6}$.}
\label{fig:rk4_with_hb6_p1_global_error}
\end{figure}

\begin{figure}[H]
\centering
\includegraphics[width=0.7\linewidth]{./figures/rk4_with_hb6_p1_scaled_defects}
\caption{Scaled defects for RK4 with HB6 on problem 1 at an absolute tolerance of $10^{-6}$ mapped into $[0, 1]$.}
\label{fig:rk4_with_hb6_p1_scaled_defects}
\end{figure}

\paragraph{Problem 2 results}
Figures $\ref{fig:rk4_with_hb6_p2_global_defect}$, $\ref{fig:rk4_with_hb6_p2_global_error}$ and $\ref{fig:rk4_with_hb6_p2_scaled_defects}$ shows the results of using the modification of RK4 with HB6 on Problem 2. We note that an absolute tolerance of $10^{-6}$ is applied on the maximum defect within the step and this can be shown to occur at $0.8h$ along a step of size, h. See Figure $\ref{fig:rk4_with_hb6_p2_scaled_defects}$, to see the scaled defect reaching a maximum near these points. We note that we are able to successfully control the defect of the continuous numerical solution using this approach, see Figure $\ref{fig:rk4_with_hb6_p2_global_defect}$. For Problem 2, the defect is noisy on small steps and we do not get two clean peaks. However, we note that we quite consistently get the maximum defects at $0.4h$ and $0.8h$ and thus we only require two samplings.

\begin{figure}[H]
\centering
\includegraphics[width=0.7\linewidth]{./figures/rk4_with_hb6_p2_global_defect}
\caption{Defect across the entire domain for RK4 with HB6 on problem 2 at an absolute tolerance of $10^{-6}$.}
\label{fig:rk4_with_hb6_p2_global_defect}
\end{figure}

\begin{figure}[H]
\centering
\includegraphics[width=0.7\linewidth]{./figures/rk4_with_hb6_p2_global_error}
\caption{Global Error for RK4 with HB6 on problem 2 at an absolute tolerance of $10^{-6}$.}
\label{fig:rk4_with_hb6_p2_global_error}
\end{figure}

\begin{figure}[H]
\centering
\includegraphics[width=0.7\linewidth]{./figures/rk4_with_hb6_p2_scaled_defects}
\caption{Scaled defects for RK4 with HB6 on problem 2 at an absolute tolerance of $10^{-6}$ mapped into $[0, 1]$.}
\label{fig:rk4_with_hb6_p2_scaled_defects}
\end{figure}

\begin{figure}[H]
\centering
\includegraphics[width=0.7\linewidth]{./figures/rk4_with_hb6_p2_scaled_defects_small_steps}
\caption{Scaled defects for RK4 with HB6 on small steps on problem 2 at an absolute tolerance of $10^{-6}$ mapped into $[0, 1]$. Despite the noise, the maximum defect occurs near $0.8h$.}
\label{fig:rk4_with_hb6_p2_scaled_defects_small_steps}
\end{figure}

\paragraph{Problem 3 results}
Figures $\ref{fig:rk4_with_hb6_p3_global_defect}$, $\ref{fig:rk4_with_hb6_p3_global_error}$ and $\ref{fig:rk4_with_hb6_p3_scaled_defects}$ shows the results of using RK4 with HB6 on Problem 3. We note that an absolute tolerance of $10^{-6}$ is applied on the maximum defect within the step and this can be shown to occur at $0.8h$ along a step of size, h. See Figure $\ref{fig:rk4_with_hb6_p3_scaled_defects}$, to see the scaled defect reaching a maximum near these points. We note that we are able to successfully control the defect of the continuous numerical solution using this approach, see Figure $\ref{fig:rk4_with_hb6_p3_global_defect}$. 


\begin{figure}[H]
\centering
\includegraphics[width=0.7\linewidth]{./figures/rk4_with_hb6_p3_global_defect}
\caption{Defect across the entire domain for RK4 with HB6 on problem 3 at an absolute tolerance of $10^{-6}$.}
\label{fig:rk4_with_hb6_p3_global_defect}
\end{figure}

\begin{figure}[H]
\centering
\includegraphics[width=0.7\linewidth]{./figures/rk4_with_hb6_p3_global_error}
\caption{Global Error for RK4 with HB6 on problem 3 at an absolute tolerance of $10^{-6}$.}
\label{fig:rk4_with_hb6_p3_global_error}
\end{figure}

\begin{figure}[H]
\centering
\includegraphics[width=0.7\linewidth]{./figures/rk4_with_hb6_p3_scaled_defects}
\caption{Scaled defects for RK4 with HB6 on problem 3 at an absolute tolerance of $10^{-6}$ mapped into $[0, 1]$.}
\label{fig:rk4_with_hb6_p3_scaled_defects}
\end{figure}

We note that the defects are not as clean they were in the case with HB4. There are two peaks most of the time at around $0.4h$ and $0.8h$ but as was the case in the third problem, the peak sometimes appears at $0.6h$. However, we can see that the defect is still being controlled. We will also see that it is twice as fast as it uses around half the number of steps as HB4.

\begin{table}[h]
\caption {Number of steps taken by RK4 when modified to do defect control with HB6 vs HB4.} \label{tab:rk4_with_hb6_nsteps}
\begin{center}
\begin{tabular}{ c c c c c } 
Problem & succ. steps HB4 & succ. steps HB6 & nsteps HB4  & nsteps HB6 \\ 
1       & 88                 &        27          & 88         & 27 \\ 
2       & 59                 &        36          & 62         & 40 \\
3       & 225                &        62          & 232        & 73 \\
\end{tabular}
\end{center}
\end{table}

Table $\ref{tab:rk4_with_hb6_nsteps}$ shows how the number of steps is less than half when we use HB6 as opposed to HB4. This is entirely because the error of the interpolant and its derivative are smaller than the error of the discrete solution values at the end of each step.

\paragraph{Issues with $\alpha$ values}
The issue with this scheme is that the interpolant is a multistep interpolant while the Runge-Kutta method is a one step method. The Hermite Birkhoff interpolant, HB6, is based on two steps and the parameter $\alpha$ defines how big the previous step is compared to the actual step. The error term in the Hermite-Birkhoff interpolant is minimised when the size of the two steps are the same size, i.e, when $\alpha$ is 1. The error term is proportional to $(t + \alpha)^2t^2(t - 1)^2$. When $\alpha$ differs from 1, the accuracy of the interpolant is reduced.

In Figure $\ref{fig:hb6_alpha_v_shape}$, we show the results of a simple experiment to illustrate how the accuracy of this scheme depends on the value of $\alpha$. We place 3 data points along the t-axis such that their distances apart are $\alpha$h and h for several values of $\alpha$ and we vary h from 1 to $10^{-7}$; we then report on the maximum defect for each h.

\begin{figure}[H]
\centering
\includegraphics[width=0.7\linewidth]{./figures/hb6_alpha_v_shape}
\caption{HB6 maximum defect based on different values of $\alpha$.}
\label{fig:hb6_alpha_v_shape}
\end{figure}

Figure $\ref{fig:hb6_alpha_v_shape}$ shows that at $\alpha$ = 2, $\frac{1}{2}$, 4 and $\frac{1}{4}$, the defect are comparable to those that we get when $\alpha$ is at 1. However, we see that $\alpha = 256$ and $\frac{1}{256}$, the smallest defect that we can obtain for any $h$ value is much larger.

We note that in solving the 3 test problems, $\alpha$ is very rarely bigger than 4 or smaller than $\frac{1}{4}$ and thus, we can be satisfied with the approach that we have considered in this section. We discuss the situation further in Section $\ref{section:keeping_alpha_at_1}$. We note that in order for the results given in Figure $\ref{fig:hb6_alpha_v_shape}$ to be relevant, we would have to be considering a very sharp tolerance since for reasonable values of $\alpha$, e.g, $2$, $\frac{1}{2}$, $4$ and $\frac{1}{4}$, the interpolants all deliver very small defects of $10^{-12}$ to $10^{-14}$.

Another idea would be to use an even higher order interpolant so as to reduce the interpolation error more. We note that with with the approach we consider here, there is no additional cost to get the higher order interpolant. In the next section, we discuss an $8^{th}$ order interpolant and show how to derive such an interpolant.

\subsection{RK4 with an $8^{th}$ order Hermite-Birkhoff interpolant}
\label{section:HB8_derivation}
\paragraph{Derivation of HB8}
In this section, we discuss a derivation of an $8^{th}$ order interpolant. To derive an 8 order interpolant, we need 4 data points over 3 steps. We need the data values to be $(t_i, y_i, f_i)$, as well as the two previous steps $(t_{i-1}, y_{i-1}, f_{i-1})$ and $(t_{i-2}, y_{i-2}, f_{i-2})$, and the right hand end of the current step step, $(t_{i+1}, y_{i+1}, f_{i+1})$. We present two schemes: 
\begin{itemize}
\item The first scheme has the parameters $\alpha$ and $\beta$ are associated with the previous two steps, so the three steps are of the size $\beta h$, $\alpha h$ and $h$ respectively. Thus the scheme establishes the base step-size to be that of the third step. 

\item The second scheme has the parameter $\alpha$ associated with the rightmost step and the parameter $\beta$ associated with the previous step; the middle step is the base step. Thus the sizes for the steps are $\alpha h$, $h$ and $\beta h$. 

\end{itemize}


\paragraph{First HB8 Scheme}
In the first scheme, the step sizes are $\beta h$, $\alpha h$ and $h$ respectively. The interpolant defined on $(t_{i-2}, y_{i-2}, f_{i-2})$, $(t_{i-1}, y_{i-1}, f_{i-1})$, $(t_i, y_i, f_i)$ and $(t_{i + 1}, y_{i + 1}, f_{i + 1})$ is defined as such:

\begin{equation}
\begin{split}
u(t_i + \theta h) = d_{0}(\theta)y_{i-2} +  h_id_{1}(\theta)f_{i-2} 
+ d_{2}(\theta)y_{i-1}     +  h_id_{3}(\theta)f_{i-1} \\
+ d_{4}(\theta)y_i     +  h_id_{5}(\theta)f_i 
+ d_{6}(\theta)y_{i + 1} + h_id_{7}(\theta)f_{i + 1}, \\
\end{split}
\end{equation}
and the derivative is:
\begin{equation}
\begin{split}
u'(t_i + \theta h) = d_{0}'(\theta)y_{i-2}/h_i +  d_{1}'(\theta)f_{i-2} 
+ d_{2}'(\theta)y_{i-1}/h_i   +  d_{3}'(\theta)f_{i-1} \\
+ d_{4}'(\theta)y_i/h_i       +  d_{5}'(\theta)f_i 
+ d_{6}'(\theta)y_{i + 1}/h_i +  d_{7}'(\theta)f_{i + 1}. \\
\end{split}
\end{equation}
Again $\theta$ is:
\begin{equation}
\theta = (t - t_i) / h_i.
\end{equation}

This time $\theta$ is allowed to vary between $-\alpha-\beta$ and 1 such that $t_i + \theta$h is $t_{i-2}$ when $\theta$ is $-\alpha-\beta$, $t_{i-1}$ when $\theta$ is $-\alpha$, $t_i$ when $\theta$ is 0 and $t_{i + 1}$ when $\theta$ is 1. Also $d_0(\theta)$, $d_1(\theta)$, $d_2(\theta)$, $d_3(\theta)$, $d_4(\theta)$, $d_5(\theta)$, $d_6(\theta)$ and $d_7(\theta)$ are all septic polynomials that will each have 8 coefficients.

Each septic polynomial is assumed to have the form $a\theta^7 + b\theta^6 + c\theta^5 + d\theta^4 + e\theta^3 + f\theta^2 + g\theta + h$ where the eight coefficients for each can be found in terms of $\alpha$ and $\beta$ by solving a linear system of 8 equations in terms of $\alpha$ and $\beta$. First at $\theta = -\alpha-\beta$, only $d_0(\theta)$ evaluates to 1 and all the other septic polynomials evaluate to 0 as $u(t_i - (\alpha+\beta) h) = u(t_{i - 2}) = y_{i - 2}$. Also at this $\theta$ value, only the derivative of $d_1(\theta)$ evaluates to 1 and all the other septic polynomial derivatives evaluate to 0 as $u'(t_i - (\alpha+\beta) h) = u'(t_{i - 2}) = f_{i - 2}$. When $\theta = -\alpha$, only $d_2(\theta)$ evaluates to 1 and all the other septic polynomials evaluate to 0 as $u(t_i - \alpha h) = u(t_{i - 1}) = y_{i - 1}$. Also at this $\theta$ value, only the derivative of $d_3(\theta)$ evaluates to 1 and all the other septic polynomial derivatives evaluate to 0 as $u'(t_i - \alpha h) = u'(t_{i - 1}) = f_{i - 1}$. When $\theta$ is 0, only $d_4(\theta)$ evaluates to 1 and all the other polynomials evaluate to 0 as $u(t_i - 0(h)) = u(t_i) = y_i$. Also at this $\theta$ value, only the derivative of $d_5(\theta)$ evaluates to 1 and all the other septic polynomial derivatives evaluate to 0 as $u'(t_i - 0(h)) = u'(t_i) = f_i$. When $\theta$ is 1, only $d_6(\theta)$ evaluates to 1 and all the other polynomials evaluate to 0 as $u(t_i + 1(h)) = u(t_{i+1}) = y_{i+1}$. Also at this $\theta$ value, only the derivative of $d_7(\theta)$ evaluates to 1 and all the other septic polynomial derivatives evaluate to 0 as $u'(t_i - 1(h)) = u'(t_{i+1}) = f_{i+1}$. With these eight conditions, we can get eight equations for each polynomial in terms of $\alpha$ and $\beta$ and, using a symbolic management package, we can solve all of these to find the eight septic polynomials.

We again note that as the solver is stepping through the problem, these interpolants can be obtained without the need for any extra evaluations of $f$. We just need to store the 4 data points $(t_{i-1}, y_{i - 1}, f_{i - 1})$, $(t_{i-1}, y_{i - 1}, f_{i - 1})$, $(t_i, y_i, f_i)$ and $(t_{i + 1}, y_{i + 1}, f_{i + 1})$.

The interpolant defined as above will be referred to as `HB8 First Scheme' for the remainder of this chapter. We note that it is of order 8 and its derivative is of order 7.

Unfortunately, this scheme has a serious issue. The accuracy of the interpolant is very sensitive to a slight change in $\alpha$ and/or $\beta$. This is because the error term is now proportional to $(t + \alpha + \beta)^2(t+\alpha)^2(t)^2(t-1)^2$. We note that the first term depends on both $\alpha$ and $\beta$ and thus very small deviations of these values from 1 will result in reduced accuracy. 

In Figure $\ref{fig:hb8_first_scheme_alpha_beta_test}$, we present the results of a simple experiment to illustrate this issue. We place 4 data points along the t-axis such that their distances apart are $\beta h$, $\alpha h$ and $h$ for several values of $\alpha$ and $\beta$ and we vary h from 1 to $10^{-10}$; we then report on the order of the maximum defect at each h for these values.

\begin{figure}[H]
\centering
\includegraphics[width=0.7\linewidth]{./figures/hb8_first_scheme_alpha_beta_test}
\caption{HB8 First Scheme - maximum size of the defect based on different values of $\alpha$ and $\beta$.}
\label{fig:hb8_first_scheme_alpha_beta_test}
\end{figure}

From Figure $\ref{fig:hb8_first_scheme_alpha_beta_test}$, we can see the issue with this scheme. It only works if both $\alpha$ and $\beta$ are 1 and even small deviations from 1 for either $\alpha$ or $\beta$ drastically reduces the accuracy. More importantly, we can never halve the step with this method as if either $\alpha$ or $\beta$ is 2, the interpolant is not accurate to even one order of magnitude. We will now consider a second scheme which is more stable with respect to changes in $\alpha$ and $\beta$.

\paragraph{Second HB8 Scheme}
In this second scheme, we still use a 3 step interpolant with 4 data points $(t_{i-1}, y_{i-1}, f_{i-1})$, $(t_{i-2}, y_{i-2}, f_{i-2})$, $(t_{i}, y_{i}, f_{i})$ and $(t_{i+1}, y_{i+1}, f_{i+1})$ but now the distance between the points are $\alpha h$, $h$ and then $\beta h$. The middle step is the base step. This way the error term is approximately proportional to $(t- (1+\alpha))^2(t-1)^2t^2(t+\beta)^2$. We avoid the $(t-(\alpha+\beta))^2$ factor. We will show that this scheme is more resilient to changes in $\alpha$ and $\beta$.

Its derivation is very similar to the first scheme. The equation for the interpolant is:
\begin{equation}
\begin{split}
u(t_i + \theta h) = d_{0}(\theta)y_{i-2} +  h_id_{1}(\theta)f_{i-2} 
+ d_{2}(\theta)y_{i-1}     +  h_id_{3}(\theta)f_{i-1} \\
+ d_{4}(\theta)y_i     +  h_id_{5}(\theta)f_i 
+ d_{6}(\theta)y_{i + 1} + h_id_{7}(\theta)f_{i + 1}, \\
\end{split}
\end{equation}
and its derivative is:
\begin{equation}
\begin{split}
u'(t_i + \theta h) = d_{0}'(\theta)y_{i-2}/h_i +  d_{1}'(\theta)f_{i-2}
+ d_{2}'(\theta)y_{i-1}/h_i   +  d_{3}'(\theta)f_{i-1} \\
+ d_{4}'(\theta)y_i/h_i       +  d_{5}'(\theta)f_i
+ d_{6}'(\theta)y_{i + 1}/h_i +  d_{7}'(\theta)f_{i + 1}. \\
\end{split}
\end{equation}
Again $\theta$ is:
\begin{equation}
\theta = (t - t_i) / h_i.
\end{equation}

This time $\theta$ is allowed to vary between $-1-\alpha$ and $\beta$ such that $t_i + \theta h$ is $t_{i-2}$ when $\theta$ is $-1-\alpha$, $t_{i-1}$ when $\theta$ is -1, $t_i$ when $\theta$ is 0 and $t_{i + 1}$ when $\theta$ is $\beta$. Also $d_0(\theta)$, $d_1(\theta)$, $d_2(\theta)$, $d_3(\theta)$, $d_4(\theta)$, $d_5(\theta)$, $d_6(\theta)$ and $d_7(\theta)$ are all septic polynomials and will each have 8 conditions from which we can build a system to find their coefficients in terms of $\alpha$ and $\beta$.

Each is a septic of the form $a\theta^7 + b\theta^6 + c\theta^5 + d\theta^4 + e\theta^3 + f\theta^2 + g\theta + h$ where the eight coefficients for each can be found in terms of $\alpha$ and $\beta$ by solving a linear system of 8 equations in terms of $\alpha$ and $\beta$. First at $\theta = -1-\alpha$, only $d_0(\theta)$ evaluates to 1 and all the other septic polynomials evaluate to 0 as $u(t_i - (1+\alpha) h) = u(t_{i - 2}) = y_{i - 2}$. Also at this $\theta$ value, only the derivative of $d_1(\theta)$ evaluates to 1 and all the other septic polynomial derivatives evaluate to 0 as $u'(t_i - (1+\alpha) h) = u'(t_{i - 2}) = f_{i - 2}$. When $\theta = -1$, only $d_2(\theta)$ evaluates to 1 and all the other septic polynomials evaluate to 0 as $u(t_i - 1(h)) = u(t_{i - 1}) = y_{i - 1}$. Also at this $\theta$ value, only the derivative of $d_3(\theta)$ evaluates to 1 and all the other septic polynomials' derivatives evaluate to 0 as $u'(t_i - 1(h)) = u'(t_{i - 1}) = f_{i - 1}$. When $\theta$ is 0, only $d_4(\theta)$ evaluates to 1 and all the other polynomials evaluate to 0 as $u(t_i - 0(h)) = u(t_i) = y_i$. Also at this $\theta$ value, only the derivative of $d_5(\theta)$ evaluates to 1 and all the other septic polynomials' derivatives evaluate to 0 as $u'(t_i - 0(h)) = u'(t_i) = f_i$. When $\theta$ is $\beta$, only $d_6(\theta)$ evaluates to 1 and all the other polynomials evaluate to 0 as $u(t_i + \beta h) = u(t_{i+1}) = y_{i+1}$. Also at this $\theta$ value, only the derivative of $d_7(\theta)$ evaluates to 1 and all the other septic polynomial derivatives evaluate to 0 as $u'(t_i - \beta h) = u'(t_{i+1}) = f_{i+1}$. With these eight conditions, we can get eight equations for each septic in terms of $\alpha$ and $\beta$ and using a symbolic management package, we can solve all of these to find the 8 coefficients for each septic polynomial.

We now perform a simple experiment to show that the resultant interpolant is more resilient to changes in $\alpha$ and $\beta$. We place 4 data points along the x-axis such that their distance apart are $\alpha h$, $h$ and $\beta h$ for several values of $\alpha$ and $\beta$ and we vary $h$ from 1 to $10^{-10}$, we then report on the maximum defect at each $h$ for these values.

\begin{figure}[H]
\centering
\includegraphics[width=0.7\linewidth]{./figures/hb8_second_scheme_alpha_beta_test}
\caption{HB8 Second Scheme - maximum order of accuracy based on different values of alpha and beta.}
\label{fig:hb8_second_scheme_alpha_beta_test}
\end{figure}

From Figure $\ref{fig:hb8_second_scheme_alpha_beta_test}$, we can see how this scheme is better than the first scheme. We can use $\alpha$ and $\beta$ equal to 2 and to 1/2 and still get a maximum defect of around $10^{-12}$ and we can even be accurate to $10^{-10}$ for both $\alpha$ and $\beta$ equal to 4 and $\frac{1}{4}$. 

For the remainder of this chapter, we will denote this $8^{th}$ order interpolant by HB8. Its derivative has order 7 and any subsequent higher derivative will have one less order.

\paragraph{Results}
We will now use RK4 with the second HB8 scheme and use the new defect control solver to solve the three test problems. We will show that we need to sample the defect only twice to estimate the maximum defect and that this scheme can provide good quality defect control.

\paragraph{Problem 1 results}
Figures $\ref{fig:rk4_with_hb8_p1_global_defect}$, $\ref{fig:rk4_with_hb8_p1_global_error}$ and $\ref{fig:rk4_with_hb8_p1_scaled_defects}$ shows the results of using RK4 with HB8 on Problem 1. We note that an absolute tolerance of $10^{-6}$ is applied on the maximum defect within the step and this can be shown to occur at $0.3h$ and $0.8h$ along a step of size, h. See Figure $\ref{fig:rk4_with_hb8_p1_scaled_defects}$, to see the scaled defect reaching a maximum near these points. We note that we are able to successfully control the defect of the continuous numerical solution using this approach, see Figure $\ref{fig:rk4_with_hb8_p1_global_defect}$. 

\begin{figure}[H]
\centering
\includegraphics[width=0.7\linewidth]{./figures/rk4_with_hb8_p1_global_defect}
\caption{Defect across the entire domain for RK4 with HB8 on problem 1 at an absolute tolerance of $10^{-6}$.}
\label{fig:rk4_with_hb8_p1_global_defect}
\end{figure}

\begin{figure}[H]
\centering
\includegraphics[width=0.7\linewidth]{./figures/rk4_with_hb8_p1_global_error}
\caption{Global Error for RK4 with HB8 on problem 1 at an absolute tolerance of $10^{-6}$.}
\label{fig:rk4_with_hb8_p1_global_error}
\end{figure}

\begin{figure}[H]
\centering
\includegraphics[width=0.7\linewidth]{./figures/rk4_with_hb8_p1_scaled_defects}
\caption{Scaled defects for RK4 with HB8 on problem 1 at an absolute tolerance of $10^{-6}$ mapped into $[0, 1]$.}
\label{fig:rk4_with_hb8_p1_scaled_defects}
\end{figure}

\paragraph{Problem 2 results}
Figures $\ref{fig:rk4_with_hb8_p2_global_defect}$, $\ref{fig:rk4_with_hb8_p2_global_error}$ and $\ref{fig:rk4_with_hb8_p2_scaled_defects}$ shows the results of using RK4 with HB8 on Problem 2. We note that an absolute tolerance of $10^{-6}$ is applied on the maximum defect within the step and this can be shown to occur at $0.8h$ along a step of size, h. See Figure $\ref{fig:rk4_with_hb8_p2_scaled_defects}$, to see the scaled defect reaching a maximum near these points. We note that we are able to successfully control the defect of the continuous numerical solution using this approach, see Figure $\ref{fig:rk4_with_hb8_p2_global_defect}$. 

\begin{figure}[H]
\centering
\includegraphics[width=0.7\linewidth]{./figures/rk4_with_hb8_p2_global_defect}
\caption{Defect across the entire domain for RK4 with HB8 on problem 2 at an absolute tolerance of $10^{-6}$.}
\label{fig:rk4_with_hb8_p2_global_defect}
\end{figure}

\begin{figure}[H]
\centering
\includegraphics[width=0.7\linewidth]{./figures/rk4_with_hb8_p2_global_error}
\caption{Global Error for RK4 with HB8 on problem 2 at an absolute tolerance of $10^{-6}$.}
\label{fig:rk4_with_hb8_p2_global_error}
\end{figure}

\begin{figure}[H]
\centering
\includegraphics[width=0.7\linewidth]{./figures/rk4_with_hb8_p2_scaled_defects}
\caption{Scaled defects of RK4 with HB8 for problem 2 at an absolute tolerance of $10^{-6}$ mapped into $[0, 1]$.}
\label{fig:rk4_with_hb8_p2_scaled_defects}
\end{figure}

\paragraph{Problem 3 results}
Figures $\ref{fig:rk4_with_hb8_p3_global_defect}$, $\ref{fig:rk4_with_hb8_p3_global_error}$ and $\ref{fig:rk4_with_hb8_p3_scaled_defects}$ shows the results of using RK4 with HB8 on Problem 3. 
We note that an absolute tolerance of $10^{-6}$ is applied on the maximum defect within the step and this can be shown to occur at $0.3h$ or $0.8h$ along a step of size, h.  See Figure $\ref{fig:rk4_with_hb8_p3_scaled_defects}$, to see the scaled defect reaching a maximum near these points. We note that we are able to successfully control the defect of the continuous numerical solution using this approach, see Figure $\ref{fig:rk4_with_hb8_p3_global_defect}$. 

\begin{figure}[H]
\centering
\includegraphics[width=0.7\linewidth]{./figures/rk4_with_hb8_p3_global_defect}
\caption{Defect across the entire domain of RK4 with HB8 on problem 3 at an absolute tolerance of $10^{-6}$.}
\label{fig:rk4_with_hb8_p3_global_defect}
\end{figure}

\begin{figure}[H]
\centering
\includegraphics[width=0.7\linewidth]{./figures/rk4_with_hb8_p3_global_error}
\caption{Global Error of RK4 with HB8 on problem 3 at an absolute tolerance of $10^{-6}$.}
\label{fig:rk4_with_hb8_p3_global_error}
\end{figure}

\begin{figure}[H]
\centering
\includegraphics[width=0.7\linewidth]{./figures/rk4_with_hb8_p3_scaled_defects}
\caption{Scaled defects of RK4 with HB8 on problem 3 at an absolute tolerance of $10^{-6}$ mapped onto $[0, 1]$.}
\label{fig:rk4_with_hb8_p3_scaled_defects}
\end{figure}

We note that the defects are not as clean they were in the case with HB4. There are two peaks most of the time at around $0.3h$ and $0.8h$ but as was the case for the third problem in the previous tests, the peak sometimes appears at $0.6h$. However, we can see that the defect is still being controlled and thus that the error is still being controlled. We will also see that it is twice as fast as it uses around half the number of steps as HB4.

\begin{table}[h]
\caption {Number of steps taken by RK4 when modified to do defect control with HB8 vs when modified with HB6.} \label{tab:rk4_with_hb8_nsteps}
\begin{center}
\begin{tabular}{ c c c c c } 
Problem & succ. steps HB8 & succ. steps HB6 & nsteps HB8 & nsteps HB6 \\ 
1       & 20                 &        27          & 20         & 27\\ 
2       & 37                 &        36          & 60         & 40\\
3       & 69                 &        62          & 89         & 73\\
\end{tabular}
\end{center}
\end{table}

From Table $\ref{tab:rk4_with_hb8_nsteps}$, we can see that the number of steps with HB6 and with HB8 are relatively similar. This indicates that the interpolation error is no longer the limiting factor, even in HB6. The limiting factor is the discrete numerical solution which is as required. Thus though we can use RK4 with HB8 at the same cost as modifying RK4 with HB6, using HB8 does not improve the efficiency. Furthermore, HB8 is less stable to changes in $\alpha$ and $\beta$ than HB6 is to changes to $\alpha$. Thus RK4 is best augmented with HB6. However HB8 provides a new opportunity, we can now augment a $6^{th}$ order Runge Kutta method and possibly an $8^{th}$ order Runge-Kutta method, but in the latter case, the interpolation error will still affect the accuracy. Augmenting higher order methods with our HB6 and HB8 schemes is significant because as we have discussed in Section $\ref{section:crk_related_work}$ when using continuous Runge-Kutta solvers to obtain the interpolants, the number of stages grows exponentially with the order of the method. Our scheme is zero-cost and thus effective defect control using this approach will be relatively more efficient.


\subsection{RK4 with an $10^{th}$ order Hermite-Birkhoff interpolant}
\label{section:HB10_derivation}
\paragraph{Derivation of HB10 - the one that worked}


\paragraph{HB10 which by using HB8 in the middle}
We tried several HB10 scheme and the one that worked is created as follows. 
We extend HB8 second scheme by breaking its middle step and using the first half of the split middle step as the base step. So we break the step between $x_{i-1}$ and $x_i$ with an $x_{i - \frac{1}{2}}$. We then use the embedded HB8 that has been extended to calculate $y_{i- \frac{1}{2}}$ and make a function evaluation with these values to get $f_{i- \frac{1}{2}}$ to get 5 data points on which to build the interpolant.

The interpolant is thus defined over 4 steps of size $\alpha h$, $h$, $h$ and $\beta h$ over the data points $(x_{i-2}, y_{i-2}, f_{i-2})$, $(x_{i-1}, y_{i-1}, f_{i-1})$, $(x_{i-\frac{1}{2}}, y_{i-\frac{1}{2}}, f_{i-\frac{1}{2}})$, $(x_i, y_i, f_i)$ and $(x_{i + 1}, y_{i + 1}, f_{i + 1})$ is defined as such:

\begin{equation}
\begin{split}
u(t_{i-\frac{1}{2}} + \theta h) =
      d_0(\theta)y_{i-2}           + h_id_1(\theta)f_{i-2}\\
    + d_2(\theta)y_{i-1}           + h_id_3(\theta)f_{i-1}\\
    + d_4(\theta)y_{i-\frac{1}{2}} + h_id_5(\theta)f_{i-\frac{1}{2}}\\
    + d_6(\theta)y_i               + h_id_7(\theta)f_i \\
    + d_8(\theta)y_{i+1}           + h_id_9(\theta)f_{i+1}\\
\end{split}
\end{equation}
and the derivative is:
\begin{equation}
\begin{split}
u'(t_{i-\frac{1}{2}} + \theta h) = 
      d_0'(\theta)y_{i-2}/h_i           + d_1'(\theta)f_{i-2}\\
    + d_2'(\theta)y_{i-1}/h_i           + d_3'(\theta)f_{i-1}\\
    + d_4'(\theta)y_{i-\frac{1}{2}}/h_i + d_5'(\theta)f_{i-\frac{1}{2}}\\
    + d_6'(\theta)y_i/h_i               + d_7'(\theta)f_i \\
    + d_8'(\theta)y_{i+1}/h_i           + d_9'(\theta)f_{i+1}\\
\end{split}
\end{equation}
Again $\theta$ is:
\begin{equation}
\theta = (t - x_{i-\frac{1}{2}}) / h_i.
\end{equation}

The step between $x_{i-1}$ and $x_{i-\frac{1}{2}}$ is the base step with $\theta=0$ at $x_\frac{1}{2}$. This way the error terms have terms in $1 + \alpha$ and $1 + \beta$ instead of $2 + \alpha$ on one side and $\beta$ on another side. Not having $2 + \alpha$ reduces the error term. 

This time $\theta$ is allowed to vary between $-1-\alpha$ and $1+\beta$ such that $t_{i-\frac{1}{2}} + \theta$h is $t_{i-2}$ when $\theta$ is $-1-\alpha$, $t_{i-1}$ when $\theta$ is $-1$, $t_{i-\frac{1}{2}}$ at $\theta$ is 0, $t_i$ when $\theta$ is $1$ and $t_{i + 1}$ when $\theta$ is $1+\beta$. Also $d_0(\theta)$, $d_1(\theta)$, $d_2(\theta)$, $d_3(\theta)$, $d_4(\theta)$, $d_5(\theta)$, $d_6(\theta)$, $d_7(\theta)$, $d_8(\theta)$ and $d_9(\theta)$ are all nonic polynomials that will each have 10 coefficients.

Each nonic polynomial is assumed to have the form $a\theta^9 + b\theta^8 + c\theta^7 + d\theta^6 + e\theta^5 + f\theta^4 + g\theta^3 + h\theta^2 + i\theta + j$ where the 10 coefficients for each can be found in terms of $\alpha$ and $\beta$ by solving a linear system of 10 equations in terms of $\alpha$ and $\beta$. 


First at $\theta = -1-\alpha$, only $d_0(\theta)$ evaluates to 1 and all the other nonic polynomials evaluate to 0 as $u(t_{i-\frac{1}{2}} - (1+\alpha) h) = u(t_{i - 2}) = y_{i - 2}$.  Also at this $\theta$ value, only the derivative of $d_1(\theta)$ evaluates to 1 and all the other nonic polynomial derivatives evaluate to 0 as $u'(t_{i-\frac{1}{2}} - (1+\alpha) h) = u'(t_{i - 2}) = f_{i - 2}$. 

When $\theta = -1$, only $d_2(\theta)$ evaluates to 1 and all the other nonic polynomials evaluate to 0 as $u(t_{i-\frac{1}{2}} - h) = u(t_{i - 1}) = y_{i - 1}$. Also at this $\theta$ value, only the derivative of $d_3(\theta)$ evaluates to 1 and all the other nonic polynomial derivatives evaluate to 0 as $u'(t_{i-\frac{1}{2}} - h) = u'(t_{i - 1}) = f_{i - 1}$. 

When $\theta$ is 0, only $d_4(\theta)$ evaluates to 1 and all the other nonic polynomials evaluate to 0 as $u(t_{i-\frac{1}{2}} - 0(h)) = u(t_{i-\frac{1}{2}}) = y_{i-\frac{1}{2}}$. Also at this $\theta$ value, only the derivative of $d_5(\theta)$ evaluates to 1 and all the other nonic polynomial derivatives evaluate to 0 as $u'(t_{i-\frac{1}{2}} - 0(h)) = u'(t_{i-\frac{1}{2}}) = f_{i-\frac{1}{2}}$. 

When $\theta$ is 1, only $d_6(\theta)$ evaluates to 1 and all the other nonic polynomials evaluate to 0 as $u(t_{i-\frac{1}{2}} + 1(h)) = u(t_{i}) = y_{i}$. Also at this $\theta$ value, only the derivative of $d_7(\theta)$ evaluates to 1 and all the other nonic polynomial derivatives evaluate to 0 as $u'(t_{i-\frac{1}{2}} - 1(h)) = u'(t_{i}) = f_{i}$. 

When $\theta$ is $1 + \beta$, only $d_8(\theta)$ evaluates to 1 and all the other nonic polynomials evaluate to 0 as $u(t_{i-\frac{1}{2}} + (1 + \beta)(h)) = u(t_{i+1}) = y_{i+1}$. Also at this $\theta$ value, only the derivative of $d_9(\theta)$ evaluates to 1 and all the other nonic polynomial derivatives evaluate to 0 as $u'(t_{i-\frac{1}{2}} - (1 + \beta)(h)) = u'(t_{i+1}) = f_{i+1}$. 

With these 10 conditions, we can get 10 equations for each polynomial in terms of $\alpha$ and $\beta$ and, using a symbolic management package, we can solve all of these to find the 10 nonic polynomials.

The interpolant defined as above using the embedded HB8 to get $y_{i-\frac{1}{2}}$ and an additional function evaluation at $f_{i-\frac{1}{2}}$ will be referred to as HB10 for the remainder of this chapter. We note that it is of order 10 and its derivative is of order 9.

In Figure $\ref{fig:hb10_alpha_beta_test}$, we see that the interpolant is relatively accuracte with a variety of different $\alpha$ and $\beta$ values and the scheme can be used to do defect control.

\begin{figure}[H]
\centering
\includegraphics[width=0.7\linewidth]{./figures/hb10_alpha_beta_test}
\caption{HB10 - maximum size of the defect based on different values of $\alpha$ and $\beta$.}
\label{fig:hb10_alpha_beta_test}
\end{figure}


\paragraph{Results}
\paragraph{Problem 1 results}
Figures $\ref{fig:rk4_with_hb10_p1_global_defect}$, $\ref{fig:rk4_with_hb10_p1_global_error}$ and $\ref{fig:rk4_with_hb10_p1_scaled_defects}$ shows the results of using RK4 with HB10 on Problem 1. We note that an absolute tolerance of $10^{-6}$ is applied on the maximum defect within the step and this can be shown to occur at $0.3h$ and $0.8h$ along a step of size, h. See Figure $\ref{fig:rk4_with_hb10_p1_scaled_defects}$, to see the scaled defect reaching a maximum near these points. We note that we are able to successfully control the defect of the continuous numerical solution using this approach, see Figure $\ref{fig:rk4_with_hb10_p1_global_defect}$. 

\begin{figure}[H]
\centering
\includegraphics[width=0.7\linewidth]{./figures/rk4_with_hb10_p1_global_defect}
\caption{Defect across the entire domain for RK4 with HB10 on problem 1 at an absolute tolerance of $10^{-6}$.}
\label{fig:rk4_with_hb10_p1_global_defect}
\end{figure}

\begin{figure}[H]
\centering
\includegraphics[width=0.7\linewidth]{./figures/rk4_with_hb10_p1_global_error}
\caption{Global Error for RK4 with HB10 on problem 1 at an absolute tolerance of $10^{-6}$.}
\label{fig:rk4_with_hb10_p1_global_error}
\end{figure}

\begin{figure}[H]
\centering
\includegraphics[width=0.7\linewidth]{./figures/rk4_with_hb10_p1_scaled_defects}
\caption{Scaled defects for RK4 with HB10 on problem 1 at an absolute tolerance of $10^{-6}$ mapped into $[0, 1]$.}
\label{fig:rk4_with_hb10_p1_scaled_defects}
\end{figure}

\paragraph{Problem 2 results}
Figures $\ref{fig:rk4_with_hb10_p2_global_defect}$, $\ref{fig:rk4_with_hb10_p2_global_error}$ and $\ref{fig:rk4_with_hb10_p2_scaled_defects}$ shows the results of using RK4 with HB10 on Problem 2. We note that an absolute tolerance of $10^{-6}$ is applied on the maximum defect within the step and this can be shown to occur at $0.8h$ along a step of size, h. See Figure $\ref{fig:rk4_with_hb10_p2_scaled_defects}$, to see the scaled defect reaching a maximum near these points. We note that we are able to successfully control the defect of the continuous numerical solution using this approach, see Figure $\ref{fig:rk4_with_hb10_p2_global_defect}$. 

\begin{figure}[H]
\centering
\includegraphics[width=0.7\linewidth]{./figures/rk4_with_hb10_p2_global_defect}
\caption{Defect across the entire domain for RK4 with HB10 on problem 2 at an absolute tolerance of $10^{-6}$.}
\label{fig:rk4_with_hb10_p2_global_defect}
\end{figure}

\begin{figure}[H]
\centering
\includegraphics[width=0.7\linewidth]{./figures/rk4_with_hb10_p2_global_error}
\caption{Global Error for RK4 with HB10 on problem 2 at an absolute tolerance of $10^{-6}$.}
\label{fig:rk4_with_hb10_p2_global_error}
\end{figure}

\begin{figure}[H]
\centering
\includegraphics[width=0.7\linewidth]{./figures/rk4_with_hb10_p2_scaled_defects}
\caption{Scaled defects of RK4 with HB10 for problem 2 at an absolute tolerance of $10^{-6}$ mapped into $[0, 1]$.}
\label{fig:rk4_with_hb10_p2_scaled_defects}
\end{figure}

\paragraph{Problem 3 results}
Figures $\ref{fig:rk4_with_hb10_p3_global_defect}$, $\ref{fig:rk4_with_hb10_p3_global_error}$ and $\ref{fig:rk4_with_hb10_p3_scaled_defects}$ shows the results of using RK4 with HB10 on Problem 3. 
We note that an absolute tolerance of $10^{-6}$ is applied on the maximum defect within the step and this can be shown to occur at $0.3h$ or $0.8h$ along a step of size, h.  See Figure $\ref{fig:rk4_with_hb10_p3_scaled_defects}$, to see the scaled defect reaching a maximum near these points. We note that we are able to successfully control the defect of the continuous numerical solution using this approach, see Figure $\ref{fig:rk4_with_hb10_p3_global_defect}$. 

\begin{figure}[H]
\centering
\includegraphics[width=0.7\linewidth]{./figures/rk4_with_hb10_p3_global_defect}
\caption{Defect across the entire domain of RK4 with HB10 on problem 3 at an absolute tolerance of $10^{-6}$.}
\label{fig:rk4_with_hb10_p3_global_defect}
\end{figure}

\begin{figure}[H]
\centering
\includegraphics[width=0.7\linewidth]{./figures/rk4_with_hb10_p3_global_error}
\caption{Global Error of RK4 with HB10 on problem 3 at an absolute tolerance of $10^{-6}$.}
\label{fig:rk4_with_hb10_p3_global_error}
\end{figure}

\begin{figure}[H]
\centering
\includegraphics[width=0.7\linewidth]{./figures/rk4_with_hb10_p3_scaled_defects}
\caption{Scaled defects of RK4 with HB10 on problem 3 at an absolute tolerance of $10^{-6}$ mapped onto $[0, 1]$.}
\label{fig:rk4_with_hb10_p3_scaled_defects}
\end{figure}


\begin{table}[h]
\caption {Number of steps taken by RK4 when modified to do defect control with HB8 vs when modified with HB10.} \label{tab:rk4_with_hb10_nsteps}
\begin{center}
\begin{tabular}{ c c c c c } 
Problem & succ. steps HB8 & succ. steps HB10 & nsteps HB8 & nsteps HB10 \\ 
1       & 20                 &        24          & 20         & 25\\ 
2       & 37                 &        50          & 60         & 50\\
3       & 69                 &        93          & 89         & 97\\
\end{tabular}
\end{center}
\end{table}

HB10 seems to be worse than HB8. We will explain this in Section $\ref{section:crks_vs_hbs}$. This is because at coarser tolerance HB10 works worse than HB8. This is because though the order is $O(h^{10})$ versus $O(h^8)$, the coefficient for HB10 is much worse. Thus it is only when the step sizes are smaller that HB10 works better than HB8.




\section{Higher Order Runge Kutta Methods}
\label{section:HBs_and_higher_order_RK}
In this section, we attempt to perform defect control based on efficient multistep interpolants for higher order Runge-Kutta methods. We recall that related previous work used significantly more stages to obtain a continuous $6^{th}$ order Runge-Kutta method and a continuous $8^{th}$ order Runge-Kutta method.

In this section, we will first augment the RK6 method (see Section $\ref{section:basic_runge_kutta}$ for details about the discrete method) with HB6 and then with HB8. We hope to perform defect control and to find that the use of HB8 allows significantly fewer steps. We will then augment the RK8 method (see Section $\ref{section:basic_runge_kutta}$ for more details) with HB8 to show that though interpolation error is present, the scheme does allow defect control of a continuous $8^{th}$ order solution. 

For both methods, we will sample the defect only twice in a step, at $0.4h$ and $0.8h$ in a step of size $h$, as the previous experiments appears to indicate that the maximum defects tend to occur at these locations within each step.

\subsection{RK6 with HB6}
\paragraph{Problem 1 results}
Figures $\ref{fig:rk6_with_hb6_p1_global_defect}$, $\ref{fig:rk6_with_hb6_p1_global_error}$ and $\ref{fig:rk6_with_hb6_p1_scaled_defects}$ shows the results of using RK6 with HB6 on Problem 1. We note that an absolute tolerance of $10^{-6}$ is applied on the maximum defect within the step and this can be shown to occur at $0.3h$ and $0.8h$ along a step of size, h. See Figure $\ref{fig:rk6_with_hb6_p1_scaled_defects}$, to see the scaled defect reaching a maximum near these points. We note that we are able to successfully control the defect of the continuous numerical solution using this approach, see Figure $\ref{fig:rk6_with_hb6_p1_global_defect}$. 

\begin{figure}[H]
\centering
\includegraphics[width=0.7\linewidth]{./figures/rk6_with_hb6_p1_global_defect}
\caption{Defect across the entire domain for RK6 with HB6 on problem 1 at an absolute tolerance of $10^{-6}$.}
\label{fig:rk6_with_hb6_p1_global_defect}
\end{figure}

\begin{figure}[H]
\centering
\includegraphics[width=0.7\linewidth]{./figures/rk6_with_hb6_p1_global_error}
\caption{Global Error for RK6 with HB6 on problem 1 at an absolute tolerance of $10^{-6}$.}
\label{fig:rk6_with_hb6_p1_global_error}
\end{figure}

\begin{figure}[H]
\centering
\includegraphics[width=0.7\linewidth]{./figures/rk6_with_hb6_p1_scaled_defects}
\caption{Scaled defects for RK6 with HB6 on problem 1 at an absolute tolerance of $10^{-6}$ mapped onto $[0, 1]$.}
\label{fig:rk6_with_hb6_p1_scaled_defects}
\end{figure}

\paragraph{Problem 2 results}
Figures $\ref{fig:rk6_with_hb6_p2_global_defect}$, $\ref{fig:rk6_with_hb6_p2_global_error}$ and $\ref{fig:rk6_with_hb6_p2_scaled_defects}$ shows the results of using RK6 with HB6 on Problem 2. We note that an absolute tolerance of $10^{-6}$ is applied on the maximum defect within the step and this can be shown to occur at $0.3h$ or $0.8h$ along a step of size, h. See Figure $\ref{fig:rk6_with_hb6_p2_scaled_defects}$, to see the scaled defect reaching a maximum near these points. We note that we are able to successfully control the defect of the continuous numerical solution using this approach, see Figure $\ref{fig:rk6_with_hb6_p2_global_defect}$. 

\begin{figure}[H]
\centering
\includegraphics[width=0.7\linewidth]{./figures/rk6_with_hb6_p2_global_defect}
\caption{Defect across the entire domain for RK6 with HB6 on problem 2 at an absolute tolerance of $10^{-6}$.}
\label{fig:rk6_with_hb6_p2_global_defect}
\end{figure}

\begin{figure}[H]
\centering
\includegraphics[width=0.7\linewidth]{./figures/rk6_with_hb6_p2_global_error}
\caption{Global Error for RK6 with HB6 on problem 2 at an absolute tolerance of $10^{-6}$.}
\label{fig:rk6_with_hb6_p2_global_error}
\end{figure}

\begin{figure}[H]
\centering
\includegraphics[width=0.7\linewidth]{./figures/rk6_with_hb6_p2_scaled_defects}
\caption{Scaled defects for RK6 with HB6 on problem 2 at an absolute tolerance of $10^{-6}$ mapped onto $[0, 1]$.}
\label{fig:rk6_with_hb6_p2_scaled_defects}
\end{figure}

\begin{figure}[H]
\centering
\includegraphics[width=0.7\linewidth]{./figures/rk6_with_hb6_p2_scaled_defects_small_steps}
\caption{Scaled defects for RK6 with HB6 on small steps on problem 2 at an absolute tolerance of $10^{-6}$ mapped onto $[0, 1]$. Despite the noise, for most steps, the maximum defect mostly appears near $0.8h$.}
\label{fig:rk6_with_hb6_p2_scaled_defects_small_steps}
\end{figure}

\paragraph{Problem 3 results}
Figures $\ref{fig:rk6_with_hb6_p3_global_defect}$, $\ref{fig:rk6_with_hb6_p3_global_error}$ and $\ref{fig:rk6_with_hb6_p3_scaled_defects}$ shows the results of using RK6 with HB6 on Problem 3. 
We note that an absolute tolerance of $10^{-6}$ is applied on the maximum defect within the step and this can be shown to occur at $0.3h$ or $0.8h$ along a step of size, h. See Figure $\ref{fig:rk6_with_hb6_p3_scaled_defects}$, to see the scaled defect reaching a maximum near these points. We note that we are able to successfully control the defect of the continuous numerical solution using this approach, see Figure $\ref{fig:rk6_with_hb6_p3_global_defect}$. 

\begin{figure}[H]
\centering
\includegraphics[width=0.7\linewidth]{./figures/rk6_with_hb6_p3_global_defect}
\caption{Defect across the entire domain for RK6 with HB6 on problem 3 at an absolute tolerance of $10^{-6}$.}
\label{fig:rk6_with_hb6_p3_global_defect}
\end{figure}

\begin{figure}[H]
\centering
\includegraphics[width=0.7\linewidth]{./figures/rk6_with_hb6_p3_global_error}
\caption{Global Error for RK6 with HB6 on problem 3 at an absolute tolerance of $10^{-6}$.}
\label{fig:rk6_with_hb6_p3_global_error}
\end{figure}

\begin{figure}[H]
\centering
\includegraphics[width=0.7\linewidth]{./figures/rk6_with_hb6_p3_scaled_defects}
\caption{Scaled defects for RK6 with HB6 on problem 3 at an absolute tolerance of $10^{-6}$ mapped onto $[0, 1]$.}
\label{fig:rk6_with_hb6_p3_scaled_defects}
\end{figure}

\begin{figure}[H]
\centering
\includegraphics[width=0.7\linewidth]{./figures/rk6_with_hb6_p3_scaled_defects_small_steps}
\caption{Scaled defects for RK6 with HB6 on small steps on problem 3 at an absolute tolerance of $10^{-6}$ mapped onto $[0, 1]$.}
\label{fig:rk6_with_hb6_p3_scaled_defects_small_steps}
\end{figure}

\subsection{RK6 with HB8}
\paragraph{Problem 1 results}
Figures $\ref{fig:rk6_with_hb8_p1_global_defect}$, $\ref{fig:rk6_with_hb8_p1_global_error}$ and $\ref{fig:rk6_with_hb8_p1_scaled_defects}$ shows the results of using RK6 with HB8 on Problem 1. We note that an absolute tolerance of $10^{-6}$ is applied on the maximum defect within the step and this can be shown to occur at $0.4h$ and $0.8h$ along a step of size, h. See Figure $\ref{fig:rk6_with_hb8_p1_scaled_defects}$, to see the scaled defect reaching a maximum near these points. We note that we are able to successfully control the defect of the continuous numerical solution using this approach, see Figure $\ref{fig:rk6_with_hb8_p1_global_defect}$. 


\begin{figure}[H]
\centering
\includegraphics[width=0.7\linewidth]{./figures/rk6_with_hb8_p1_global_defect}
\caption{Defect across the entire domain for RK6 with HB8 on problem 1 at an absolute tolerance of $10^{-6}$.}
\label{fig:rk6_with_hb8_p1_global_defect}
\end{figure}

\begin{figure}[H]
\centering
\includegraphics[width=0.7\linewidth]{./figures/rk6_with_hb8_p1_global_error}
\caption{Global Error for RK6 with HB8 on problem 1 at an absolute tolerance of $10^{-6}$.}
\label{fig:rk6_with_hb8_p1_global_error}
\end{figure}

\begin{figure}[H]
\centering
\includegraphics[width=0.7\linewidth]{./figures/rk6_with_hb8_p1_scaled_defects}
\caption{Scaled defects for RK6 with HB8 on problem 1 at an absolute tolerance of $10^{-6}$ mapped onto $[0, 1]$.}
\label{fig:rk6_with_hb8_p1_scaled_defects}
\end{figure}

\paragraph{Problem 2 results}
Figures $\ref{fig:rk6_with_hb8_p2_global_defect}$, $\ref{fig:rk6_with_hb8_p2_global_error}$ and $\ref{fig:rk6_with_hb8_p2_scaled_defects}$ shows the results of using RK6 with HB8 on Problem 2. We note that an absolute tolerance of $10^{-6}$ is applied on the maximum defect within the step and this can be shown to occur at $0.8h$ along a step of size, h. See Figure $\ref{fig:rk6_with_hb8_p2_scaled_defects}$, to see the scaled defect reaching a maximum near these points. We note that we are able to successfully control the defect of the continuous numerical solution using this approach, see Figure $\ref{fig:rk6_with_hb8_p2_global_defect}$. 


\begin{figure}[H]
\centering
\includegraphics[width=0.7\linewidth]{./figures/rk6_with_hb8_p2_global_defect}
\caption{Defect across the entire domain for RK6 with HB8 on problem 2 at an absolute tolerance of $10^{-6}$.}
\label{fig:rk6_with_hb8_p2_global_defect}
\end{figure}

\begin{figure}[H]
\centering
\includegraphics[width=0.7\linewidth]{./figures/rk6_with_hb8_p2_global_error}
\caption{Global Error for RK6 with HB8 on problem 2 at an absolute tolerance of $10^{-6}$.}
\label{fig:rk6_with_hb8_p2_global_error}
\end{figure}

\begin{figure}[H]
\centering
\includegraphics[width=0.7\linewidth]{./figures/rk6_with_hb8_p2_scaled_defects}
\caption{Scaled defects for RK6 with HB8 on problem 2 at an absolute tolerance of $10^{-6}$ mapped onto $[0, 1]$.}
\label{fig:rk6_with_hb8_p2_scaled_defects}
\end{figure}

\begin{figure}[H]
\centering
\includegraphics[width=0.7\linewidth]{./figures/rk6_with_hb8_p2_scaled_defects_small_steps}
\caption{Scaled defects for RK6 with HB8 on small steps on problem 2 at an absolute tolerance of $10^{-6}$ mapped onto $[0, 1]$. Despite the noise, the maximum defect mostly appears near $0.8h$.}
\label{fig:rk6_with_hb8_p2_scaled_defects_small_steps}
\end{figure}

\paragraph{Problem 3 results}
Figures $\ref{fig:rk6_with_hb8_p3_global_defect}$, $\ref{fig:rk6_with_hb8_p3_global_error}$ and $\ref{fig:rk6_with_hb8_p3_scaled_defects}$ shows the results of RK6 with HB8 on Problem 3. 
We note that an absolute tolerance of $10^{-6}$ is applied on the maximum defect within the step and this can be shown to occur at either $0.4h$ or $0.8h$ along a step of size, h. See Figure $\ref{fig:rk6_with_hb8_p3_scaled_defects}$, to see the scaled defect reaching a maximum near these points. We note that we are able to successfully control the defect of the continuous numerical solution using this approach, see Figure $\ref{fig:rk6_with_hb8_p3_global_defect}$. 
 

\begin{figure}[H]
\centering
\includegraphics[width=0.7\linewidth]{./figures/rk6_with_hb8_p3_global_defect}
\caption{Defect across the entire domain for RK6 with HB8 on problem 3 at an absolute tolerance of $10^{-6}$.}
\label{fig:rk6_with_hb8_p3_global_defect}
\end{figure}

\begin{figure}[H]
\centering
\includegraphics[width=0.7\linewidth]{./figures/rk6_with_hb8_p3_global_error}
\caption{Global Error for RK6 with HB8 on problem 3 at an absolute tolerance of $10^{-6}$.}
\label{fig:rk6_with_hb8_p3_global_error}
\end{figure}

\begin{figure}[H]
\centering
\includegraphics[width=0.7\linewidth]{./figures/rk6_with_hb8_p3_scaled_defects}
\caption{Scaled defects for RK6 with HB8 on problem 3 at an absolute tolerance of $10^{-6}$ mapped onto $[0, 1]$.}
\label{fig:rk6_with_hb8_p3_scaled_defects}
\end{figure}

\begin{figure}[H]
\centering
\includegraphics[width=0.7\linewidth]{./figures/rk6_with_hb8_p3_scaled_defects_small_steps}
\caption{Scaled defects for RK6 with HB8 on small steps on problem 3 at an absolute tolerance of $10^{-6}$ mapped onto $[0, 1]$. Despite the noise, the maximum defect mostly appears near $0.8h$.}
\label{fig:rk6_with_hb8_p3_scaled_defects_small_steps}
\end{figure}

\begin{table}[h]
\caption {Number of steps taken by RK6 when modified to do defect control with HB8 vs when modified with HB6.} \label{tab:rk6_with_hb6_vs_hb8_nsteps}
\begin{center}
\begin{tabular}{ c c c c c } 
Problem & succ. steps HB8 & succ. steps HB6 & nsteps HB8 & nsteps HB6 \\ 
1       & 18                 &        26          & 18         & 27\\ 
2       & 14                 &        18          & 17         & 23\\
3       & 24                 &        42          & 26         & 51\\
\end{tabular}
\end{center}
\end{table}

From Table $\ref{tab:rk6_with_hb6_vs_hb8_nsteps}$, we can see that again, using an interpolant whose interpolation error and especially the interpolation error of its derivative is of higher order than the ODE solution drastically reduces the number of steps. The solver becomes more efficient as a result. Since RK6 with HB6 and with HB8 is behaving similarly to RK4 with HB4 and with HB6, we expect that using a $10^{th}$ order method would not improve the situation more than HB8 has over HB6. Though fitting RK6 with HB10 will be as efficient as fitting it with HB8, the fact that the interpolation error is no longer the limiting factor means that HB10 will not improve the situation. For RK6, HB8 is the preferred interpolant.  

We also note that during the solving of the 3 problems, the value of $\alpha$ with HB6 rarely was bigger than 4 or smaller than $\frac{1}{4}$. The values of $\alpha$ and $\beta$ with HB8 also rarely were bigger than 4 or smaller than $\frac{1}{4}$. 

\subsection{RK8 with HB8}
In this section, we fit the RK8 method, described in Section $\ref{section:basic_runge_kutta}$, with HB8. Though we expect the interpolation error to reduce the efficiency of this scheme, we provide a proof of concept that an RK8 can have defect control with the scheme presented in this chapter. We will look into the challenges and possibilities of deriving an HB10 scheme in the Future Work section. 

\paragraph{Problem 1 results}
Figures $\ref{fig:rk8_with_hb8_p1_global_defect}$, $\ref{fig:rk8_with_hb8_p1_global_error}$ and $\ref{fig:rk8_with_hb8_p1_scaled_defects}$ shows the results of using RK8 with HB8 on Problem 1. We note that an absolute tolerance of $10^{-6}$ is applied on the maximum defect within the step and this can be shown to occur at $0.3h$ and $0.8h$ along a step of size, h. See Figure $\ref{fig:rk8_with_hb8_p1_scaled_defects}$, to see the scaled defect reaching a maximum near these points. We note that we are able to successfully control the defect of the continuous numerical solution using this approach, see Figure $\ref{fig:rk8_with_hb8_p1_global_defect}$. 
 

\begin{figure}[H]
\centering
\includegraphics[width=0.7\linewidth]{./figures/rk8_with_hb8_p1_global_defect}
\caption{Defect across the entire domain for RK8 with HB8 on problem 1 at an absolute tolerance of $10^{-6}$.}
\label{fig:rk8_with_hb8_p1_global_defect}
\end{figure}

\begin{figure}[H]
\centering
\includegraphics[width=0.7\linewidth]{./figures/rk8_with_hb8_p1_global_error}
\caption{Global Error for RK8 with HB8 on problem 1 at an absolute tolerance of $10^{-6}$.}
\label{fig:rk8_with_hb8_p1_global_error}
\end{figure}

\begin{figure}[H]
\centering
\includegraphics[width=0.7\linewidth]{./figures/rk8_with_hb8_p1_scaled_defects}
\caption{Scaled defects for RK8 with HB8 on problem 1 at an absolute tolerance of $10^{-6}$  mapped onto $[0, 1]$.}
\label{fig:rk8_with_hb8_p1_scaled_defects}
\end{figure}

\paragraph{Problem 2 results}
Figures $\ref{fig:rk8_with_hb8_p2_global_defect}$, $\ref{fig:rk8_with_hb8_p2_global_error}$ and $\ref{fig:rk8_with_hb8_p2_scaled_defects}$ shows the results of using RK8 with HB8 on Problem 2. We note that an absolute tolerance of $10^{-6}$ is applied on the maximum defect within the step and this can be shown to occur at $0.3h$ or $0.8h$ along a step of size, h. See Figure $\ref{fig:rk8_with_hb8_p2_scaled_defects}$, to see the scaled defect reaching a maximum near these points. We note that we are able to successfully control the defect of the continuous numerical solution using this approach, see Figure $\ref{fig:rk8_with_hb8_p2_global_defect}$. 

\begin{figure}[H]
\centering
\includegraphics[width=0.7\linewidth]{./figures/rk8_with_hb8_p2_global_defect}
\caption{Defect across the entire domain for RK8 with HB8 on problem 2 at an absolute tolerance of $10^{-6}$.}
\label{fig:rk8_with_hb8_p2_global_defect}
\end{figure}

\begin{figure}[H]
\centering
\includegraphics[width=0.7\linewidth]{./figures/rk8_with_hb8_p2_global_error}
\caption{Global Error for RK8 with HB8 on problem 2 at an absolute tolerance of $10^{-6}$.}
\label{fig:rk8_with_hb8_p2_global_error}
\end{figure}

\begin{figure}[H]
\centering
\includegraphics[width=0.7\linewidth]{./figures/rk8_with_hb8_p2_scaled_defects}
\caption{Scaled defects for RK8 with HB8 on problem 2 at an absolute tolerance of $10^{-6}$  mapped onto $[0, 1]$.}
\label{fig:rk8_with_hb8_p2_scaled_defects}
\end{figure}

\begin{figure}[H]
\centering
\includegraphics[width=0.7\linewidth]{./figures/rk8_with_hb8_p2_scaled_defects_small_steps}
\caption{Scaled defects of RK8 with HB8 on small steps on problem 2 at an absolute tolerance of $10^{-6}$ mapped onto $[0, 1]$. Despite the noise, the maximum defect mostly appears near $0.8h$.}
\label{fig:rk8_with_hb8_p2_scaled_defects_small_steps}
\end{figure}

\paragraph{Problem 3 results}
Figures $\ref{fig:rk8_with_hb8_p3_global_defect}$, $\ref{fig:rk8_with_hb8_p3_global_error}$ and $\ref{fig:rk8_with_hb8_p3_scaled_defects}$ shows the results of using the modification of RK8 with HB8 on Problem 3. 
We note that an absolute tolerance of $10^{-6}$ is applied on the maximum defect within the step and this can be shown to occur at $0.3h$ or $0.8h$ along a step of size, h. See Figure $\ref{fig:rk8_with_hb8_p3_scaled_defects}$, to see the scaled defect reaching a maximum near these points. We note that we are able to successfully control the defect of the continuous numerical solution using this approach, see Figure $\ref{fig:rk8_with_hb8_p3_global_defect}$. 


\begin{figure}[H]
\centering
\includegraphics[width=0.7\linewidth]{./figures/rk8_with_hb8_p3_global_defect}
\caption{Defect across the entire domain for RK8 with HB8 on problem 3 at an absolute tolerance of $10^{-6}$.}
\label{fig:rk8_with_hb8_p3_global_defect}
\end{figure}

\begin{figure}[H]
\centering
\includegraphics[width=0.7\linewidth]{./figures/rk8_with_hb8_p3_global_error}
\caption{Global Error for RK8 with HB8 on problem 3 at an absolute tolerance of $10^{-6}$.}
\label{fig:rk8_with_hb8_p3_global_error}
\end{figure}

\begin{figure}[H]
\centering
\includegraphics[width=0.7\linewidth]{./figures/rk8_with_hb8_p3_scaled_defects}
\caption{Scaled defects for RK8 with HB8 on problem 3 at an absolute tolerance of $10^{-6}$  mapped onto $[0, 1]$. Despite the noise, the maximum defect mostly appears near $0.8h$.}
\label{fig:rk8_with_hb8_p3_scaled_defects}
\end{figure}

\begin{figure}[H]
\centering
\includegraphics[width=0.7\linewidth]{./figures/rk8_with_hb8_p3_scaled_defects_small_steps}
\caption{Scaled Defects of RK8 with HB8 on small steps on problem 3 at an absolute tolerance of $10^{-6}$ mapped onto $[0, 1]$.}
\label{fig:rk8_with_hb8_p3_scaled_defects_small_steps}
\end{figure}

\begin{table}[h]
\caption {Number of steps taken by RK8 when modified to do defect control with HB8.} \label{tab:rk8_with_hb8_nsteps}
\begin{center}
\begin{tabular}{ c c c } 
Problem & succ. steps & nsteps \\ 
1       & 18             &        19 \\ 
2       & 12             &        15 \\
3       & 24             &        29 \\
\end{tabular}
\end{center}
\end{table}

\section{Using the previous interpolants to keep the weight parameters at 1}
\label{section:keeping_alpha_at_1}
One idea to solve the problem of the loss in accuracy due to deviation of the $\alpha$ and $\beta$ parameters from 1 is to construct the interpolant using a value of 1 for these parameters. We note that the best accuracy we can hope to get is with a value of 1 and thus employing data values situated at uniform distances apart guarantees the minimum error.

\subsection{rk4 with HB6} For the HB6 case, a simple way to guarantee that the value of $\alpha$ is 1 is by using the previous interpolants to get the required values at $x_{i - 1}=x_i - h$. Say we are at a value $x_i$ and we took a step of size $h$ to get to the value $x_{i + 1}$ where the function evaluation was $f_{i + 1}$ and the solution was $y_{i + 1}$. We could use the previous interpolants defined on the range $[0, x_i]$ to get a value at $x_i - h$ for the solution approximation, $y_{i - 1}$, and then we can use this $y_{i-1}$ value to compute $f(t_{i-1}, y_{i-1})$ to get the derivative $f_{i - 1}$.  We could thus create the new interpolant using these data points to guarantee that $\alpha$ stays at 1. This technique costs one extra function evaluation to obtain $f_{i-1}$.

This technique works (see Figures $\ref{fig:static_alpha_rk4_with_hb6_p1_global_defect}$ to $\ref{fig:static_alpha_rk4_with_hb6_p3_scaled_defects}$ to see the defect being controlled) but will require that the step-size is artificially limited on the first few steps so that we can interpolant back $x_i - h$ and still be in a range where our interpolants are correctly defined. For example, If we go from $t_0$ to $t_0 + h$ and the error estimate is much lower than the tolerance, we cannot use a step size of $2h$ as $t_0 + h - 2h$ = $t_0-h$ because we do not have an interpolant in the region $\leq t_0$. However this is not an issue as we can perform the first few steps with a CRK scheme for example.

\paragraph{Problem 1 results}
Figures $\ref{fig:static_alpha_rk4_with_hb6_p1_global_defect}$, $\ref{fig:static_alpha_rk4_with_hb6_p1_global_error}$ and $\ref{fig:static_alpha_rk4_with_hb6_p1_scaled_defects}$ shows the results of using the RK4 with HB6 and $\alpha = 1$ on Problem 1. We note that an absolute tolerance of $10^{-6}$ is applied on the maximum defect within the step and this can be shown to occur at $0.3h$ and $0.8h$ along a step of size, h. See Figure $\ref{fig:static_alpha_rk4_with_hb6_p1_scaled_defects}$, to see the scaled defect reaching a maximum near these points. We note that we are able to successfully control the defect of the continuous numerical solution using this approach, see Figure $\ref{fig:static_alpha_rk4_with_hb6_p1_global_defect}$. 


\begin{figure}[H]
\centering
\includegraphics[width=0.7\linewidth]{./figures/static_alpha_rk4_with_hb6_p1_global_defect}
\caption{Defect across the entire domain for RK4 with HB6 using $\alpha$ = 1 on problem 1 at an absolute tolerance of $10^{-6}$.}
\label{fig:static_alpha_rk4_with_hb6_p1_global_defect}
\end{figure}

\begin{figure}[H]
\centering
\includegraphics[width=0.7\linewidth]{./figures/static_alpha_rk4_with_hb6_p1_global_error}
\caption{Global Error for RK4 with HB6 using $\alpha$ = 1 on problem 1 at an absolute tolerance of $10^{-6}$.}
\label{fig:static_alpha_rk4_with_hb6_p1_global_error}
\end{figure}

\begin{figure}[H]
\centering
\includegraphics[width=0.7\linewidth]{./figures/static_alpha_rk4_with_hb6_p1_scaled_defects}
\caption{Scaled defects for RK4 with HB6 using $\alpha$ = 1 on problem 1 at an absolute tolerance of $10^{-6}$ mapped onto $[0, 1]$.}
\label{fig:static_alpha_rk4_with_hb6_p1_scaled_defects}
\end{figure}

\paragraph{Problem 2 results}
Figures $\ref{fig:static_alpha_rk4_with_hb6_p2_global_defect}$, $\ref{fig:static_alpha_rk4_with_hb6_p2_global_error}$ and $\ref{fig:static_alpha_rk4_with_hb6_p2_scaled_defects}$ shows the results of using RK4 with HB6 and $\alpha = 1$ on Problem 2. We note that an absolute tolerance of $10^{-6}$ is applied on the maximum defect within the step and this can be shown to occur at $0.8h$ along a step of size, h. See Figure $\ref{fig:static_alpha_rk4_with_hb6_p2_scaled_defects}$, to see the scaled defect reaching a maximum near these points. We note that we are able to successfully control the defect of the continuous numerical solution using this approach, see Figure $\ref{fig:static_alpha_rk4_with_hb6_p2_global_defect}$. 
\begin{figure}[H]
\centering
\includegraphics[width=0.7\linewidth]{./figures/static_alpha_rk4_with_hb6_p2_global_defect}
\caption{Defect across the entire domain for RK4 with HB6 using $\alpha$ = 1 on problem 2 at an absolute tolerance of $10^{-6}$.}
\label{fig:static_alpha_rk4_with_hb6_p2_global_defect}
\end{figure}

\begin{figure}[H]
\centering
\includegraphics[width=0.7\linewidth]{./figures/static_alpha_rk4_with_hb6_p2_global_error}
\caption{Global Error for RK4 with HB6 using $\alpha$ = 1 on problem 2 at an absolute tolerance of $10^{-6}$.}
\label{fig:static_alpha_rk4_with_hb6_p2_global_error}
\end{figure}

\begin{figure}[H]
\centering
\includegraphics[width=0.7\linewidth]{./figures/static_alpha_rk4_with_hb6_p2_scaled_defects}
\caption{Scaled defects for RK4 with HB6 using $\alpha$ = 1 on problem 2 at an absolute tolerance of $10^{-6}$ mapped onto $[0, 1]$.}
\label{fig:static_alpha_rk4_with_hb6_p2_scaled_defects}
\end{figure}

\begin{figure}[H]
\centering
\includegraphics[width=0.7\linewidth]{./figures/static_alpha_rk4_with_hb6_p2_scaled_defects_small_steps}
\caption{Scaled defects for RK4 with HB6 using $\alpha$ = 1 on small steps on problem 2 at an absolute tolerance of $10^{-6}$ mapped onto $[0, 1]$. Despite the noise, the maximum defect mostly appears near $0.8h$.}
\label{fig:static_alpha_rk4_with_hb6_p2_scaled_defects_small_steps}
\end{figure}

\paragraph{Problem 3 results}
Figures $\ref{fig:static_alpha_rk4_with_hb6_p3_global_defect}$, $\ref{fig:static_alpha_rk4_with_hb6_p3_global_error}$ and $\ref{fig:static_alpha_rk4_with_hb6_p3_scaled_defects}$ shows the results of using RK4 with HB6 and $\alpha = 1$ on Problem 3. We note that an absolute tolerance of $10^{-6}$ is applied on the maximum defect within the step and this can be shown to occur at $0.3h$ or $0.8h$along a step of size, h though this problem produces a more diverse location for the peaks. See Figure $\ref{fig:static_alpha_rk4_with_hb6_p3_scaled_defects}$, to see the scaled defect reaching a maximum near these points. We note that we are able to successfully control the defect of the continuous numerical solution using this approach, see Figure $\ref{fig:static_alpha_rk4_with_hb6_p3_global_defect}$. 

\begin{figure}[H]
\centering
\includegraphics[width=0.7\linewidth]{./figures/static_alpha_rk4_with_hb6_p3_global_defect}
\caption{Defect across the entire domain for RK4 with HB6 using $\alpha$ = 1 on problem 3 at an absolute tolerance of $10^{-6}$.}
\label{fig:static_alpha_rk4_with_hb6_p3_global_defect}
\end{figure}

\begin{figure}[H]
\centering
\includegraphics[width=0.7\linewidth]{./figures/static_alpha_rk4_with_hb6_p3_global_error}
\caption{Global Error for RK4 with HB6 using $\alpha$ = 1 on problem 3 at an absolute tolerance of $10^{-6}$.}
\label{fig:static_alpha_rk4_with_hb6_p3_global_error}
\end{figure}

\begin{figure}[H]
\centering
\includegraphics[width=0.7\linewidth]{./figures/static_alpha_rk4_with_hb6_p3_scaled_defects}
\caption{Scaled defects for RK4 with HB6 using $\alpha$ = 1 on problem 3 at an absolute tolerance of $10^{-6}$ mapped onto $[0, 1]$.}
\label{fig:static_alpha_rk4_with_hb6_p3_scaled_defects}
\end{figure}

\begin{table}[h]
\caption {Number of steps taken by RK4 when modified to do defect control with HB6 when we fix $\alpha$ at 1 vs when we allow it to fluctuate during the integration.} \label{tab:rk4_with_hb6_static_vs_variable_alpha}
\begin{center}
\begin{tabular}{ c c c c c } 
Problem & succ. steps & $\alpha$=1 succ. & nsteps & $\alpha$=1 nsteps \\ 
1       & 27                      &        29               & 27         & 33\\ 
2       & 36                      &        34               & 40         & 36\\
3       & 62                      &        65               & 73         & 82\\
\end{tabular}
\end{center}
\end{table}

Table $\ref{tab:rk4_with_hb6_static_vs_variable_alpha}$ shows how the solver with $\alpha$ fixed at 1 and the solver with $\alpha$ allowed to vary differ in the number of steps that they take. The results are very similar because, as we noted before, $\alpha$ tends to stay close to 1 and rarely deviates to a value smaller than $\frac{1}{4}$ or larger than 4.

\subsection{rk6 with HB8} For the HB8 case, a simple way to guarantee that the values of $\alpha$ and $\beta$ are 1 is by using the previous interpolants to get the required values at $x_{i - 1}=x_i-h$ and $x_{i - 2}=x_i-2h$. Suppose that we are at $x_i$ and we took a step of size $h$ to get to the value $x_{i + 1}$ where the function evaluation was $f_{i + 1}$ and the solution was $y_{i + 1}$. We could get the values of the solution and the derivative by using the previous interpolants defined on the range $[0, x_i]$ to get the value at exactly $x_i - h$ for the solution, $y_{i - 1}$, and then evaluate $f(t_{i-1}, y_{i-1})$ to get the derivative $f_{i - 1}$.  We can also use the previous interpolants on $[0, x_i]$ to get the values of the solution, $y_{i-2}$ and use it to evaluate $f(t_{i-2}, y_{i-2})$ to get the values of the derivative, $f_{i-2}$, at $x_{i-2} = x_i - 2h$. We could thus create create the new interpolant using the data values defined as such to guarantee that $\alpha$ and $\beta$ equal to 1. We note that we have built interpolants for both the solution and the derivative that  can interpolate up to $x_i$ for all cases except for the case $x_i = 0$ or for the first few steps. This scheme uses two more function evaluations.

This technique works (see Figures $\ref{fig:static_alpha_rk6_with_hb8_p1_global_defect}$ to $\ref{fig:static_alpha_rk6_with_hb8_p3_scaled_defects_small_steps}$ to see the defect being controlled) but will require that the step-size is artificially limited on the first few steps so that we can interpolant back $x_i - h$ and/or $x_i - 2h$ and still be in a range where our interpolants are correctly defined. This is not an issue as a CRK scheme could be used for the first few steps.

\paragraph{Problem 1 results}
Figures $\ref{fig:static_alpha_rk6_with_hb8_p1_global_defect}$, $\ref{fig:static_alpha_rk6_with_hb8_p1_global_error}$ and $\ref{fig:static_alpha_rk6_with_hb8_p1_scaled_defects}$ shows the results of using RK6 with HB8 and $\alpha = 1$ and $\beta = 1$ on Problem 1. We note that an absolute tolerance of $10^{-6}$ is applied on the maximum defect within the step and this can be shown to occur at $0.4h$ or $0.8h$ along a step of size, h. See Figure $\ref{fig:static_alpha_rk6_with_hb8_p1_scaled_defects}$, to see the scaled defect reaching a maximum near these points. We note that we are able to successfully control the defect of the continuous numerical solution using this approach, see Figure $\ref{fig:static_alpha_rk6_with_hb8_p1_global_defect}$. 
\begin{figure}[H]
\centering
\includegraphics[width=0.7\linewidth]{./figures/static_alpha_rk6_with_hb8_p1_global_defect}
\caption{Defect across the entire domain for RK6 with HB8 using $\alpha$ and $\beta$ = 1 on problem 1 at an absolute tolerance of $10^{-6}$.}
\label{fig:static_alpha_rk6_with_hb8_p1_global_defect}
\end{figure}

\begin{figure}[H]
\centering
\includegraphics[width=0.7\linewidth]{./figures/static_alpha_rk6_with_hb8_p1_global_error}
\caption{Global Error for RK6 with HB8 using $\alpha$ and $\beta$ = 1 on problem 1 at an absolute tolerance of $10^{-6}$.}
\label{fig:static_alpha_rk6_with_hb8_p1_global_error}
\end{figure}

\begin{figure}[H]
\centering
\includegraphics[width=0.7\linewidth]{./figures/static_alpha_rk6_with_hb8_p1_scaled_defects}
\caption{Scaled Defects for RK6 with HB8 using $\alpha$ and $\beta$ = 1 on problem 1 at an absolute tolerance of $10^{-6}$ mapped onto $[0, 1]$.}
\label{fig:static_alpha_rk6_with_hb8_p1_scaled_defects}
\end{figure}

\paragraph{Problem 2 results}
Figures $\ref{fig:static_alpha_rk6_with_hb8_p2_global_defect}$, $\ref{fig:static_alpha_rk6_with_hb8_p2_global_error}$ and $\ref{fig:static_alpha_rk6_with_hb8_p2_scaled_defects}$ shows the results of using RK6 with HB8 and $\alpha = 1$ and $\beta = 1$ on Problem 2. We note that an absolute tolerance of $10^{-6}$ is applied on the maximum defect within the step and this can be shown to occur at $0.8h$ along a step of size, h. See Figure $\ref{fig:static_alpha_rk6_with_hb8_p2_scaled_defects}$, to see the scaled defect reaching a maximum near these points. We note that we are able to successfully control the defect of the continuous numerical solution using this approach, see Figure $\ref{fig:static_alpha_rk6_with_hb8_p2_global_defect}$. 
\begin{figure}[H]
\centering
\includegraphics[width=0.7\linewidth]{./figures/static_alpha_rk6_with_hb8_p2_global_defect}
\caption{Defect across the entire domain for RK6 with HB8 using $\alpha$ and $\beta$ = 1 on problem 2 at an absolute tolerance of $10^{-6}$.}
\label{fig:static_alpha_rk6_with_hb8_p2_global_defect}
\end{figure}

\begin{figure}[H]
\centering
\includegraphics[width=0.7\linewidth]{./figures/static_alpha_rk6_with_hb8_p2_global_error}
\caption{Global Error for RK6 with HB8 using $\alpha$ and $\beta$ = 1 on problem 2 at an absolute tolerance of $10^{-6}$.}
\label{fig:static_alpha_rk6_with_hb8_p2_global_error}
\end{figure}

\begin{figure}[H]
\centering
\includegraphics[width=0.7\linewidth]{./figures/static_alpha_rk6_with_hb8_p2_scaled_defects}
\caption{Scaled Defects for RK6 with HB8 using $\alpha$ and $\beta$ = 1 on problem 2 at an absolute tolerance of $10^{-6}$ mapped onto $[0, 1]$.}
\label{fig:static_alpha_rk6_with_hb8_p2_scaled_defects}
\end{figure}

\begin{figure}[H]
\centering
\includegraphics[width=0.7\linewidth]{./figures/static_alpha_rk6_with_hb8_p2_scaled_defects_small_steps}
\caption{Scaled Defects for RK6 with HB8 using $\alpha$ and $\beta$ = 1 on small steps on problem 2 at an absolute tolerance of $10^{-6}$ mapped onto $[0, 1]$. Despite the noise, the maximum defect mostly appears near $0.8h$.}
\label{fig:static_alpha_rk6_with_hb8_p2_scaled_defects_small_steps}
\end{figure}

\paragraph{Problem 3 results}
Figures $\ref{fig:static_alpha_rk6_with_hb8_p3_global_defect}$, $\ref{fig:static_alpha_rk6_with_hb8_p3_global_error}$ and $\ref{fig:static_alpha_rk6_with_hb8_p3_scaled_defects}$ shows the results of using RK6 with HB8 and $\alpha = 1$ and $\beta = 1$ on Problem 3. We note that an absolute tolerance of $10^{-6}$ is applied on the maximum defect within the step and this can be shown to occur at $0.3h$ and $0.8h$ along a step of size, h. See Figure $\ref{fig:static_alpha_rk6_with_hb8_p3_scaled_defects}$, to see the scaled defect reaching a maximum near these points. We note that we are able to successfully control the defect of the continuous numerical solution using this approach, see Figure $\ref{fig:static_alpha_rk6_with_hb8_p3_global_defect}$. 

\begin{figure}[H]
\centering
\includegraphics[width=0.7\linewidth]{./figures/static_alpha_rk4_with_hb6_p3_global_defect}
\caption{Defect across the entire domain for RK6 with HB8 using $\alpha$ and $\beta$ = 1 on problem 3 at an absolute tolerance of $10^{-6}$.}
\label{fig:static_alpha_rk6_with_hb8_p3_global_defect}
\end{figure}

\begin{figure}[H]
\centering
\includegraphics[width=0.7\linewidth]{./figures/static_alpha_rk6_with_hb8_p3_global_error}
\caption{Global Error for RK6 with HB8 using $\alpha$ and $\beta$ = 1 on problem 3 at an absolute tolerance of $10^{-6}$.}
\label{fig:static_alpha_rk6_with_hb8_p3_global_error}
\end{figure}

\begin{figure}[H]
\centering
\includegraphics[width=0.7\linewidth]{./figures/static_alpha_rk6_with_hb8_p3_scaled_defects}
\caption{Scaled Defects for RK6 with HB8 using $\alpha$ and $\beta$ = 1 on problem 3 at an absolute tolerance of $10^{-6}$ mapped onto $[0, 1]$.}
\label{fig:static_alpha_rk6_with_hb8_p3_scaled_defects}
\end{figure}

\begin{figure}[H]
\centering
\includegraphics[width=0.7\linewidth]{./figures/static_alpha_rk6_with_hb8_p3_scaled_defects_small_steps}
\caption{Scaled Defects for RK6 with HB8 using $\alpha$ and $\beta$ = 1 on small steps on problem 3 at an absolute tolerance of $10^{-6}$ mapped onto $[0, 1]$. Despite the noise, the maximum defect mostly appears near $0.8h$.}
\label{fig:static_alpha_rk6_with_hb8_p3_scaled_defects_small_steps}
\end{figure}

\begin{table}[h]
\caption {Number of steps taken by RK6 when modified to do defect control with HB8 with $\alpha$ and $\beta$ forcibly at 1 and variable $\alpha$ and $\beta$.} \label{tab:rk6_with_hb8_static_vs_variable}
\begin{center}
\begin{tabular}{ c c c c c } 
Problem & succ. & params=1 succ. steps & nsteps & params=1 nsteps \\ 
1       & 18                      &        23               & 18         & 31\\ 
2       & 14                      &        18               & 17         & 21\\
3       & 24                      &        26               & 26         & 27\\
\end{tabular}
\end{center}
\end{table}	

Table $\ref{tab:rk6_with_hb8_static_vs_variable}$ shows how the solver with $\alpha$ and $\beta$ fixed at 1 and the solver with $\alpha$ and $\beta$ allowed to vary differs in the number of steps that they take. The results are somewhat similar because, as we noted before, $\alpha$ and $\beta$ tends to stay close to 1 and rarely deviates to a value smaller than $\frac{1}{4}$ or larger than 4.

=======================================================
\subsection{rk8 with HB10} For the HB10 case, a simple way to guarantee that the values of $\alpha$ and $\beta$ are 1 is by using the previous interpolants to get the required values at $x_{i - 1}=x_i-2h$ and $x_{i - 2}=x_i-h$. We need to use $2h$ for the $x_{i-1}$ as this step is broken into 2. This way $alpha$ and $beta$ are 1 when we break the step. Suppose that we are at $x_i$ and we took a step of size $h$ to get to the value $x_{i + 1}$ where the function evaluation was $f_{i + 1}$ and the solution was $y_{i + 1}$. We could get the values of the solution and the derivative by using the previous interpolants defined on the range $[0, x_i]$ to get the value at exactly $x_i - 2h$ for the solution, $y_{i - 1}$, and then evaluate $f(t_{i-1}, y_{i-1})$ to get the derivative $f_{i - 1}$.  We can also use the previous interpolants on $[0, x_i]$ to get the values of the solution, $y_{i-2}$ and use it to evaluate $f(t_{i-2}, y_{i-2})$ to get the values of the derivative, $f_{i-2}$, at $x_{i-2} = x_i - 3h$. We could thus create create the new interpolant using the data values defined as such to guarantee that $\alpha$ and $\beta$ equal to 1. We then use the embedded HB8 to get the solution value at $x_{i - \frac{1}{2}}$ and make a function evaluation to get the derivative value. We can then build HB10 on that data. We note that we have built interpolants for both the solution and the derivative that  can interpolate up to $x_i$ for all cases except for the case $x_i = 0$ or for the first few steps. This scheme uses two more function evaluations.

This technique works (see Figures $\ref{fig:static_alpha_rk8_with_hb10_p1_global_defect}$ to $\ref{fig:static_alpha_rk8_with_hb10_p3_scaled_defects}$ to see the defect being controlled) but will require that the step-size is artificially limited on the first few steps so that we can interpolant back $x_i - h$ and/or $x_i - 2h$ and still be in a range where our interpolants are correctly defined. This is not an issue as a CRK scheme could be used for the first few steps.

\paragraph{Problem 1 results}
Figures $\ref{fig:static_alpha_rk8_with_hb10_p1_global_defect}$, $\ref{fig:static_alpha_rk8_with_hb10_p1_global_error}$ and $\ref{fig:static_alpha_rk8_with_hb10_p1_scaled_defects}$ shows the results of using RK8 with HB10 and $\alpha = 1$ and $\beta = 1$ on Problem 1. We note that an absolute tolerance of $10^{-6}$ is applied on the maximum defect within the step and this can be shown to occur at $0.3h$ or $0.8h$ along a step of size, h. See Figure $\ref{fig:static_alpha_rk8_with_hb10_p1_scaled_defects}$, to see the scaled defect reaching a maximum near these points. We note that we are able to successfully control the defect of the continuous numerical solution using this approach, see Figure $\ref{fig:static_alpha_rk8_with_hb10_p1_global_defect}$. 
\begin{figure}[H]
\centering
\includegraphics[width=0.7\linewidth]{./figures/static_alpha_rk8_with_hb10_p1_global_defect}
\caption{Defect across the entire domain for RK8 with HB10 using $\alpha$ and $\beta$ = 1 on problem 1 at an absolute tolerance of $10^{-6}$.}
\label{fig:static_alpha_rk8_with_hb10_p1_global_defect}
\end{figure}

\begin{figure}[H]
\centering
\includegraphics[width=0.7\linewidth]{./figures/static_alpha_rk8_with_hb10_p1_global_error}
\caption{Global Error for RK8 with HB10 using $\alpha$ and $\beta$ = 1 on problem 1 at an absolute tolerance of $10^{-6}$.}
\label{fig:static_alpha_rk8_with_hb10_p1_global_error}
\end{figure}

\begin{figure}[H]
\centering
\includegraphics[width=0.7\linewidth]{./figures/static_alpha_rk8_with_hb10_p1_scaled_defects}
\caption{Scaled Defects for RK8 with HB10 using $\alpha$ and $\beta$ = 1 on problem 1 at an absolute tolerance of $10^{-6}$ mapped onto $[0, 1]$.}
\label{fig:static_alpha_rk8_with_hb10_p1_scaled_defects}
\end{figure}

\paragraph{Problem 2 results}
Figures $\ref{fig:static_alpha_rk8_with_hb10_p2_global_defect}$, $\ref{fig:static_alpha_rk8_with_hb10_p2_global_error}$ and $\ref{fig:static_alpha_rk8_with_hb10_p2_scaled_defects}$ shows the results of using RK8 with HB10 and $\alpha = 1$ and $\beta = 1$ on Problem 2. We note that an absolute tolerance of $10^{-6}$ is applied on the maximum defect within the step and this can be shown to occur at $0.8h$ along a step of size, h. See Figure $\ref{fig:static_alpha_rk8_with_hb10_p2_scaled_defects}$, to see the scaled defect reaching a maximum near these points. We note that we are able to successfully control the defect of the continuous numerical solution using this approach, see Figure $\ref{fig:static_alpha_rk8_with_hb10_p2_global_defect}$. 
\begin{figure}[H]
\centering
\includegraphics[width=0.7\linewidth]{./figures/static_alpha_rk8_with_hb10_p2_global_defect}
\caption{Defect across the entire domain for RK8 with HB10 using $\alpha$ and $\beta$ = 1 on problem 2 at an absolute tolerance of $10^{-6}$.}
\label{fig:static_alpha_rk8_with_hb10_p2_global_defect}
\end{figure}

\begin{figure}[H]
\centering
\includegraphics[width=0.7\linewidth]{./figures/static_alpha_rk8_with_hb10_p2_global_error}
\caption{Global Error for RK8 with HB10 using $\alpha$ and $\beta$ = 1 on problem 2 at an absolute tolerance of $10^{-6}$.}
\label{fig:static_alpha_rk8_with_hb10_p2_global_error}
\end{figure}

\begin{figure}[H]
\centering
\includegraphics[width=0.7\linewidth]{./figures/static_alpha_rk8_with_hb10_p2_scaled_defects}
\caption{Scaled Defects for RK8 with HB10 using $\alpha$ and $\beta$ = 1 on problem 2 at an absolute tolerance of $10^{-6}$ mapped onto $[0, 1]$.}
\label{fig:static_alpha_rk8_with_hb10_p2_scaled_defects}
\end{figure}

\paragraph{Problem 3 results}
Figures $\ref{fig:static_alpha_rk8_with_hb10_p3_global_defect}$, $\ref{fig:static_alpha_rk8_with_hb10_p3_global_error}$ and $\ref{fig:static_alpha_rk8_with_hb10_p3_scaled_defects}$ shows the results of using RK8 with HB10 and $\alpha = 1$ and $\beta = 1$ on Problem 3. We note that an absolute tolerance of $10^{-6}$ is applied on the maximum defect within the step and this can be shown to occur at $0.3h$ and $0.8h$ along a step of size, h. See Figure $\ref{fig:static_alpha_rk8_with_hb10_p3_scaled_defects}$, to see the scaled defect reaching a maximum near these points. We note that we are able to successfully control the defect of the continuous numerical solution using this approach, see Figure $\ref{fig:static_alpha_rk8_with_hb10_p3_global_defect}$. 

\begin{figure}[H]
\centering
\includegraphics[width=0.7\linewidth]{./figures/static_alpha_rk8_with_hb10_p3_global_defect}
\caption{Defect across the entire domain for RK8 with HB10 using $\alpha$ and $\beta$ = 1 on problem 3 at an absolute tolerance of $10^{-6}$.}
\label{fig:static_alpha_rk8_with_hb10_p3_global_defect}
\end{figure}

\begin{figure}[H]
\centering
\includegraphics[width=0.7\linewidth]{./figures/static_alpha_rk8_with_hb10_p3_global_error}
\caption{Global Error for RK8 with HB10 using $\alpha$ and $\beta$ = 1 on problem 3 at an absolute tolerance of $10^{-6}$.}
\label{fig:static_alpha_rk8_with_hb10_p3_global_error}
\end{figure}

\begin{figure}[H]
\centering
\includegraphics[width=0.7\linewidth]{./figures/static_alpha_rk8_with_hb10_p3_scaled_defects}
\caption{Scaled Defects for RK8 with HB10 using $\alpha$ and $\beta$ = 1 on problem 3 at an absolute tolerance of $10^{-6}$ mapped onto $[0, 1]$.}
\label{fig:static_alpha_rk8_with_hb10_p3_scaled_defects}
\end{figure}


\begin{table}[h]
\caption {Number of steps taken by RK8 when modified to do defect control with HB10 with $\alpha$ and $\beta$ forcibly at 1 and variable $\alpha$ and $\beta$.} \label{tab:rk8_with_hb10_static_vs_variable}
\begin{center}
\begin{tabular}{ c c c c c } 
Problem & succ. & params=1 succ. steps & nsteps & params=1 nsteps \\ 
1       & 14  &        22          & 16   & 35\\ 
2       & 10  &        11          & 13   & 27\\
3       & 20  &        23          & 28   & 32\\
\end{tabular}
\end{center}
\end{table}	

Table $\ref{tab:rk8_with_hb10_static_vs_variable}$ shows that the static parameter version is less efficient. We note that part of this is because of the first steps being wrong but also that the static parameter version is more accurate. This is a problem is the step selection algorithm. A better selection algorithm for such a high order method would drastically reduce the number of failed  steps.






\section{Final recommendations}
\label{section:defect_final_recommendations}
As we have noted before, all the interpolants have a V-shaped defect. We should note that experimentally, the trough for these V-shapes seem to be problem-independent. We thus used the optimal $h$ value, that is the $h$ value where the minimum maximum defect is found, as the initial $h$ value for each solver. We need this, especially in the variable parameter case, so that the first few steps are accepted. These optimal values are as follows. For HB4, the optimal $h$ value seems to be at about $10^{-3}$, for HB6, the optimal $h$ value seems to be at around $10^{-2}$ and for HB8, the optimal value seems to be at around $10^{-1}$ and $10^{-2}$. See Appendix $\ref{section:v_shaped_graph}$ for more details.

We also experimented with using representations of the polynomials other than the monomial form to reduce the effect of the rounding-off error. We can look to use the Barycentric or Horner form of the polynomials to improve the accuracy for example. (See Appendix $\ref{section:horner_bary_forms}$ for more details.)

Some recommendations for a final solver will be to start with the optimal h for the respective interpolant as the first step size so that the $\alpha$ and $\beta$ parameters do not get too large or too small for the first steps. In an ideal case, we would like to keep accepting steps at the start and allow the parameters to be close to 1 for as long as possible.

We should solve with a solver with variable $\alpha$ at the start and then if the solver fails too many step repeatedly, that is, the parameters get too far from 1, we should use the technique of forcing the parameters to be 1 and using the previous interpolants.

The first recommendation guarantees that the first few steps are taken at the minimum error possible and thus that they succeed. The second recommendation guarantees that if we meet a challenging behaviour at some point, we would be able to step through it with static parameters at the cost of additional function evaluations.

We note that because of the V-shape, there is a high likelihood that we cannot solve any problem at very sharp tolerances (as sharper as $10^{-12}$). However, as have shown in Appendix $\ref{section:solving_at_sharp_tolerances}$, the solvers that were created were able to solve for tolerances of $2.5 \times 10^{-12}$.

\include{defeet_control_sharp}
\section{assymptotically correct defect control}
\subsection{what is assymptotically correct}
\subsection{order conditions proof}
\subsection{derivation of assymptotically correct HB6}
\subsection{using rk4 with HB6 on test problems}
\section{error control instead of defect control}
Another idea is to consider error control instead of defect control for the continuous approximate solution. We would thus need a way to create two interpolants, one of a higher order and one of a lower order and sample the difference between these two interpolants to estimate the error of the continuous solution approximation. 
=== A step-size selection algorithm based on that error estimate could provide an effective error controlled solution.

An issue with defect control is the V-shape of the defect. 
We know that this is entirely because of the $\frac{1}{h}$ in the derivative definition of the Hermite-Birkhoff interpolants as the interpolant itself does not suffer from round-off error but its derivative does.

\subsection{error is not v-shaped}
For all the schemes, the defect is V-shaped but the error itself is not. 
This is because the Hermite-Birkhoff interpolant does not contain a term in $\frac{1}{h}$ whereas its derivative does contain such a term. 
Figure $\ref{fig:defect_is_v_shape}$ and $\ref{fig:error_is_not_v_shape}$ shows this phenomenon for HB6 but the same can be see for HB4 and HB8. 

\begin{figure}[H]
\centering
\includegraphics[width=0.7\linewidth]{./figures/further_work_defect_is_v_shape_hb6}
\caption{Defect has V-shape.}
\label{fig:defect_is_v_shape}
\end{figure}

\begin{figure}[H]
\centering
\includegraphics[width=0.7\linewidth]{./figures/further_work_error_is_not_v_shape_hb6}
\caption{Error does not have V-shape.}
\label{fig:error_is_not_v_shape}
\end{figure}

\subsection{sampling error}
we sample
\begin{equation}
| h(x) - l(x) |
\end{equation}
at two x values within $x_i$ and $x_{i+1}$ and take the max to get the error estimate within the step.

\subsubsection{rk4 with hb4 vs hb6}
Because we do not differentiate and that hb4 has order 4, we can use with rk4 with HB4 for error control. Thus HB4 and HB6 should not suffer from much interpolation error and we can use rk4 with the hb4 and hb6 using the error scheme

\subsubsection{rk4 with hb6 vs hb8}
rk4 can also be use with hb6  and hb8 and not suffer from much interpolation error.

\subsubsection{rk6 with hb6 vs hb8}
Because we do not differentiate and that hb6 has order 6, we can use with rk6 with HB6 for error control. Thus HB6 and HB8 should not suffer from much interpolation error and we can use rk6 with the hb4 and hb6 using the error scheme

\subsubsection{continuous L2 norm}
The error estimate is calculated with a continuous L2 norm as follows.
\begin{equation}
error_i = \sqrt{ \int_{x_i}^{x_{i+1}} (\frac{h(x) - l(x)}{1 + |l(x)|})^2 \,dx }
\end{equation}
where h(x) is the higher order interpolant and l(x) is the lower order interpolant. This formula gives the error for one step and we compare it with the user provided atol for error control. 

========================
We ignored rtol as we had to factorise tol from the denominator so that we can use tol and $tol/10$ for comparisons. Essentially we assume $atol == rtol$
===================

\subsubsection{rk4 with hb4 vs hb6}
Because we do not differentiate and that hb4 has order 4, we can use with rk4 with HB4 for error control. Thus HB4 and HB6 should not suffer from much interpolation error and we can use rk4 with the hb4 and hb6 using the error scheme

\subsubsection{rk4 with hb6 vs hb8}
rk4 can also be use with hb6  and hb8 and not suffer from much interpolation error.

\subsubsection{rk6 with hb6 vs hb8}
Because we do not differentiate and that hb6 has order 6, we can use with rk6 with HB6 for error control. Thus HB6 and HB8 should not suffer from much interpolation error and we can use rk6 with the hb4 and hb6 using the error scheme

\subsection{conclusion}
\section{Summary and Future works}
\subsection{Summary}
In this chapter, we discussed the importance of defect control and the challenges that the standard approach faces in implementing it. We then derived $4^{th}$, $6^{th}$ and $8^{th}$ order Hermite-Birkhoff interpolants (HB4, HB6, HB8) that were used to augment the Classical $4^{th}$ order Runge-Kutta method (RK4). We showed that HB4 is not an appropriate way to provide an interpolant for defect control of RK4 as the interpolation error of the derivative is of a lower order than the numerical solution computed by RK4. We then showed that HB6 provides reliable and efficient defect control and that HB8 does not provide much of an improvement over HB6. We next showed how the HB6 and HB8 interpolants can be used to augment $6^{th}$ and $8^{th}$ order Runge-Kutta methods to allow them to provide efficient defect control. We also noted throughout the chapter that the multistep interpolants HB6 and HB8 have accuracies that rely on their step-size parameters, $\alpha$ and $\beta$, being close to 1. We then discussed an interpolant that forces these parameters to be 1 by using the evaluations of previous interpolants.

paragraph on asympt and error control.

\subsection{Future Works}
\label{section:HB_future_work}

\paragraph{The first few steps}
Throughout this chapter we have used the exact solution values for the first few steps in order to allow us to create the first interpolant. Another important research project in this area is to try different techniques including but not limited to the use of CRK methods, error control with a sharper tolerance than the user provided tolerance, and possibly other methods to perform the first few steps.

\paragraph{Asymptotically correct defect control with multistep interpolants}
paragraph on HB6 that we developed and potential for HB8.

We can also look into developing interpolants that could lead to asymptotically  correct defect control. 
This would guarantee that the maximum defect is always at the same relative location within each step and would thus only require one function evaluation to sample the defect.

\paragraph{HB10}
An idea for future work is to derive a $10^{th}$ order interpolant. Such an interpolant will be forced to use 3 step-size parameters but an idea is to fix one or more of the parameters at 1. This can be done by using the technique that we employed in Section $\ref{section:keeping_alpha_at_1}$ or by using another technique such as computing a solution value in the middle of the step $[x_{i-1}, x_i]$ using the interpolant from that step and performing an additional function evaluation at that data point to obtain the two values required to build an interpolant. Thus we get to use just two parameters $\alpha$ and $\beta$ for the step from $x_i$ to $x_{i+1}$ and the step from $x_{i-2}$ to $x_{i-1}$.  This will give the required 10 data points to produce such an interpolant which could then be used to augment RK8 to provide a more efficient defect control scheme for the $8^{th}$ order case. 

Early explorations into creating an HB10 interpolants seem to be promising. See Figure $\ref{fig:future_work_hb10_v_shape}$ to see how an HB10 derived by `breaking the middle step' is resilient to changes to its parameters $\alpha$ and $\beta$. We note that the interpolant was built with step-size $[\alpha h, \frac{h}{2}, \frac{h}{2}, \beta h]$ and thus $\alpha$ and $\beta$ is usually 2 when there are no step-size changes. We also note that $\theta$ was allowed to vary between $-2-\alpha$ to $\beta$.

\begin{figure}[H]
\centering
\includegraphics[width=0.7\linewidth]{./figures/future_work_hb10_v_shape}
\caption{V-shape of HB10 created by `breaking the middle step'.}
\label{fig:future_work_hb10_v_shape}
\end{figure}

\paragraph{Problems with error sampling and using a continuous L2 norm.}
================================================================
This technique is better for the error control case because we know we need a 3 point Gauss rule. i.e, if we applied it for the defect control case, we would be making 3 additional function evaluations...
Either way we can add a 'switch' in a final solver where error/defect sampling controls the maximum error/defect but 
================================================================
The shape of the error across each step is very inconsistent (See the Scaled errors plots in the previous section).One idea would be to find the max error by sampling at even more points. However, a lot of samplings would be required to reach a consistent error estimate scheme. In the next section, we present a way to sample the error using a continuous L2 norm. This essentially measures the distance between the two interpolants. By designing a continuous L2-norm scheme that maintain ratio of the error estimate from an L2 norm between two interpolants and the actual error between the interpolant and the actual solution, we guarantee that controlling the L2 norm controls the global error.

The error estimate is calculated with a continuous L2 norm as follows.
\begin{equation}
estimated\_error_i = \sqrt{ \int_{x_i}^{x_{i+1}} (\frac{h(x) - l(x)}{1 + |l(x)|})^2 \,dx }
\end{equation}
where h(x) is the higher order interpolant and l(x) is the lower order interpolant. This formula gives the error for one step and we compare it with the user provided atol for error control. 

========================
We ignored rtol as we had to factorise tol from the denominator so that we can use tol and $tol/10$ for comparisons. Essentially we assume $atol == rtol$
===================

We note that the solver performs error control through the continuous L2 norm but we want to control the actual error between the exact solution and the interpolant that was returned. To that effect, we define the following ratio: $\frac{estimated\_error_i}{exact\_error_i}$. Ideally, we want this ratio to be 1 for the scheme that we designed as a ratio of 1 entails that controlling the L2-norm would control the error between the interpolant and the exact solution.

We note that we consider the exact error is calculated with
\begin{equation}
exact\_error_i = \sqrt{ \int_{x_i}^{x_{i+1}} (\frac{sol(x) - l(x)}{1 + |l(x)|})^2 \,dx }
\end{equation}
where $sol(x)$ is the solution at point, x so that we can plot $\frac{estimated\_error_i}{exact\_error_i}$. 

To perform the integration, we use a Gauss rule. Through experimentation, we have seen that a 3-point Gauss rule was enough to produce efficient solution. A 2-point rule increased was being affected by the integration error whereas 4- and 5-points rules were not improving the number of steps taken by much showing that the integration error was no longer a limiting factor as from the 3-point rule. We note that we want to use as the smallest Gauss rule we can get away with to improve efficiency.

\subsubsection{rk4 with hb4 vs hb6 with continuous L2 norm}
Because we do not differentiate and that hb4 has order 4, we can use with rk4 with HB4 for error control. Thus HB4 and HB6 should not suffer from much interpolation error and we can use rk4 with the hb4 and hb6 using the error scheme


\paragraph{Problem 1} Figures $\ref{fig:rk4_with_hb4_hb6_L2norm_p1_global_error}$ and $\ref{fig:rk4_with_hb4_hb6_L2norm_p1_error_ratio}$ shows the global error and the ratio $\frac{estimated\_error_i}{exact\_error_i}$ on each step of using rk4 with hb4 vs hb6 with continuous L2 norm on Problem 1 at an absolute tolerance of $10^{-6}$. From Figure $\ref{fig:rk4_with_hb4_hb6_L2norm_p1_error_ratio}$, we can see that the ratio is relatively close to 1 and we can see in Figure  $\ref{fig:rk4_with_hb4_hb6_L2norm_p1_global_error}$ that the exact error is within the tolerance even when the continuous L2 norm between the two interpolant was used to control the error.

\begin{figure}[H]
\centering
\includegraphics[width=0.7\linewidth]{./figures/rk4_with_hb4_hb6_L2norm_p1_global_error}
\caption{Global Error for RK4 with HB4 vs HB6 with continuous L2 norm on problem 1 at an absolute tolerance of $10^{-6}$.}
\label{fig:rk4_with_hb4_hb6_L2norm_p1_global_error}
\end{figure}

\begin{figure}[H]
\centering
\includegraphics[width=0.7\linewidth]{./figures/rk4_with_hb4_hb6_L2norm_p1_error_ratio}
\caption{Ratio $\frac{estimated\_error_i}{exact\_error_i}$ at the end of each successful step for RK4 with HB4 vs HB6 with continuous L2 norm on problem 1 at an absolute tolerance of $10^{-6}$.}
\label{fig:rk4_with_hb4_hb6_L2norm_p1_error_ratio}
\end{figure}

\paragraph{Problem 2} Figures $\ref{fig:rk4_with_hb4_hb6_L2norm_p2_global_error}$ and $\ref{fig:rk4_with_hb4_hb6_L2norm_p2_error_ratio}$ shows the global error and the ratio $\frac{estimated\_error_i}{exact\_error_i}$ on each step of using rk4 with hb4 vs hb6 with continuous L2 norm on Problem 2 at an absolute tolerance of $10^{-6}$. From Figure $\ref{fig:rk4_with_hb4_hb6_L2norm_p2_error_ratio}$, we can see that the ratio is not relatively close to 1 and we can see in Figure $\ref{fig:rk4_with_hb4_hb6_L2norm_p2_global_error}$ that the exact error is not within the tolerance.

\begin{figure}[H]
\centering
\includegraphics[width=0.7\linewidth]{./figures/rk4_with_hb4_hb6_L2norm_p2_global_error}
\caption{Global Error for RK4 with HB4 vs HB6 with continuous L2 norm on problem 2 at an absolute tolerance of $10^{-6}$.}
\label{fig:rk4_with_hb4_hb6_L2norm_p2_global_error}
\end{figure}

\begin{figure}[H]
\centering
\includegraphics[width=0.7\linewidth]{./figures/rk4_with_hb4_hb6_L2norm_p2_error_ratio}
\caption{Ratio $\frac{estimated\_error_i}{exact\_error_i}$ at the end of each successful step for RK4 with HB4 vs HB6 with continuous L2 norm on problem 2 at an absolute tolerance of $10^{-6}$.}
\label{fig:rk4_with_hb4_hb6_L2norm_p2_error_ratio}
\end{figure}

\paragraph{Problem 3} Figures $\ref{fig:rk4_with_hb4_hb6_L2norm_p3_global_error}$ and $\ref{fig:rk4_with_hb4_hb6_L2norm_p3_error_ratio}$ shows the global error and the ratio $\frac{estimated\_error_i}{exact\_error_i}$ on each step of using rk4 with hb4 vs hb6 with continuous L2 norm on Problem 3 at an absolute tolerance of $10^{-6}$. From Figure $\ref{fig:rk4_with_hb4_hb6_L2norm_p3_error_ratio}$, we can see that the ratio is relatively close to 1 and we can see in Figure  $\ref{fig:rk4_with_hb4_hb6_L2norm_p3_global_error}$ that the exact error is within the tolerance even when the continuous L2 norm between the two interpolant was used to control the error.

\begin{figure}[H]
\centering
\includegraphics[width=0.7\linewidth]{./figures/rk4_with_hb4_hb6_L2norm_p3_global_error}
\caption{Global Error for RK4 with HB4 vs HB6 with continuous L2 norm on problem 3 at an absolute tolerance of $10^{-6}$.}
\label{fig:rk4_with_hb4_hb6_L2norm_p3_global_error}
\end{figure}

\begin{figure}[H]
\centering
\includegraphics[width=0.7\linewidth]{./figures/rk4_with_hb4_hb6_L2norm_p3_error_ratio}
\caption{Ratio $\frac{estimated\_error_i}{exact\_error_i}$ at the end of each successful step for RK4 with HB4 vs HB6 with continuous L2 norm on problem 3 at an absolute tolerance of $10^{-6}$.}
\label{fig:rk4_with_hb4_hb6_L2norm_p3_error_ratio}
\end{figure}

\begin{table}[h]
\caption {Number of steps taken by RK4 using error control with HB4 vs HB6 by using a continuous L2 norm.} \label{tab:rk4_with_hb4_hb6_L2norm_nsteps}
\begin{center}
\begin{tabular}{ c c c } 
Problem & successful steps & total steps \\ 
1       & 27                         & 29 \\ 
2       & 20                         & 21 \\
3       & 61                         & 68 \\
\end{tabular}
\end{center}
\end{table}

\subsubsection{rk4 with hb6 vs hb8 with continuous L2 norm}
rk4 can also be use with hb6  and hb8 and not suffer from much interpolation error.



\paragraph{Problem 1} Figures $\ref{fig:rk4_with_hb6_hb8_L2norm_p1_global_error}$ and $\ref{fig:rk4_with_hb6_hb8_L2norm_p1_error_ratio}$ shows the global error and the ratio $\frac{estimated\_error_i}{exact\_error_i}$ on each step of using rk4 with hb6 vs hb8 with continuous L2 norm on Problem 1 at an absolute tolerance of $10^{-6}$. From Figure $\ref{fig:rk4_with_hb6_hb8_L2norm_p1_error_ratio}$, we can see that the ratio is relatively close to 1 and we can see in Figure  $\ref{fig:rk4_with_hb6_hb8_L2norm_p1_global_error}$ that the exact error is within the tolerance even when the continuous L2 norm between the two interpolant was used to control the error.

\begin{figure}[H]
\centering
\includegraphics[width=0.7\linewidth]{./figures/rk4_with_hb6_hb8_L2norm_p1_global_error}
\caption{Global Error for RK4 with HB6 vs HB8 with continuous L2 norm on problem 1 at an absolute tolerance of $10^{-6}$.}
\label{fig:rk4_with_hb6_hb8_L2norm_p1_global_error}
\end{figure}

\begin{figure}[H]
\centering
\includegraphics[width=0.7\linewidth]{./figures/rk4_with_hb6_hb8_L2norm_p1_error_ratio}
\caption{Ratio $\frac{estimated\_error_i}{exact\_error_i}$ at the end of each successful step for RK4 with HB6 vs HB8 with continuous L2 norm on problem 1 at an absolute tolerance of $10^{-6}$.}
\label{fig:rk4_with_hb6_hb8_L2norm_p1_error_ratio}
\end{figure}

\paragraph{Problem 2} Figures $\ref{fig:rk4_with_hb6_hb8_L2norm_p2_global_error}$ and $\ref{fig:rk4_with_hb6_hb8_L2norm_p2_error_ratio}$ shows the global error and the ratio $\frac{estimated\_error_i}{exact\_error_i}$ on each step of using rk4 with hb6 vs hb8 with continuous L2 norm on Problem 2 at an absolute tolerance of $10^{-6}$. From Figure $\ref{fig:rk4_with_hb6_hb8_L2norm_p2_error_ratio}$, we can see that the ratio is not relatively close to 1 and we can see in Figure $\ref{fig:rk4_with_hb6_hb8_L2norm_p2_global_error}$ that the exact error is not within the tolerance.

\begin{figure}[H]
\centering
\includegraphics[width=0.7\linewidth]{./figures/rk4_with_hb6_hb8_L2norm_p2_global_error}
\caption{Global Error for RK4 with HB6 vs HB8 with continuous L2 norm on problem 2 at an absolute tolerance of $10^{-6}$.}
\label{fig:rk4_with_hb6_hb8_L2norm_p2_global_error}
\end{figure}

\begin{figure}[H]
\centering
\includegraphics[width=0.7\linewidth]{./figures/rk4_with_hb6_hb8_L2norm_p2_error_ratio}
\caption{Ratio $\frac{estimated\_error_i}{exact\_error_i}$ at the end of each successful step for RK4 with HB6 vs HB8 with continuous L2 norm on problem 2 at an absolute tolerance of $10^{-6}$.}
\label{fig:rk4_with_hb6_hb8_L2norm_p2_error_ratio}
\end{figure}

\paragraph{Problem 3} Figures $\ref{fig:rk4_with_hb6_hb8_L2norm_p3_global_error}$ and $\ref{fig:rk4_with_hb6_hb8_L2norm_p3_error_ratio}$ shows the global error and the ratio $\frac{estimated\_error_i}{exact\_error_i}$ on each step of using rk4 with hb6 vs hb8 with continuous L2 norm on Problem 3 at an absolute tolerance of $10^{-6}$. From Figure $\ref{fig:rk4_with_hb6_hb8_L2norm_p3_error_ratio}$, we can see that the ratio is not relatively close to 1 and we can see in Figure $\ref{fig:rk4_with_hb6_hb8_L2norm_p3_global_error}$ that the exact error is not within the tolerance.


\begin{figure}[H]
\centering
\includegraphics[width=0.7\linewidth]{./figures/rk4_with_hb6_hb8_L2norm_p3_global_error}
\caption{Global Error for RK4 with HB6 vs HB8 with continuous L2 norm on problem 3 at an absolute tolerance of $10^{-6}$.}
\label{fig:rk4_with_hb6_hb8_L2norm_p3_global_error}
\end{figure}

\begin{figure}[H]
\centering
\includegraphics[width=0.7\linewidth]{./figures/rk4_with_hb6_hb8_L2norm_p3_error_ratio}
\caption{Ratio $\frac{estimated\_error_i}{exact\_error_i}$ at the end of each successful step for RK4 with HB6 vs HB8 with continuous L2 norm on problem 3 at an absolute tolerance of $10^{-6}$.}
\label{fig:rk4_with_hb6_hb8_L2norm_p3_error_ratio}
\end{figure}

\begin{table}[h]
\caption {Number of steps taken by RK4 using error control with HB6 vs HB8 by using a continuous L2 norm.} \label{tab:rk4_with_hb6_hb8_L2norm_nsteps}
\begin{center}
\begin{tabular}{ c c c } 
Problem & successful steps & total steps \\ 
1       & 15                         & 16 \\ 
2       & 15                         & 15 \\
3       & 26                         & 29 \\
\end{tabular}
\end{center}
\end{table}


\subsubsection{rk6 with hb6 vs hb8 with continuous L2 norm}
Because we do not differentiate and that hb6 has order 6, we can use with rk6 with HB6 for error control. Thus HB6 and HB8 should not suffer from much interpolation error and we can use rk6 with the hb4 and hb6 using the error scheme.

\paragraph{Problem 1} Figures $\ref{fig:rk6_with_hb6_hb8_L2norm_p1_global_error}$ and $\ref{fig:rk6_with_hb6_hb8_L2norm_p1_error_ratio}$ shows the global error and the ratio $\frac{estimated\_error_i}{exact\_error_i}$ on each step of using rk6 with hb6 vs hb8 with continuous L2 norm on Problem 1 at an absolute tolerance of $10^{-6}$. From Figure $\ref{fig:rk6_with_hb6_hb8_L2norm_p1_error_ratio}$, we can see that the ratio is relatively close to 1 and we can see in Figure $\ref{fig:rk6_with_hb6_hb8_L2norm_p1_global_error}$ that the exact error is within the tolerance even when the continuous L2 norm between the two interpolant was used to control the error.

\begin{figure}[H]
\centering
\includegraphics[width=0.7\linewidth]{./figures/rk6_with_hb6_hb8_L2norm_p1_global_error}
\caption{Global Error for RK6 with HB6 vs HB8 with continuous L2 norm on problem 1 at an absolute tolerance of $10^{-6}$.}
\label{fig:rk6_with_hb6_hb8_L2norm_p1_global_error}
\end{figure}

\begin{figure}[H]
\centering
\includegraphics[width=0.7\linewidth]{./figures/rk6_with_hb6_hb8_L2norm_p1_error_ratio}
\caption{Ratio $\frac{estimated\_error_i}{exact\_error_i}$ at the end of each successful step for RK6 with HB6 vs HB8 with continuous L2 norm on problem 1 at an absolute tolerance of $10^{-6}$.}
\label{fig:rk6_with_hb6_hb8_L2norm_p1_error_ratio}
\end{figure}

\paragraph{Problem 2} Figures $\ref{fig:rk6_with_hb6_hb8_L2norm_p2_global_error}$ and $\ref{fig:rk6_with_hb6_hb8_L2norm_p2_error_ratio}$ shows the global error and the ratio $\frac{estimated\_error_i}{exact\_error_i}$ on each step of using rk6 with hb6 vs hb8 with continuous L2 norm on Problem 2 at an absolute tolerance of $10^{-6}$. From Figure $\ref{fig:rk6_with_hb6_hb8_L2norm_p2_error_ratio}$, we can see that the ratio is relatively close to 1 and we can see in Figure $\ref{fig:rk6_with_hb6_hb8_L2norm_p2_global_error}$ that the exact error is within the tolerance even when the continuous L2 norm between the two interpolant was used to control the error.

\begin{figure}[H]
\centering
\includegraphics[width=0.7\linewidth]{./figures/rk6_with_hb6_hb8_L2norm_p2_global_error}
\caption{Global Error for RK6 with HB6 vs HB8 with continuous L2 norm on problem 2 at an absolute tolerance of $10^{-6}$.}
\label{fig:rk6_with_hb6_hb8_L2norm_p2_global_error}
\end{figure}

\begin{figure}[H]
\centering
\includegraphics[width=0.7\linewidth]{./figures/rk6_with_hb6_hb8_L2norm_p2_error_ratio}
\caption{Ratio $\frac{estimated\_error_i}{exact\_error_i}$ at the end of each successful step for RK6 with HB6 vs HB8 with continuous L2 norm on problem 2 at an absolute tolerance of $10^{-6}$.}
\label{fig:rk6_with_hb6_hb8_L2norm_p2_error_ratio}
\end{figure}

\paragraph{Problem 3} Figures $\ref{fig:rk6_with_hb6_hb8_L2norm_p3_global_error}$ and $\ref{fig:rk6_with_hb6_hb8_L2norm_p3_error_ratio}$ shows the global error and the ratio $\frac{estimated\_error_i}{exact\_error_i}$ on each step of using rk6 with hb6 vs hb8 with continuous L2 norm on Problem 3 at an absolute tolerance of $10^{-6}$. From Figure $\ref{fig:rk6_with_hb6_hb8_L2norm_p3_error_ratio}$, we can see that the ratio is relatively close to 1 and we can see in Figure $\ref{fig:rk6_with_hb6_hb8_L2norm_p3_global_error}$ that the exact error is within the tolerance even when the continuous L2 norm between the two interpolant was used to control the error.


\begin{figure}[H]
\centering
\includegraphics[width=0.7\linewidth]{./figures/rk6_with_hb6_hb8_L2norm_p3_global_error}
\caption{Global Error for RK6 with HB6 vs HB8 with continuous L2 norm on problem 3 at an absolute tolerance of $10^{-6}$.}
\label{fig:rk6_with_hb6_hb8_L2norm_p3_global_error}
\end{figure}

\begin{figure}[H]
\centering
\includegraphics[width=0.7\linewidth]{./figures/rk6_with_hb6_hb8_L2norm_p3_error_ratio}
\caption{Ratio $\frac{estimated\_error_i}{exact\_error_i}$ at the end of each successful step for RK6 with HB6 vs HB8 with continuous L2 norm on problem 3 at an absolute tolerance of $10^{-6}$.}
\label{fig:rk6_with_hb6_hb8_L2norm_p3_error_ratio}
\end{figure}

\begin{table}[h]
\caption {Number of steps taken by RK6 using error control with HB6 vs HB8 by using a continuous L2 norm.} \label{tab:rk6_with_hb6_hb8_L2norm_nsteps}
\begin{center}
\begin{tabular}{ c c c } 
Problem & successful steps & total steps \\ 
1       & 13                         & 15 \\ 
2       & 10                         & 10 \\
3       & 26                         & 36 \\
\end{tabular}
\end{center}
\end{table}


\bibliographystyle{plain}
\bibliography{ref}




\end{document}
