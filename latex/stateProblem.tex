\section{State dependent discontinuity problem}
In this section, we model the state-dependent discontinuity problem. We start by noting that this problem cannot be solved with the form of discontinuity handling used in the previous problem as we do not know when the discontinuity arises. Also, this problem will be harder than the time-dependent discontinuity problem as the parameter $\beta$ will be changed more than once as we attempt a long-term forecast to model the waves of the pandemic. 

As in Section $\ref{section:time_problem}$, changes in the modelling parameter $\beta$ introduce discontinuities in the function $f(t, y)$ and thus some solvers will thrash when trying to solve the problem (as described in $\ref{subsection:effect_of_discontinuity}$). But the parameter is changed more than once, the problem has more discontinuities than the previous problem. We will show that the presence of several discontinuities while not being able to do time-dependent cold starts makes the problem hard enough that all the ODE solvers we considered, even at very sharp tolerances, will not be able to solve the problem with reasonable accuracy.

The problem uses the state variable, E, which is the number of Exposed people, to determine when to change the parameter $\beta$. When the number of exposed people is greater than 25000, measures will be introduced and thus $\beta$ will change from 0.9 to 0.005. When the number of exposed people drops back to 10000, the measures will be relaxed and $\beta$ is set to 0.9. We run this model over a longer time period toggling the parameter $\beta$ back and forth to model the waves of the pandemic. This scenario corresponds to the case of an unvaccinated population where the only means of controlling the spread of the virus is through measures such as social isolation and so on. The ability of the virus to infect people is not diminished as time progresses, and when measures to stop the spread of the virus are removed, the infection rate of the virus returns to its original value.

We start with a naive treatment of the problem with if-statements inside the function that defines the right-hand side of the ODE system. We proceed to show how the problem cannot be solved this way even at sharp tolerances and finally, we will introduce a way to efficiently and accurately solve the problem using event detection.

\subsection{Naive treatment of Covid-19 state dependent discontinuity model}
\label{subsection:naive_state_problem}
The naive treatment of this problem is to use global variables for tracking when measures are implemented or not and to toggle these global variables as we reach the required thresholds. Global variables are needed because we need to know if the number of Exposed people is going up or down to know whether we need to check for the maximum or the minimum threshold. We then have an if-statement that will choose the value of parameter $\beta$ based on whether measures are being implemented. The pseudo-code for this problem thus looks as such:

\begin{minipage}{\linewidth}
\begin{lstlisting}[language=Python]
measures_implemented = False
direction = "up"
function model_with_if(_, y):
    $\vdot$
    global measures_implemented, direction
    if (direction == "up"):
        if (E > 25000):
            measures_implemented = True
            direction = "down"
    else:
        if (E < 10000):
            measures_implemented = False
            direction = "up"

    if measures_implemented:
        beta = 0.005 
    else:
        beta = 0.9
    $\vdot$
    return (dSdt, dEdt, dIdt, dRdt)
\end{lstlisting}
\end{minipage}

\subsubsection{State dependent discontinuity model in R}
\begin{figure}[h]
\centering
\includegraphics[width=0.7\linewidth]{./figures/state_discontinuity_R}
\caption{State dependent discontinuity model in R}
\label{fig:state_discontinuity_R}
\end{figure}
Figure $\ref{fig:state_discontinuity_R}$ shows how difficult this problem is with a naive treatment. We note that none of the solutions are aligned and that none of the solvers got the accurate solution (described in Section $\ref{subsection:state_with_event_detection}$) as none of the computed solutions cleanly oscillate between 10000 and 25000 with clear peaks and troughs.

We note that all the solvers, even the error-controlled ones, did not issue a warning about the integration and thus researchers may be tempted to think that their code has solved the problem to a reasonable accuracy. Having no warning also tells us that the error estimation and error control algorithms applied in all the solvers did not detect anything abnormal; the solvers return with an indication that to the given tolerances, that the provided solutions are accurate.

As we are modeling E, we expect that each graph should go from 25000 to 10000 and back to 25000 repeatedly but none of these graphs do so in the required pattern. We would also expect the solvers with error control to repeatedly reduce the step-size to satisfy the tolerance and compute solutions that align with each other but Figure $\ref{fig:state_discontinuity_R}$ shows that this is not the case.

We also note that the result for `euler' is especially poor as it reaches a maximum of 40000. This is again as expected as `euler' has no error control; `rk4', the other fixed step-size method, is also performing terribly as we see the solution it computes reach approximately 30000 in its third peak. This is all even though the space between the output points is as small as it was when performing the first experiment. Because of this, we will not run any spacing of output points experiments in this section. The step-size for these fixed-step solvers is not small enough and further step-size reductions are needed.

Another important fact to note is how poorly `radau', as shown in Figure $\ref{fig:state_discontinuity_radau_R}$, is performing. This is not a problem in the R programming environment as similar results will be seen in Python in the next section and in the Fortran code in Section $\ref{section:fortran_inaccuracies}$. It grows exponentially even after the parameter $\beta$ should be switched to 0.005. 

We note that in the way that the problem is coded, with the parameter $\beta$ depending on the state that $\beta$ is at 0.005 after the first discontinuity is met but that 'radau' is simply ignoring it. We perform an analysis with the Fortran code in $\ref{section:fortran_inaccuracies}$ to show that $\beta$ is indeed 0.005 while this exponential growth is happening. 

\begin{figure}[h]
\centering
\includegraphics[width=0.7\linewidth]{./figures/state_discontinuity_radau_R}
\caption{State dependent discontinuity model of Radau in R}
\label{fig:state_discontinuity_radau_R}
\end{figure}

We then proceed to show that sharp tolerances are not enough to solve this problem as was the case for the time-dependent discontinuity problem. We repeated the experiment at the sharpest tolerance usable before some of the solvers failed. This was at $10^{-13}$ in the R environment. We set both the absolute and relative tolerance to that value and the results are shown in Figure $\ref{fig:state_discontinuity_sharp_R}$.

\begin{figure}[h]
\centering
\includegraphics[width=0.7\linewidth]{./figures/state_discontinuity_sharp_R}
\caption{State dependent discontinuity model in R at high tolerances}
\label{fig:state_discontinuity_sharp_R}
\end{figure}

We can see from Figure $\ref{fig:state_discontinuity_sharp_R}$ that the situation has only marginally improved. None of the solvers give solutions that are in agreement and none of them cleanly oscillate between 10000 and 25000. We note that the error-controlled solvers are following the correct pattern and that until about time 20-30, some of them give solutions that are in agreement, showing that sharp tolerance error-control can step over one state-dependent discontinuity. (See the comparison against the final solution in Section $\ref{subsubsection:state_solution_comparison}$ to see that even this sharp tolerance solution is not accurate enough.)

The fixed step-size method `euler' and `rk4' are the same as in Figure $\ref{fig:state_discontinuity_R}$. This is because the tolerance does not change anything for them.

We can also point out that at such sharp tolerances, `radau' is no longer computing solutions exhibiting the abnormal behavior we saw previously. From Figure $\ref{fig:state_discontinuity_radau_sharp_R}$, we can see that it oscillates between 10000 and 25000 as it should. From supplementary experiments, we observe that `radau' starts performing in a measure that is comparable to the other solutions at about a tolerance of $10^{-9}$.

\begin{figure}[h]
\centering
\includegraphics[width=0.7\linewidth]{./figures/state_discontinuity_sharp_radau_R}
\caption{State dependent discontinuity model of Radau in R at high tolerances}
\label{fig:state_discontinuity_radau_sharp_R}
\end{figure}

\subsubsection{State dependent discontinuity model in Python}
We also perform the experiment with if-statements and global variables in Python with equally inaccurate results.

\begin{figure}[h]
\centering
\includegraphics[width=0.7\linewidth]{./figures/state_discontinuity_py}
\caption{State dependent discontinuity model in Python}
\label{fig:state_discontinuity_python}
\end{figure}
Figure $\ref{fig:state_discontinuity_python}$ shows what happens when the problem is coded with global variables and if-statements in Python. We can see that the results are similar to those in R. This is even though all solvers in Python have error control.

We note that all the solvers except `RK23' give solutions that at least oscillate between 10000 and 25000, though in completely dissimilar patterns. The solutions have peaks and troughs at different times and no warnings were given by the solvers.

The 'RK23' solver, in purple, computes a solution with a completely different pattern than the other solvers. It never reaches 25000 and only oscillates between around 10000 and 15000. 

Again, as shown in Figure $\ref{fig:state_discontinuity_radau_py}$, 'Radau' computes a solution that grows exponentially even though the parameter $\beta$ is eventually set to start an exponential decay as happens with all other solvers.

\begin{figure}[h]
\centering
\includegraphics[width=0.7\linewidth]{./figures/state_discontinuity_radau_py}
\caption{State dependent discontinuity model of Radau in Python}
\label{fig:state_discontinuity_radau_py}
\end{figure}

We then used very sharp tolerances to solve the problem but, as is the case in the R environment, none of the solvers obtained a reasonably accurate solution. The highest tolerance we could use in Python without any one method failing was $10^{-12}$. Both the absolute and relative tolerance was set to this value and Figure $\ref{fig:state_discontinuity_sharp_python}$ shows the results from this sharp tolerance experiment.

\begin{figure}[h]
\centering
\includegraphics[width=0.7\linewidth]{./figures/state_discontinuity_sharp_py}
\caption{State dependent discontinuity model in Python at sharp tolerances}
\label{fig:state_discontinuity_sharp_python}
\end{figure}

Figure $\ref{fig:state_discontinuity_sharp_python}$ shows that the results did improve. However, the solvers give solutions that are not in agreement. We note that none of the solvers are oscillating beyond 25000 as was the case with the fixed-step solvers in R. At sharp tolerances, the solutions are aligned for the first few discontinuities with only some blurring until about t=25 when the solvers give substantially different solutions. Though the pattern is correct, none of the solvers give solutions that are in agreement telling us that none got the reasonably accurate solution. Also, none of the solvers follow the solution provided in Section $\ref{subsection:state_with_event_detection}$. (See the comparison against the final solution in Section $\ref{subsubsection:state_solution_comparison}$ to see that even this sharp tolerance solution is not accurate enough.)


We note that `RK23' is now following the correct pattern in that it oscillates between 10000 and 25000 whereas it only reached 15000 at the default tolerances. 

Again, as shown in Figure $\ref{fig:state_discontinuity_sharp_radau_py}$, `Radau' begins to give reasonable solutions at these sharp tolerances; those solutions follows the pattern we are expecting but as we will show in Section $\ref{subsection:state_with_event_detection}$, they are still not reasonably accurate solutions. `Radau' starts reasonably performing well at around a tolerance of $10^{-10}$.

VI =======
add the event detection solution to the picture
====== VI

We should also note that the R and Python implementation of 'Radau' are different. The 'Radau' solver in Python is implemented in Python with the NumPy library whereas R uses the Fortran code. Thus we eliminate the possibility of a bug in the code as well as any problem stemming from the interface from R to Fortran or from Python to NumPy. The problem is simply in how the algorithm interacts with this naive implementation of the state-dependent discontinuity. In our experiments with the Fortran code, in Section $\ref{section:fortran_inaccuracies}$, the same behavior is observed.


\begin{figure}[h]
\centering
\includegraphics[width=0.7\linewidth]{./figures/state_discontinuity_sharp_radau_py}
\caption{State dependent discontinuity model of Radau in Python at sharp tolerances}
\label{fig:state_discontinuity_sharp_radau_py}
\end{figure}

\subsubsection{State dependent discontinuity model in Scilab}
We perform the experiment with if-statements and global variables in Scilab and the results are as shown in Figures $\ref{fig:state_discontinuity_scilab}$ and $\ref{fig:state_discontinuity_sharp_scilab}$.

\begin{figure}[h]
\centering
\includegraphics[width=0.7\linewidth]{./figures/state_discontinuity_scilab}
\caption{State dependent discontinuity model in Scilab}
\label{fig:state_discontinuity_scilab}
\end{figure}

Figure $\ref{fig:state_discontinuity_scilab}$ shows the same issues that we saw before, in Scilab. None of the solvers give solutions that are aligned which prompts us to conclude that none of them are getting an accurate solution. All of the solvers in Scilab have error control and we can also see that their solutions all follow the correct pattern of oscillating between 10000 and 25000. However, as we will discuss in Section $\ref{subsection:state_with_event_detection}$, none of the solutions are very accurate. We note that the spacing between points is not important in this analysis as at the current spacing, even the solvers who depend on the spacing are getting inaccurate answers.

We then repeat the experiment at sharp tolerances. Scilab's 'rkf' does not allow the use of very sharp tolerance as it has a cap of 3000 derivatives so it was omitted in this experiment. The sharpest tolerance we can use in Scilab before the other methods fail is $10^{-13}$; the results are shown in Figure $\ref{fig:state_discontinuity_sharp_scilab}$.

\begin{figure}[h]
\centering
\includegraphics[width=0.7\linewidth]{./figures/state_discontinuity_sharp_sci}
\caption{State dependent discontinuity model in Scilab with sharp tolerances}
\label{fig:state_discontinuity_sharp_scilab}
\end{figure}

Again, in Figure $\ref{fig:state_discontinuity_sharp_scilab}$ we can see that the use of sharp tolerances is not enough to force the solvers to compute accurate solutions. The solutions did improve as all the solvers seem to follow the correct pattern but none oscillate between 10000 and 25000 with clear peaks and troughs at those values respectively. For the time period between 0 to 30, the solutions all seem to show reasonable agreement but as we go further in time, all of the solutions diverge. We also note that none of the solvers compute solutions in reasonable agreement with the solution discussed in Section $\ref{subsection:state_with_event_detection}$. (See the comparison against the final solution in Section $\ref{subsubsection:state_solution_comparison}$ to see that even this sharp tolerance solution is not accurate enough.)


\subsubsection{State dependent discontinuity model in Matlab}
We perform the experiment with if-statements and global variables in Matlab and the results are as shown in Figures $\ref{fig:state_discontinuity_matlab}$ and $\ref{fig:state_discontinuity_sharp_matlab}$.

\begin{figure}[h]
\centering
\includegraphics[width=0.7\linewidth]{./figures/state_discontinuity_matlab}
\caption{State dependent discontinuity model in Matlab}
\label{fig:state_discontinuity_matlab}
\end{figure}
We see the same incorrect solutions in Matlab at the default tolerances in Figure $\ref{fig:state_discontinuity_matlab}$. The solvers do not even consistently reach 25000. We then use a sharper tolerance to see how the solvers act.

\begin{figure}[h]
\centering
\includegraphics[width=0.7\linewidth]{./figures/state_discontinuity_sharp_matlab}
\caption{State dependent discontinuity model in Matlab with sharp tolerances}
\label{fig:state_discontinuity_sharp_matlab}
\end{figure}

Figure $\ref{fig:state_discontinuity_sharp_matlab}$ shows the experiment with if-statements and global variables at sharp tolerances. We get surprisingly good solutions compared to the solutions in the previous environments. However, as we will see in Section $\ref{subsection:state_with_event_detection}$, these solutions as formed with such sharp tolerances, are computed extremely inefficiently and they are not as accurate as the solution presented in Section $\ref{subsection:state_with_event_detection}$, especially for later time periods. (See the comparison against the final solution in Section $\ref{subsubsection:state_solution_comparison}$ to see that even this sharp tolerance solution is not accurate enough.)


\subsubsection{State dependent discontinuity solutions comparison}
\label{subsubsection:state_solution_comparison}
In all the previous subsections, we have maintained that even the sharp tolerance solutions though more in agreement are not accurate. Here, we present a comparison between LSODA in Python at default tolerance, at the sharpest tolerance, and the final solution we will present shortly. We can see from Figure $\ref{comparison_state_default_sharp_event}$ that the solution both at default and the sharp tolerance does not agree with the accurate solution. We also note that the default tolerance uses 2357 function evaluation, the sharpest uses 4282 evaluations while the final solution uses 535 function evaluations.

\begin{figure}[h]
\centering
\includegraphics[width=0.7\linewidth]{./figures/comparison_state_default_sharp_event}
\caption{State dependent discontinuity model solutions comparison}
\label{fig:comparison_state_default_sharp_event}
\end{figure}

\subsection{Why the solvers fail even with sharp tolerances}
\label{subsection:state_sharp_tol_failed}
In this section we discuss why sharp tolerances were not enough to force the solvers to solve the problem in the naive way it is coded, i.e, using global variables and if-statements. 

Whenever there is a change in the value of $\beta$, the step that first encounters that change will almost always fail. As discussed in Section $\ref{subsection:effect_of_discontinuity}$, the step-size at a discontinuity will always have to be much smaller than the step-size of a step on a continuous region. Thus the first encounter of a solver with any discontinuity will always be in the context of a failed step.

During this failed step, the value of the E will cross the threshold. The global variables will thus be toggled. But then, when the solver attempts to retake the step, it will be using the wrong $\beta$ value. 

This observation is crucial as it allows us to conclude that just before discontinuity, the function evaluations should be based on the previous $\beta$ value but they are in fact using the new $\beta$ value. There is no trivial way to code this behavior in the ODE function, $f(t, y)$, if we do not know the time of the discontinuity. 

VI ====== Clarification
The solvers need to figure out how to step up to the discontinuity. Then to the left of the discontinuity, the step that we used will employ function evaluations that use the previous $\beta$, and then after the discontinuity, the next step will employ function evaluations that use the new $\beta$ value. There cannot be a step that uses both types of function evaluations
===== VI

At extremely sharp tolerances, even smaller than $10^{-12}$, the first step that encounters the discontinuity can also fail. The solver will still have to retake the step but, as discussed before, it will have to use the wrong $\beta$ value. In the next few sections, we will present the correct way to code problems with state-dependent discontinuities so that we get accurate solutions efficiently.

\subsection{Introducing event detection}
\label{subsection:intro_event_detection}
In the time-dependent discontinuity problem, we saw that if we used error-controlled software, then the solvers can work through one discontinuity at sufficiently high tolerances. We also showed that this was not the most efficient way to solve it. For the state-dependent discontinuity problem, we showed in the previous section why using the solvers with sharp tolerances will not be able to solve this problem. Because we do not know when the discontinuities occur, we cannot use the discontinuity handling technique, involving a cold restart, that we used to solve the time-dependent discontinuity problem. However, the idea that we developed in Section $\ref{subsection:time_disc_handling}$ about integrating continuous sub-problems separately and combining them into a final solution can still be applied here. 

To integrate continuous sub-problems, we need a way to detect that a threshold has been met, and then as soon as we reach such a point, we can perform a cold start. This will make the solver integrate the problem one continuous subinterval at a time. In this section, we will explain the capability of modern solvers to detect events and we will show how to encode the E(t) thresholds (either E(t)=25000 or E(t)=10000) as events so that the times at which they occur can be determined, and then we can perform a cold start when they are reached.

To perform event detection, an ODE solver will require two functions from the user: the usual ODE right-hand side function, $f(t, y)$ and another function which we will call the root function (commonly denoted by $g(t, y)$), that determines the events.

The root function is a function that, given the value of the solution to the ODE at the current step will return a real number. The ODE solution is said to have a root whenever the value of the root function is zero. The key idea is that each event must be written so that it occurs at the root of one of the root functions.

The solver calls the root function at the end of each successful step that it takes and will record its value. It will then compare the value of the root function with the corresponding value from the previous step to see if there has been a change of sign. If the value of the root-function changed sign, the solver raises a flag to say that it has detected a root and will then run a root-finding subroutine on that step until it finds the exact point where the root-function returns zero. Most solvers will then return, allowing us to perform a cold start.

Using event detection thus entails defining a function that takes the value of the ODE solution at the current point and returns a real number which is zero whenever we want it to detect an event. For example, if we want to detect whenever x is 100, it is sufficient to define (x - 100) to be the root function. In the next section, we will elaborate on how to use event detection to accurately and efficiently solve the state-dependent discontinuity problem.

We also mention that many modern solvers have event detection built-in. Thus users should just be able to use event-detection solvers from their preferred programming environments without any additional software.

\subsection{Solving the state dependent discontinuity model using event detection}
\label{subsection:state_with_event_detection}
Each toggling between the values of the parameter $\beta$ introduces a discontinuity. As none of the provided solvers are designed to solve discontinuous problems, we get the erroneous solutions reported in $\ref{subsection:naive_state_problem}$. We have seen that though sharp tolerances do result in better solutions, none of the solvers were in agreement with each other. The use of such sharp tolerances leads to inefficiencies as well. We will now present an approach using event detection that is both accurate and efficient.

The solution is to use the thresholds that we have defined in our model to define events and integrate only up to each threshold using the event detection capability of the solver. We can then cold start from there and repeat the process with another right-hand side function corresponding to the new $\beta$ value and with a different root function that encodes the next threshold we are looking for. We repeat this process until we reach the end of the time interval. This approach allows the solvers to integrate continuous sub-problems one at a time and these sub-problems can then be combined into a final solution.

For our specific problem, event detection is used as follows:
We start by solving the problem with $\beta$ at 0.9 and with a root function that detects when E is equal to 25000. Once we detect the time at which E=25000, we do a cold start. We extract the solution of the solver at the time of the event and use those values as the initial value for our next call to the solver. This next call will have $\beta$ at 0.005 and a root function that detects a root when E=10000. We again integrate up to that new threshold and cold start when we reach it. The new cold start will have $\beta$ at 0.9 and the root function looking for 25000 as the event. This is repeated until we reach the desired end time.

The pseudo-code is as follows:

\begin{minipage}{\linewidth}
\centering
\begin{lstlisting}[language=Python]
function model_no_measures(t, y):
    beta = 0.9
    // code to get dSdt, dEdt, dIdt, dRdt
    return (dSdt, dEdt, dIdt, dRdt)

function root_25000(t, y):
    E = y[1]
    return E - 25000

function model_with_measures(t, y):
    beta = 0.005
    // code to get dSdt, dEdt, dIdt, dRdt
    return (dSdt, dEdt, dIdt, dRdt)

function root_10000(t, y):
    E = y[1]
    return E - 10000

res = array()
t_initial = 0
y_initial = (S0, E0, I0, R0)
while t_initial < 180:
    tspan = [t_initial, 180]
    if (measures_implemented):
        sol = ode(model_with_measures, tspan, y_initial,
            events=root_10000)
        measures_implemented = False
    else:
        sol = ode(model_no_measures, tspan, y_initial,
            events=root_25000)
        measures_implemented = True
    t_initial = extract_last_t_from_sol(sol)
    y_initial = extract_last_row_from_sol(sol)
    res = concatenate(res, sol)

// use res as the final solution
\end{lstlisting}
\end{minipage}

Some programming environments, such as Python, by default, do not stop the integration when the first event is detected. To do a cold start, we need the solver to stop at events, and to make this happen, in some programming environments we need to set appropriate flags. 

\subsubsection{Solving state dependent discontinuity model in R}
\begin{figure}[h]
\centering
\includegraphics[width=0.7\linewidth]{./figures/solve_state_discontinuity_R}
\caption{Solving state discontinuity model in R}
\label{fig:solve_state_discontinuity_R}
\end{figure}
Several of the solvers in R have event detection capabilities. These are: `adams', `bdf', `lsoda', `radau', and will be used in this section to solve the model using the approach described in the previous subsection. From Figure $\ref{fig:solve_state_discontinuity_R}$, we can see that all the solvers give solutions that are in agreement except `Radau'. This is in contrast with what happened previously when we were integrating a discontinuous problem, even at sharp tolerances. 

The case of `Radau' is interesting as it was giving a poor quality solution at the default tolerances, without event detection but it is now giving at least a solution that is exhibiting a correct pattern. We note that at high tolerances `Radau' with event detection approach the results from the other solvers, as shown in Figure $\ref{fig:solve_state_discontinuity_sharp_R}$. We will also note the terrible performance of Radau in Table $\ref{tab:state_discontinuity_R}$.

\begin{figure}[h]
\centering
\includegraphics[width=0.7\linewidth]{./figures/solve_state_discontinuity_sharp_R}
\caption{Solving state discontinuity model at sharp tolerances in R}
\label{fig:solve_state_discontinuity_sharp_R}
\end{figure}

We will show in Table $\ref{tab:state_discontinuity_R}$ that introducing event detection also made the solvers significantly more efficient while giving us better results.

We note that it is unfair to compare the efficiency of the solvers at the default tolerances with the efficiency of the solvers when they use event detection's as the results for the former are inaccurate.

\begin{table}[h]
\caption {R state discontinuity model} 
\label{tab:state_discontinuity_R}
\begin{center}
\begin{tabular}{ c c c c c } 
method & no event & no event with sharp tol. & with event detection & with event detecton at sharp tol.\\ 
lsoda & 2135 & 4658 & 1248 & 3435 \\
radau & 1002 & 21835 & 2151 & 14681\\
bdf & 3300 & 9803 & 1678 & 7963\\
adams & 1368 & 3467 & 817 & 2689\\
\end{tabular}
\end{center}
\end{table}

We can see from Table $\ref{tab:state_discontinuity_R}$ that with event detection we are gaining an improvement of around 1000 function evaluations for `lsoda', 13000 in 'radau', 4000 in 'bdf', and 1300 in 'adams' while having more accuracy. This significant decrease in the number of function evaluations will lead to much faster CPU times, especially when the right-hand side function, $f(t, y)$ is more complex.

Also, we can see from the table that the solvers use fewer function evaluations compared with event detection than without event detection at the default tolerances.

We also note that the Fortran code for Radau does not have event detection and that event detection was added through the R interface.

\subsubsection{Solving state dependent discontinuity model in Python}
\begin{figure}[h]
\centering
\includegraphics[width=0.7\linewidth]{./figures/solve_state_discontinuity_py}
\caption{Solving state dependent discontinuity model in Python}
\label{fig:solve_state_discontinuity_py}
\end{figure}
All the solvers in Python have event detection and thus all will be used in this part of the study. In Python, $solve\_ivp()$ does not stop when an event is detected by default. We thus need to set the terminal flag of the root functions.
(Example: $root\_10000.terminal = True$).
Again, Figure $\ref{fig:solve_state_discontinuity_py}$ shows that all the solvers give solutions that are in agreement, suggesting that this is the correct solution. This is different from our results even at sharp tolerances. We will also see that this is a much more efficient approach across all the solvers.

We note that even `Radau' is in agreement with the solution using default tolerances in Python and R when using event detection even though the "Radau" implementation in Python is different from the one in R. Python uses a Python implementation while R uses the Fortran code.

As is the case with R, we cannot compare the default tolerance efficiency data to the event detection efficiency data as the former corresponds to inaccurate results. So, in Table $\ref{tab:state_discontinuity_Py}$, we compare the sharp tolerance efficiency data with the data from the event detection computation.

\begin{table}[h]
\caption {Python state discontinuity model} \label{tab:state_discontinuity_Py}
\begin{center}
\begin{tabular}{ c c c c } 
method & no event & no event with sharp tol. & with event detection \\ 
lsoda & 2357 & 4282 & 535 \\
bdf & 2301 & 11794 & 808 \\
radau & 211 & 74723 & 990 \\
rk45 & 1484 & 17648 & 674 \\
dop853 & 11129 & 21131 & 1514 \\
rk23 & 4307 & 246644 & 589 \\
\end{tabular}
\end{center}
\end{table}

Table $\ref{tab:state_discontinuity_Py}$ shows that the number of function evaluations when the solvers use event detection is far less when they do not; `LSODA' used around 3000 fewer function evaluations, `BDF' used 11000 less, `Radau' used 74000 less, `RK45' used 17000 less, `DOP853' used 20000 less and `RK23' used 246000 less. The reduction in CPU times from this will be significant across all the solvers, especially with a more complex right-hand side function.

What is more surprising is that the solvers with event detection also perform better than the solver using no event detection with default tolerance code without event detection. In all solvers except `Radau', event detection performed better than the default tolerance code without event detection. We note that `Radau' at default tolerance gives an extremely inaccurate solution and thus its better performance here should not be trusted.

\subsubsection{Solving state discontinuity dependent model in Scilab}
\begin{figure}[h]
\centering
\includegraphics[width=0.7\linewidth]{./figures/solve_state_discontinuity_scilab}
\caption{Solving state discontinuity model in Scilab}
\label{fig:solve_state_discontinuity_scilab}
\end{figure}
There is only one solver with root functionality in Scilab; it is `lsodar', the root-finding version of `lsoda'. Judging from the solutions we obtained from Python and R, it seems that `lsodar' gave a correct solution as well. It oscillates in the correct pattern and goes sharply between 10000 and 25000.

\begin{table}[h]
\caption {Scilab state discontinuity model} \label{tab:state_discontinuity_scilab}
\begin{center}
\begin{tabular}{ c c c c } 
method & no event & no event with sharp tol. & with event detection \\ 
lsoda & 2794 & 4636 & 1327 \\
\end{tabular}
\end{center}
\end{table}

From Table $\ref{tab:state_discontinuity_scilab}$, we can see that the root-finding code use fewer function evaluations that `lsoda' both at sharp and default tolerances.

\subsubsection{Solving state dependent discontinuity model in Matlab}
\begin{figure}[h]
\centering
\includegraphics[width=0.7\linewidth]{./figures/solve_state_discontinuity_matlab}
\caption{Solving state discontinuity model in Matlab}
\label{fig:solve_state_discontinuity_matlab}
\end{figure}
Both $pde45$ and $ode15s$ have an event detection capability. We applied event detection to model the problem with the solvers in the Matlab environment and the results are shown in Figure $\ref{fig:solve_state_discontinuity_matlab}$. We remember that the solutions in Matlab without event detection were surprisingly accurate but were in disagreement with each other at points further in time. We can see that with event detection, the solutions are all in agreement at the default tolerances even at points further in time. We also see, in Table $\ref{tab:state_discontinuity_matlab}$, that the use of event detection is also more efficient than without event detection.

\begin{table}[h]
\caption {Matlab state discontinuity model problem} \label{tab:state_discontinuity_matlab}
\begin{center}
\begin{tabular}{ c c c c } 
method & no event & no event with sharp tol. & with event detection \\ 
ode45 & 2023 & 22411 & 859 \\
ode15s & 1397 & 11550 & 620 \\
\end{tabular}
\end{center}
\end{table}

We can see in Table $\ref{tab:state_discontinuity_matlab}$ that the computation with event detection uses fewer function evaluations than both the code without event detection at default and sharp tolerances. We see that the computations with sharp tolerances, although they give acceptable solutions, use 20000 more function evaluations in $ode45$ than the computation with event detection and 11000 in the case of $ode15s$ than the computation with event detection.

\subsection{Efficiency data and tolerance study for the state dependent discontinuity problem}
\label{subsection:state_tolerance_study}
In this section, we will investigate how sharpening the tolerance improves the results in the case of the non-event detection experiment. We will also investigate coarsening the tolerance with event detection to show how coarse a tolerance we can use while getting acceptable results.

We will perform this analysis on LSODA across R, Python, and Scilab, as they appear to use the same source code, and with R and Python versions of DOPRI5 which do not use the same code but do use the same Runge-Kutta pair and with the Scilab version of RKF45 which is not the same code, nor the same pair but is a Runge-Kutta pair of the same order. We also use $ode45$ in Matlab as it is an implementation of DOPRI5 in Matlab. 

\subsubsection{Comparing LSODA across platforms for state discontinuous problem}
In this section, we use the R version of LSODA at multiple tolerances. We set both the relative and the absolute tolerance to a particular value and analyze the solution.

We know that without event detection, LSODA does not give accurate results even at very sharp tolerances. We will also examine how coarse we can set the tolerance to still have the event detection computation yield reasonable results.

\subparagraph{State dependent discontinuity LSODA tolerance study in R}

\begin{figure}[h]
\centering
\includegraphics[width=0.7\linewidth]{./figures/tolerance_state_lsoda_no_event_R}
\caption{State dependent discontinuity model tolerance study on the R version of LSODA without event detection}
\label{fig:tolerance_state_lsoda_no_event_R}
\end{figure}

\begin{figure}[h]
\centering
\includegraphics[width=0.7\linewidth]{./figures/tolerance_state_lsoda_with_event_R}
\caption{State dependent discontinuity model tolerance study on the R version of LSODA with event detection}
\label{fig:tolerance_state_lsoda_with_event_R}
\end{figure}

Figure $\ref{fig:tolerance_state_lsoda_no_event_R}$ shows that LSODA applied to the same problem at different tolerances gives vastly different results. We would expect the solutions at the sharper tolerances to be along very similar curves but that is not the case. The computation is suffering from the fact that the first step that encounters a discontinuity fails while still switching the global variables. This further supports our statement that for any state-dependent discontinuity, we cannot get reasonable results simply by sharpening the tolerance.

From Figures $\ref{fig:tolerance_state_lsoda_with_event_R}$ and $\ref{fig:tolerance_state_lsoda_no_event_R}$, we can see the clear advantage of using event detection. Event detection even allows us to use very coarse tolerances while solving the problem to a reasonable accuracy. Event detection allows us to use tolerances of $10^{-3}$ and sharper to get reasonable results while the computation without event detection still failed at a tolerance of $10^{-13}$. We also analyze the differences in efficiency between the two codes in Table $\ref{tab:tolerance_state_discontinuity_lsoda_R}$.

\begin{table}[h]
\caption {R version of LSODA applied to state discontinuity model tolerance study} \label{tab:tolerance_state_discontinuity_lsoda_R} 
\begin{center}
\begin{tabular}{ c c c }
tolerance & no event detection & with event detection \\
1e-01 & 675 & 560 \\
1e-02 & 1856 & 522 \\
1e-04 & 1863 & 752 \\
1e-06 & 2135 & 1248 \\
1e-07 & 2676 & 1874 \\
1e-08 & 2730 & 2060 \\
1e-10 & 3337 & 2604 \\
1e-11 & 3603 & 3054 \\
\end{tabular}
\end{center}
\end{table}

Table $\ref{tab:tolerance_state_discontinuity_lsoda_R}$ shows a decrease in the number of function evaluations across all tolerances which will translate into faster CPU times when the right-hand side function is more complex. We note that the comparison is unfair as the computations without event detection do not give a reasonably accurate answer. Furthermore, the latter computation uses more function evaluations. This supports our conclusion that event detection is the appropriate way to solve state-dependent discontinuity problems.

\subparagraph{State dependent discontinuity model LSODA tolerance study in Python}
In this section, we use the Python version of LSODA at multiple tolerances to see how it performs. We set both the absolute and relative tolerance to a particular value. 

We note that LSODA without event detection even at very sharp tolerances in Python was still giving accurate results but we will see how the solutions change as the tolerance is increased. 

We will also show that coarse tolerances can be used with the computation that uses event detection. 

\begin{figure}[h]
\centering
\includegraphics[width=0.7\linewidth]{./figures/tolerance_state_lsoda_no_event_py}
\caption{State dependent discontinuity model tolerance study on the Python version of LSODA without event detection}
\label{fig:tolerance_state_lsoda_no_event_py}
\end{figure}

\begin{figure}[h]
\centering
\includegraphics[width=0.7\linewidth]{./figures/tolerance_state_lsoda_with_event_py}
\caption{State dependent discontinuity model tolerance study on the Python version of LSODA with event detection}
\label{fig:tolerance_state_lsoda_with_event_py}
\end{figure}

Again Figure $\ref{fig:tolerance_state_lsoda_no_event_py}$ exposes that LSODA applied to the same problem at different tolerances give substantially different results. We would expect the computations at the sharper tolerances to give substantially similar results but this is not the case. This confirms that, even at sharp tolerances, the step-size when first encountering the discontinuity is still too big. That first step will fail but will still switch the global variables. As a result, this problem cannot be solved sufficiently accurately by sharpening the tolerance.

From Figures $\ref{fig:tolerance_state_lsoda_with_event_py}$ and $\ref{fig:tolerance_state_lsoda_no_event_py}$, we can see that the addition of event detection allows for the use of a coarser tolerance. We also note that the computations with event detection blur as we go further in time. This is because the coarser tolerance computations are not giving a sufficiently accurate solution. In Python, it is at a tolerance of $10^{-4}$ and sharper that we get reasonably accurate results. 

We analyse the efficiency of the computations in Table $\ref{tab:tolerance_state_discontinuity_lsoda_py}$. We must note that this analysis is unfair as the computation without event detection does not give an accurate solution to the problem. Still, we will see that the event detection computation uses fewer function evaluations while getting a more accurate answer.

\begin{table}[h]
\caption {Python version of LSODA applied to state discontinuity model tolerance study} \label{tab:tolerance_state_discontinuity_lsoda_py} 
\begin{center}
\begin{tabular}{ c c c }
tolerance & no event detection & with event detection \\
0.1 & 1207 & 425 \\
0.01 & 1627 & 454 \\
0.0001 & 1968 & 689 \\
1e-06 & 2122 & 1305 \\
1e-07 & 2684 & 1807 \\
1e-08 & 2730 & 2099 \\
1e-10 & 3337 & 2639 \\
1e-11 & 3603 & 3098 \\
\end{tabular}
\end{center}
\end{table}

\subparagraph{State dependent discontinuity model LSODA tolerance study in Scilab}

We perform the same experiment in Scilab. We set the absolute and relative tolerance to the same values and run the solvers. For the different tolerance values, we plot the solutions and analyze how the solutions computed without event detection change as the tolerance is sharpened; we also examine how coarse a tolerance we can use with the event detection solvers.

\begin{figure}[h]
\centering
\includegraphics[width=0.7\linewidth]{./figures/tolerance_state_lsoda_no_event_sci}
\caption{State dependent discontinuity model tolerance study on the Scilab version of LSODA without event detection}
\label{fig:tolerance_state_lsoda_no_event_sci}
\end{figure}

\begin{figure}[h]
\centering
\includegraphics[width=0.7\linewidth]{./figures/tolerance_state_lsoda_with_event_sci}
\caption{State dependent discontinuity model tolerance study on the Scilab version of LSODA with event detection}
\label{fig:tolerance_state_lsoda_with_event_sci}
\end{figure}

Again, Figure $\ref{fig:tolerance_state_lsoda_no_event_sci}$ exposes the behavior whereby the same solver applied to the same problem at different tolerances gives substantially different results. We would expect the code at the sharper tolerances to give very similar curves but clearly, LSODA even at sharp tolerances does not.

From Figure $\ref{fig:tolerance_state_lsoda_with_event_sci}$, we can see that the use of the event detection allows us to use a smaller tolerance. We can use a tolerance of $10^{-3}$ and still get an accurate answer whereas, without event detection, even tolerance of $10^{-12}$ is not sufficient.

\begin{table}[h]
\caption {Scilab version of LSODA applied to state discontinuity model tolerance study} \label{tab:tolerance_state_discontinuity_lsoda_scilab} 
\begin{center}
\begin{tabular}{ c c c }
tolerance & no event detection & with event detection \\
0.1 & 1141 & 287 \\
0.01 & 1606 & 262 \\
0.0001 & 1968 & 523 \\
0.000001 & 2122 & 983 \\
0.0000001 & 2684 & 1307 \\
1.000D-08 & 2730 & 1567 \\
1.000D-10 & 3380 & 1963 \\
1.000D-11 & 3603 & 2331 \\
\end{tabular}
\end{center}
\end{table}

\subsubsection{Comparing Runge-Kutta pairs across platforms for state discontinuous problem}
In this section, we consider solvers based on Runge-Kutta pairs of the same order: DOPRI5 in R aliased as `ode45', DOPRI5 in Python aliased as `RK45', RKF45 in Scilab aliased as `rkf' and $ode45$ in Matlab.

We remember that without event detection, none of these solvers across the platforms solved the problem correctly even with sharp tolerances. We will show what happens to these solvers as the tolerance is sharpened. We also coarsen the tolerance for the case where solvers use event detection where that is possible to see how coarse the tolerance can be while still obtaining sufficient accuracy.

\subparagraph{Tolerance study on state discontinuity using the R version of DOPRI5}
The R version of DOPRI5 does not have event detection but we still perform the experiment on this solver without event detection. We pick several values for the absolute and relative tolerances and run the solvers. In so doing we see how the code performs as the tolerance is sharpened. 

\begin{figure}[h]
\centering
\includegraphics[width=0.7\linewidth]{./figures/tolerance_state_rk45_no_event_R}
\caption{State dependent discontinuity model tolerance study on the R version DOPRI5 without event detection}
\label{fig:tolerance_state_rk45_no_event_R}
\end{figure}

From Figure $\ref{fig:tolerance_state_rk45_no_event_R}$, we see that LSODA, whereby solver applied to the same problem with different tolerances, gives significantly different solutions. The global variables will be switched during the first step that encounters the discontinuity and thus the problem cannot be solved accurately. This strengthens our conclusion that the problem with state-dependent discontinuities cannot be solved without event detection.

We then report on the efficiency data for this case in Table $\ref{tab:tolerance_state_discontinuity_rk45_R}$. 

\begin{table}[h]
\caption {R version of DOPRI5 state discontinuity model tolerance study} \label{tab:tolerance_state_discontinuity_rk45_R} 
\begin{center}
\begin{tabular}{ c c }
tolerance & no event detection \\
1e-01 & 1082 \\
1e-02 & 1142 \\
1e-04 & 2014 \\
1e-06 & 2027 \\
1e-07 & 2193 \\
1e-08 & 2919 \\
1e-10 & 5194 \\
1e-11 & 7690 \\
\end{tabular}
\end{center}
\end{table}

\subparagraph{Tolerance study on state discontinuity using the Python version of DOPRI5}
We perform the same experiment in Python. The absolute and relative tolerances are set to a range of values and the solver is run both with and without event detection. We report on how the code performs as the tolerance is increased in the case without event detection. Since the Python version of DOPRI5 has event detection, we will see how coarse the tolerance can be set while still giving us an accurate solution. We must note that the solver crashes if we ask for a tolerance of 0.1.

\begin{figure}[h]
\centering
\includegraphics[width=0.7\linewidth]{./figures/tolerance_state_rk45_no_event_py}
\caption{State dependent discontinuity model tolerance study on the Python version of DOPRI5 without event detection}
\label{fig:tolerance_state_rk45_no_event_py}
\end{figure}

\begin{figure}[h]
\centering
\includegraphics[width=0.7\linewidth]{./figures/tolerance_state_rk45_with_event_py}
\caption{State dependent discontinuity model tolerance study on the Python version of DOPRI5 with event detection}
\label{fig:tolerance_state_rk45_with_event_py}
\end{figure}

In Figure $\ref{fig:tolerance_state_rk45_no_event_py}$, we can see that even at sharp tolerances, the solver is not able to compute a reasonably accurate solution. The global variables are switched at the first encounter with the discontinuity and thus the problem cannot be solved accurately simply by decreasing the tolerance.

In contrast, when using event detection, the code can use very coarse tolerances. We can see that a tolerance of $10^{-4}$ is sharp enough to solve the given problem accurately; the blurring that occurs is due to the coarser tolerances. We present the efficiency data in Table $\ref{tab:tolerance_state_discontinuity_rk45_py}$ to show how the code with event detection is also far more efficient.

\begin{table}[h]
\caption {The Python version of DOPRI5 state discontinuity model tolerance study} \label{tab:tolerance_state_discontinuity_rk45_py} 
\begin{center}
\begin{tabular}{ c c c }
tolerance & no event detection & with event detection \\
0.01 & 1400.0 & 664.0 \\
0.0001 & 8462.0 & 806.0 \\
1e-06 & 6248.0 & 1232.0 \\
1e-07 & 6848.0 & 1754.0 \\
1e-08 & 7082.0 & 2354.0 \\
1e-10 & 10262.0 & 5066.0 \\
1e-11 & 13058.0 & 7688.0 \\
\end{tabular}
\end{center}
\end{table}

We can see in Table $\ref{tab:tolerance_state_discontinuity_rk45_py}$ that across all the different tolerances, the solver with event detection requires fewer function evaluations, around several thousand fewer for the sharper tolerances. 

\subparagraph{State dependent discontinuity RKF45 tolerance study in Scilab}
Scilab uses RKF45 which is a different Runge-Kutta pair from what is used in DOPRI5 but the pairs have the same order. It does not have event detection but we can still perform the experiment on the solver without event detection. We pick several values for the absolute and relative tolerances and run the solvers. In so doing we see how the solver performs as the tolerance is sharpened. 

The Scilab version of `rkf' can only integrate up to time 90 as it has a hard cap of 3000 derivative evaluations but this is enough to see that even at sharper tolerances, the solutions are not in agreement. Figure $\ref{fig:tolerance_state_rk45_no_event_sci}$ shows that the problem cannot be solved by simply using sharper tolerances. We can conclude that event detection is required. 

\begin{figure}[h]
\centering
\includegraphics[width=0.7\linewidth]{./figures/tolerance_state_rk45_no_event_sci}
\caption{State dependent discontinuity model tolerance study on the Scilab version of RKF45 without event detection}
\label{fig:tolerance_state_rk45_no_event_sci}
\end{figure}

\begin{table}[h]
\caption {The Scilab version of RKF45 State Discontinuity tolerance study} \label{tab:tolerance_state_discontinuity_rk45_scilab} 
\begin{center}
\begin{tabular}{ c c }
tolerance & no event detection \\ 
0.1 & 547 \\
0.01 & 732 \\
0.001 & 1294 \\
1e-4 & 1956 \\
1e-5 & 2364 \\
1e-6 & 2662 \\
1e-7 & 2802 \\
\end{tabular}
\end{center}
\end{table}

\subparagraph{Tolerance study on state discontinuity using the Matlab version of DOPRI5}
We apply different tolerances to the state problem with and without event detection on the $ode45$ function which is a Matlab implementation of DOPRI5.

\begin{figure}[h]
\centering
\includegraphics[width=0.7\linewidth]{./figures/tolerance_state_rk45_no_event_matlab}
\caption{State dependent discontinuity model tolerance study on the Matlab version of DOPRI5 without event detection}
\label{fig:tolerance_state_rk45_no_event_matlab}
\end{figure}

From Figure $\ref{fig:tolerance_state_rk45_no_event_matlab}$, we can see that the solution obtained with a tolerance of 0.1 is of poor quality without event detection. It does not follow the correct pattern of oscillating between 10000 and 25000. The computations of the other tolerances follow the correct pattern but are not in agreement.

\begin{figure}[h]
\centering
\includegraphics[width=0.7\linewidth]{./figures/tolerance_state_rk45_with_event_matlab}
\caption{State dependent discontinuity Model tolerance study on the Matlab version of DOPRI5 with event detection}
\label{fig:tolerance_state_rk45_with_event_matlab}
\end{figure}
In Figure $\ref{fig:tolerance_state_rk45_with_event_matlab}$, we can see that the computations corresponding to most tolerances give solutions that are in agreement. A tolerance of 0.1 now follows the correct pattern but is not in agreement with the other tolerances at further points in time. For tolerances of $10^{-2}$ and sharper, we get accurate solutions. We also see show event detection allows us to use fewer function evaluations.

\begin{table}[h]
\caption {Matlab DOPRI5 state discontinuity model tolerance study} \label{tab:tolerance_state_discontinuity_rk45_matlab} 
\begin{center}
\begin{tabular}{ c c c }
tolerance & no event detection & with event detection \\
0.1 & 415 & 650 \\
0.01 & 1339 & 661 \\
0.0001 & 4891 & 901 \\
1e-06 & 5803 & 1411 \\
1e-07 & 7225 & 1873 \\
1e-09 & 9739 & 4039 \\
1e-10 & 12385 & 6043 \\
1e-11 & 16357 & 9277 \\
\end{tabular}
\end{center}
\end{table}

Table $\ref{tab:tolerance_state_discontinuity_rk45_matlab}$, although being an unfair comparison since the solver without event detection did not give accurate solutions, shows that this way of solving the problem is also less efficient. At the tolerance of 0.1, the smaller number of function evaluations for the solver without event detection is not relevant since the solution at a tolerance of 0.1 is very inaccurate. At all the other tolerances, the code with event detection is both more accurate and more efficient, usually using less than half the number of function evaluations.

