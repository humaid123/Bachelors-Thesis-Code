\section{Summary}
\label{section:summary}
In this chapter, we considered the numerical solution of two Covid-19 models based on a standard SEIR model. The models include discontinuities associated with interventions introduced to slow down the spread of the virus. We were particularly interested in investigating the performance of standard ODE solvers available in computational platforms.

We reported on the stability and discontinuity issues associated with the models. We showed how stability issues affect the accuracy of the computed solutions even if there is only a relatively small change in the initial values. We showed how discontinuities reduce the efficiency of the solvers and presented a straightforward way to detect that the problem is discontinuous.

We then used ODE solvers in R, Python, Scilab, and Matlab to solve the two Covid-19 problems, one with a time-dependent discontinuity and one with a state-dependent discontinuity. A key assumption for both models is that we first consider reasonable implementations that might typically be employed by a researcher. This includes fixed-step size solvers as well as implementations based on the introduction of if-else statements into the functions that define the ODE systems. 

For the time-dependent discontinuity problem, we have shown that error-control ODE solvers can step over the one discontinuity that is present with sufficiently sharp tolerances while fixed step-size solvers cannot. We have shown that although error-controlled solvers can solve the problem to reasonable accuracy if the tolerance is sufficiently sharp, the use of discontinuity handling in the form of cold starts leads to more efficient solutions that can be obtained using coarser tolerances. We therefore recommend that if the time of a discontinuity is known, cold starts at these times should be employed as they result in more accurate, more efficient solutions that can be obtained at coarser tolerances.

For the state-dependent discontinuity problem, we have shown that even error control solvers cannot successfully step over multiple state-dependent discontinuities. We then introduced event detection and showed how it can be used to accurately and efficiently solve state-dependent discontinuity problems by encoding the intervention imposition and relaxation thresholds as events and applying cold starts. We conclude that using event detection provides an efficient and accurate way to solve such problems.

From the usage of the different packages, we also found a certain inconsistency. We noted that R and Scilab do not use the interpolation capabilities for some of their solvers by default. We would advise software implementers to take advantage of the capabilities of the solvers to use interpolation. Using the method of forcing the solver to integrate exactly to given output points reduces the efficiency of the solver.

We recommend using some form of discontinuity handling rather than introducing an if-statement into the right-hand side function that defines the ODE wherever applicable.

When a researcher has a problem that has a time-dependent discontinuity that occurs at a known time, we recommend that they use the form of discontinuity handling presented in this chapter. Using cold starts allows the researcher to integrate continuous subintervals of the problem in separate calls leading to efficient and accurate solutions.

When a researcher has a problem that has a state-dependent discontinuity, they should identify the conditions under which these discontinuities occur and use event detection with these conditions as events. They can then cold start at each event and integrate continuous subintervals of the problem in separate calls to the solvers. This leads to efficiency and accuracy that is not possible using a simple implementation. 

%\subsection{Future Work}
%\label{subsection:future_work}
%In Section $\ref{subsection:naive_state_problem}$, we show that `Radau' exhibits an unusual behavior when solving the state-dependent problem. Further analysis needs to be done on the algorithm itself as two different implementations of the algorithm in R and Python and the Fortran code itself gave similarly poor quality solutions.

% We also need to perform a further analysis on why the Scilab `lsodar' solver comes up with a solution with 13 peaks only when the other solvers provide a solution with 18 peaks to the state-dependent discontinuity problem.
%
%We also propose to do the same analysis on Covid-19 PDE models with discontinuities to see how well PDE solvers can handle these models. We will also investigate the use of a PDE solver with event detection for these models.


